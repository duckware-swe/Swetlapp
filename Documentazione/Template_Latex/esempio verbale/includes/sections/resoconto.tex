\clearpage
\section{Resoconto}
\subsection{Esaminazione Studio di Fattibilità}
	Il documento in questione viene letto da tutti i partecipanti del team duckware, vengono fatti notare alcuni punti di stesura, verificato e approvato.
	 
\subsection{Email a Zero12}
	Dopo un confronto collettivo il team ha deciso di inviare una mail a Zero12 S.p.A.\\
	Nella mail viene fatta una presentazione del gruppo, esposti alcuni dubbi e richiesto un incontro con il proponente del capitolato.\\
	Di seguito viene lasciato il contenuto della mail:
	\begin{quote}
	\emph{
	 	Alla cortese attenzione di Stefano Dindo.\\[0.25cm]
		Salve, con la presente mail il gruppo 3, del corso di Ingegneria del Software - Informatica dell'Università degli Studi di Padova, vuole presentarsi e proporsi per la realizzazione del capitolato d'appalto proposto in sede universitaria il 16/11/2018.\\[0.25cm]
		Il gruppo di progetto, chiamato duckware, è composto da 7 studenti: Sonia, Matteo, Luca, Alberto, Andrea, Pardeep e Alessandro.\\[0.25cm]
		In merito ad un primo studio di fattibilità duckware desidererebbe chiarire alcuni punti:
		\begin{itemize}
		\item scelta sull'uso di tecnologie per la realizzazione dell'app smartphone e web, nel particolare se possibile realizzare le app Android e iOS con un framework crossplatform nativo. Esso permetterebbe la realizzazione di un'applicazione per entrambe le piattaforme
		\item come, quando necessario, testare il software
		\end{itemize}
		Si chiede inoltre la disponibilità, se possibile, di un incontro, anche nella vostra sede, per poter discutere faccia faccia ed entrare nel dettaglio sulle scelte da adottare.\\[0.25cm]
		In attesa di una vostra risposta\\
		Vi ringraziamo \\
		Team Duckware\\
	}
	\end{quote}

\subsection{Scadenze e suddivisione documenti}
	Viene fissato un obbiettivo comune e unanime da parte del team nel voler presentare tutti la documentazione del RR con scadenza di consegna il 14-01-2019.\\
	La suddivisione dei documenti viene assegnata nel seguente modo:\\\\
	\textbf{Analisi dei Requisiti}\\[0.25cm]
		\renewcommand{\arraystretch}{1.5}
		\begin{tabular}{  l | c }
			\hline
			\textbf{Punto del capitolo}&\textbf{AT(@)}\\
			\hline
			\emph{1 - Introduzione} & \multirow{2}{*}{Luca Stocco}\\ \emph{2 - Descrizione generale}  \\
			\hline
			\multirow{4}{*}{\emph{3 - Casi d'Uso}} & Alessandro Pegoraro\\&Pardeep Singh\\&Andrea Pavin\\&Matteo Pellanda\\
			\hline
			\multirow{2}{*}{\emph{4 - Requisisti}} &  Alberto Miola\\&Sonia Menon\\
			\hline
		\end{tabular}
	\\[0.7cm]
	\textbf{Piano di Qualifica}\\[0.25cm]
	\begin{tabular}{  l | c }
		\hline
		\textbf{Punto del capitolo}&\textbf{AT(@)}\\
		\hline
		\emph{1 - Introduzione} & Luca Stocco  \\ \emph{2 - Qualità di processo} & Matteo Pellanda\\
		\hline
		\emph{4 - Specifiche dei test} & Andrea Pavin  \\ \emph{5 - Misure e metriche per i test} & Alberto Miola\\
		\hline
		\multirow{2}{*}{\emph{3 - Qualità di prodotto}} & Pardeep Singh\\&Alessandro Pegoraro\\
		\hline
	   	\emph{6 - Resoconto + Standard ISO} & Sonia Menon\\
		\hline
	\end{tabular}
	\\[1cm]
	\textbf{Piano di Progetto}\\[0.25cm]
	\begin{tabular}{  l | c }
		\hline
		\textbf{Punto del capitolo}&\textbf{AT(@)}\\
		\hline
		\emph{1 - Introduzione} & Andrea Pavin  \\ \emph{2 - Analisi dei rischi} & Pardeep Singh\\ \emph{A - Organigramma} & Alessandro Pegoraro\\
		\hline
		\emph{4 - Modello di sviluppo} & Luca Stocco  \\ \emph{5 - Pianificazione} & Matteo Pellanda\\
		\hline
		\emph{5 - Suddivisione risorse} e preventivo & Alberto Miola\\ \emph{6 - Consuntivo di periodo e preventivo a finire} & Sonia Menon\\
		\hline
	\end{tabular}
	\\
\subsection{Discussione linguaggi e tecnologie}
	Al team vengono proposti i seguenti linguaggi e tecnologie per la realizzazione delle piattaforme richieste dal committente.
	\begin{itemize}
		\item App per iOS e Android
			\begin{itemize}
				\item Delphi
				\item Delphi® - Overview IDE 
				\item Firemonkey framework 
			\end{itemize}
		\item Piattaforma Web
			\begin{itemize}
				\item HTML5 
				\item CSS3
				\item PHP
				\item Javasript
			\end{itemize}
	\end{itemize}
