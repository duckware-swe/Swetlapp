\section{Introduzione}
\subsection{Scopo del documento}
Lo scopo di questo documento è quello di fissare tutte le regole e le metodologie da applicare durante la realizzazione del progetto di modo che il gruppo possa lavorare seguendo certi standard ben definiti. In particolare, verranno definiti gli strumenti da utilizzare e le procedure da applicare durante le varie fasi dello sviluppo del \markg{software}. In tal modo ci sarà una efficiente collaborazione tra i membri.
\subsection{Scopo del prodotto}
L'obiettivo del prodotto è la realizzazione di un'applicazione per smartphone, nello specifico per la piattaforma \markg{Android} OS, che permetta la creazione di \markg{workflow} per l'assistente vocale \markg{Amazon} \markg{Alexa}. Il \markg{back-end} sarà realizzato in \markg{Java} opportunamente integrato con le \markg{API} di \markg{AWS}, per il \markg{front-end} verrà utilizzato \markg{XML} per stabilire i layout e \markg{Java} per gestirne il comportamento. Si parlerà del \markg{front-end} dell'assistente vocale riferendosi a \markg{VUI}(voice user interface).
\subsection{Glossario}
Nel documento ci sono dei termini con un significato ambiguo a seconda del contesto nel quale sono stati utilizzati. Per ovviare a questo problema è presente il documento \emph{Glossario v1.0.0} che conterrà una lista di termini con la specifica descrizione del suo significato. Un qualsiasi termine presente nel glossario verrà indicato in questo documento scritto in corsivo con una G a pedice che seguirà la parola in questione.
\subsection{Riferimenti utili}
\subsubsection{Riferimenti normativi}
\begin{itemize}
	\item \markg{ISO}/IEC
	\begin{itemize} 
		\item \href{http://www.colonese.it/SviluppoSw_Standard_ISO15504.html}{\markg{ISO}/IEC 15504}
	\end{itemize}
	\item \href{https://it.wikipedia.org/wiki/Indice_Gulpease}{Indice di Gulpease}.
\end{itemize}
\subsubsection{Riferimenti informativi}
\begin{itemize}
	\item Piano di progetto v 1.0.0;
	\item Piano di Qualifica v 1.0.0;
	\item \href{https://www.math.unipd.it/~tullio/IS-1/2018/}{Sito del corso di Ingegneria del Software}.
\end{itemize}
