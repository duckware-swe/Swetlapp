\clearpage
\section{Resoconto}
\subsection{Discussione dei capitolati}
	Prima della riunione tutti i membri del gruppo hanno compilato una tabella elettronica, creata e condivisa dal membro del gruppo \emph{Matteo Pellanda}, indicando i capitolati che hanno suscitato interesse durante l'esposizione in aula il 16/11/2018. \\
	Durante la  riunione, visionando la tabella, sono stati analizzati tutti  i  capitolati proposti escludendo alcuni di essi motivandone le ragioni al fine di ridurre il campo di scelte. \\
	Alla termine della discussione è stata fatta una votazione unanime e la scelta é ricaduta sul \emph{capitolato C4} con il progetto \emph{Meg\markg{Alexa}} proposto dall’azienda \emph{Zero12}. \\ 
	Il gruppo ha ritenuto interessante e stimolante l’utilizzo della tecnologia \markg{AWS} di \markg{Amazon} sul suo ultimo prodotto lanciato in commercio, \markg{Amazon} \markg{Alexa}.
	
	\subsection{Nome e logo gruppo}
	Durante l'incontro sono state ascoltate alcune idee sul nome e logo del gruppo, per ora identificato come \emph{gruppo 3}.
	Con scarsi risultati è stato deciso di rimandare la scelta di questa decisione  alla riunione successiva.
	
	\subsection{Organizzazione}
	Per facilitare e coordinare al meglio il lavoro è stato scelto di adottare i seguenti strumenti organizzativi:
	\begin{itemize}
		\item \textbf{\markg{Google Drive}:} sono state utilizzate file e cartelle condivise dalla piattaforma \markg{Google Drive} personale del componenete \emph{Matteo Pellanda} accessibili a tutti i membri del gruppo.
		\item \textbf{\markg{Slack}:} piattaforma scelta per la comunicazione, essa permette la creazione didiversi canali e l’integrazione con Drive e \markg{Git}Lab[?].
		\item \textbf{\markg{Git} Lab:} scelto come strumento per il versionamento.
		
	\end{itemize}
