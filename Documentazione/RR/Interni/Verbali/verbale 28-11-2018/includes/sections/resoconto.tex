\clearpage
\section{Resoconto}
	\subsection{Documento Studio di Fattibilità}
	In questa riunione è stato discusso come impostare e dividere la stesura del documento. In particolare lo studio è stato suddiviso assegnando un capitolato da analizzare ciascuno, trascrivendo il proprio elaborato su file separati per poi riportare il tutto su un unico elaborato. 
	
	\subsection{Nome e logo gruppo}
	Viene ripreso l'argomento sul nome e sul logo del gruppo lasciato in sospeso nella riunione precedente in data 20-11-2018.
	Da ora in avanti il \emph{gruppo 3} verrà identificato con il nome duckware, accompagnato dal logo raffigurante una papera di color gialla.
	
	\subsection{Creazione degli account per gli strumenti}
	Vengono creati gli account delle seguenti piattaforme:
	\begin{itemize}
		\item \textbf{\markg{Google Drive}:} associato all'account Google \href{mailto:duckware.swe@gmail.com} {duckware.swe@gmail.com}.
		\item \textbf{\markg{Slack}:} nome della chat \emph{duckware.slack.com}.
		\item \textbf{\markg{Git}Lab:} nome della \markg{repository}  \href{https://gitlab.com/AlessandroPegoraro/progetto-swe}{\emph{progetto-swe}}.
	\end{itemize}
