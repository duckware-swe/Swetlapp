\clearpage
\section{Resoconto}
	\subsection{Elenco domande a Zero12}
		\begin{itemize}
			\item Il team desidererebbe una risposta esplicita per quanto riguarda le esigenze del committente: nonostante le richieste di chiarimenti  da parte di Duckware, non è ancora perfettamente chiaro se la piattaforma dovrà  essere prettamente web, mobile o entrambe le soluzioni.
			\begin{itemize}
				\item Nel particolare se possibile l'uso di linguaggio \emph{cross-platform nativo} – Delphi
			\end{itemize}
			\item Come testare l’App o Web con \markg{Alexa}
			\item Implementazione
			\begin{itemize}
				\item All'affermazione \emph{“You pass back a graphical response”} è possibile interpretare la richiesta con una notifica di risposta al \markg{workflow}?
				\item Il login e la registrazione avvengono con un account \markg{Amazon} \markg{Alexa} oppure con un nuovo account che verrà in seguito collegato ad \markg{Amazon} \markg{Alexa}?
				\item Database: è messo a disposizione un database \markg{AWS} o un altro DBMS?
				\item Come vengono gestiti gli \markg{error}i se un \markg{workflow} non termina correttamente?
				\item Cosa succede se il comando è ambiguo e non viene interpretato?
			\end{itemize}
		\end{itemize}
	
	\subsection{Email a Zero12}
	Di seguito viene lasciato il contenuto della mail:
\begin{quote}
	\textit{
 	Grazie per la cortese risposta.
 	Domani saremo presenti all'incontro con Zero12 "Da \markg{software} a voice interface" e, se disponibili a parlare, avremo delle domande da porvi sul progetto.\\[0.5cm]
 	Vi ringraziamo \\
 	Team Duckware
 	}
\end{quote}
\clearpage
	\subsection{Tabella scadenza documenti - RR}
	Viene formulato un tabellario con le scadenze per i singoli documenti RR:\\[0.5cm]
\begin{center}
	\renewcommand{\arraystretch}{2.0}
	\begin{tabular}{  l | c | c}
		\hline
		\textbf{Documento}&\textbf{Inizio}&\textbf{Fine}\\
		\hline
		 Analisi dei Requisiti & Oggi & 01-01-2019\\
		\hline
		Piano di Progetto & 01-01-2019 & 06-01-2019\\
		\hline
		Piano di Qualifica & 01-01-2019 & 10-01-2019\\
		\hline
	\end{tabular}
\end{center}
