\clearpage
\section{Capitolato C3 - Grafana \& Bayes}
\subsection{Descrizione}
Il capitolato C3 si pone l'obiettivo di sviluppare un plug-in per la piattaforma Grafana, che gestisca e monitori il flusso di dati proveniente dal reparto DevOps, tramite l'implementazione di una o più reti Bayesiane. 
L'applicativo, attraverso il modello probabilistico, dovrebbe essere in grado di evidenziare le criticità nei \markg{processi} di DevOps, aiutando quindi le \markg{software} houses a capire dove intervenire per migliorare l'erogazione del servizio.

\subsection{Dominio Applicativo}
Costruire un plug-in di Grafana che applichi la definizione di una o più reti bayesiane al flusso di dati provenienti dal campo, per identificare i problemi che gli operatori di erogazione possono segnalare alla fabbrica del \markg{software} al fine di migliorare la qualità del servizio.

\subsection{Dominio Tecnologico}
\begin{description}	
	\item[Grafana:] Piattaforma open source di analisi e monitoraggio di flussi di dati in real-time. Fortemente modulare, consente agli sviluppatori di creare plug-in personalizzati sfruttando la duttilità di javascript.
Grafana archivia i dati in InfluxDB, un database specializzato per serie temporali.\\
Link: \href{https://grafana.com/}{https://grafana.com/}
	
	\item[Javascript:] Linguaggio di scripting orientato agli oggetti e agli eventi, comunemente utilizzato nella programmazione web lato \markg{client} per la creazione, in siti e applicazioni web, di effetti dinamici interattivi tramite funzioni di script invocate da eventi.
Ultimamente il suo campo di utilizzo è stato esteso alle cosiddette Hybrid App (app ibride), con le quali è possibile creare app per più sistemi operativi utilizzando un unico codice sorgente.\\
Link: \href{https://www.javascript.com/}{https://www.javascript.com/}
	
	\item[Reti Bayesiane:] Modello grafico probabilistico che rappresenta un insieme di variabili stocastiche con le loro dipendenze condizionali attraverso l'uso di un grafo aciclico, in cui i nodi rappresentano le variabili, mentre gli archi rappresentano le relazioni di dipendenza statistica tra le variabili e le distribuzioni locali di probabilità dei nodi figlio rispetto ai valori dei nodi padre.\\
Libreria utilizzata:\href{https://github.com/vangj/jsbayes}{https://github.com/vangj/jsbayes} 
\end{description}

\subsection{Valutazione Finale}
Questo capitolato è stato accolto discretamente bene dal gruppo, forse per merito dell'ottima presentazione e dall'ambizione dei singoli di interfacciarsi con una grossa azienda del settore, tanto da essersi guadagnato il terzo posto tra le preferenze	generali.
Tuttavia, in seguito ad un'attenta analisi dei progetti disponibili e delle specifiche richieste, il team di Duckware ha deciso di scartare questo progetto in quanto ritenuto troppo circoscritto, oltre a fornire un apporto cognitivo inferiore rispetto alle altre proposte.
