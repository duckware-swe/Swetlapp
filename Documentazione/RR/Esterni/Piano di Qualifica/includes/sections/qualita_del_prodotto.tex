\clearpage
\section{Qualità del prodotto}
\subsection{Scopo}
Basandosi sullo standard ISO/IEC 9126, sono state individuate le qualità che secondo il gruppo \emph{duckware} risultano importanti nell'arco del ciclo di vita del prodotto.
\subsection{Qualità dei documenti}
I documenti prodotti dal gruppo \emph{duckware} dovranno essere leggibili, comprensibili e corretti dal punto di vista ortografico, sintattico, logico e semantico.
\subsubsection{Comprensione}
\textbf{Obiettivi di qualità}
\begin{itemize}
	\item \textbf{Leggibilità}: i documenti prodotti devono essere leggibili e comprensibili a persone con almeno licenza di scuola superiore di primo grado;
	\item \textbf{Correttezza ortografica}: i documenti prodotti non devono presentare errori ortografici.
\end{itemize}

\subsection{Qualità del \markg{software}}
\subsubsection{Funzionalità}
Il prodotto deve fornire tutte le funzionalità che sono state individuate durante la redazione del documento \emph{Analisi dei \markg{requisiti}}.\\[0.4cm]
\textbf{Obiettivi di qualità}\\[0.4cm]
Il gruppo \emph{duckware} si impegna a perseguire
\begin{itemize}
	\item \textbf{Adeguatezza}: le funzionalità offerte dal prodotto risultano conformi rispetto alle aspettative;
	\item \textbf{Accuratezza}: il prodotto fornisce i risultati attesi, soddisfacendo il livello di dettaglio richiesto;
	\item \textbf{Sicurezza}: il prodotto assicura la protezione dei dati e delle informazioni che gli verranno forniti, affinché non sia permesso ne l'accesso ne la modifica a utenti o sistemi non autorizzati.
\end{itemize}

\subsubsection{Affidabilità}
Il prodotto \markg{software} deve svolgere correttamente le sue funzioni durante il suo utilizzo, anche in caso in cui si presentino situazioni non previste (anomale).
\textbf{Obiettivi di qualità}\\
L'esecuzione del prodotto dovrà avere le seguenti caratteristiche:
\begin{itemize}
	\item \textbf{Maturità:} principalmente si vuole evitare che si verifichino dei malfunzionamenti in seguito a difetti del \markg{software};
	\item \textbf{Tolleranza agli errori:} nel caso in cui si verifichino degli errori, dovuti a guasti o ad un uso scorretto dell'applicativo, questi devono essere gestiti correttamente.
\end{itemize}

\subsubsection{Usabilità}
Rappresenta la capacità del prodotto finale di poter essere usato e compreso facilmente, in ogni sua parte, da qualsiasi utente che lo voglia usare.\\[0.4cm]
\textbf{Obiettivi di qualità}\\[0.4cm]
Il prodotto dovrà puntare ai seguenti obiettivi di usabilità:
\begin{itemize}
	\item \textbf{Comprensibilità:} l'utente deve essere in grado di riconoscere le funzionalità che sono offerte dal \markg{software}, e deve poter comprendere le sue modalità di utilizzo per raggiungere i risultati attesi;
	\item \textbf{Apprendibilità:} all'utente viene data la possibilità di poter imparare le funzionalità offerte dal \markg{software};
	\item \textbf{Operabilità:} le funzioni presenti devono essere coerenti con le aspettative dell'utente;
	\item \textbf{Attrattiva:} l'utilizzo del \markg{software} deve risultare piacevole per l'utente.
\end{itemize}
\textbf{Misurazione}
Queste metriche di usabilità saranno misurate tramite alcune sessioni di prova con utenti esterni per dar modo di ottenere \markg{feedback} reali e misurazioni attendibili. Per questa procedura non è ancora possibile stabilire che metriche usare per misurarne l'usabilità del prodotto. Verrà decisa in successiva revisione dopo un'analisi approfondita con la proponente.

\subsubsection{Efficienza}
Attraverso questa metrica è possibile determinare la capacità del prodotto di eseguire le funzionalità offerte nel minor tempo possibile.
Inoltre con la misurazione dell'efficienza si vuole anche ridurre il numero di risorse usate dal \markg{software} per eseguire le funzionalità offerte.\\[0.4cm]
\textbf{Obiettivi di qualità}\\[0.4cm]
Il prodotto deve essere il più efficiente possibile secondo i seguenti criteri:
\begin{itemize}
	\item \textbf{Comportamento rispetto al tempo:} il \markg{software} deve eseguire le funzionalità che offre in tempi adeguati;
	\item \textbf{Utilizzo delle risorse:} il \markg{software}, per eseguire le sue funzionalità, deve avvalersi di un numero appropriato numero e tipo di risorse.
\end{itemize}

\subsubsection{Manutenibilità}
Questa metrica indica la capacità del \markg{software} di poter essere modificato, adattato o migliorato a seconda delle esigenze.\\[0.4cm]
\textbf{Obiettivi di qualità}\\[0.4cm]
Per misurare la misurabilità si andranno a valutare le seguenti caratteristiche del \markg{software}:
\begin{itemize}
	\item \textbf{Stabilità:} a seguito di modifiche del \markg{software} non devono insorgere effetti non voluti;
	\item \textbf{Testabilità:} si deve poter facilmente testare il \markg{software};
	\item \textbf{Modificabilità:} il \markg{software} deve poter essere modificato in alcune delle parti che lo compongono;
	\item \textbf{Analizzabilità:} si deve poter identificare facilmente le possibili cause di eventuali errori/malfunzionamenti.
\end{itemize}

\subsection{Tabella riassuntiva delle metriche e degli obbiettivi}
Di seguito viene riportata la tabella riassuntiva delle metriche e degli obbiettivi riconosciuti durante la qualità del prodotto.
\begin{center}
	\renewcommand{\arraystretch}{1.5}
	\rowcolors{3}{tableLightYellow}{}
	\begin{longtable}{  p{2cm}  p{5cm} p{2.5cm}  p{2.5cm}  }
		\rowcolor{tableHeadYellow}
		\textbf{Metrica}   & \textbf{Obbiettivo} & \textbf{Valori \mbox{accettati}} & \textbf{Valori \mbox{ottimali}}\\
		\textbf{MPRD001} Indice di Gulpease & Leggibilità del documento & 50 < x < 100 & 60 < x < 100 \\
		\textbf{MPRD002} Correttezza ortografica & Documenti privi di errori & 100\% privi & 100\% privi \\
		\caption{Tabella riassuntiva delle metriche e degli obbiettivi}
	\end{longtable}
\end{center}
