\clearpage
\section{Qualità del \markg{software}}
Di seguito viene riportata la tabella riassuntiva delle metriche e degli obbiettivi riconosciuti, con il range di accettazione e di ottimalità, per le misure effettuate nel \markg{software}.
\begin{center}
	\centering
	\renewcommand{\arraystretch}{1.5}
	\rowcolors{3}{tableLightYellow}{}
	\begin{longtable}{  p{7cm}  p{3cm} p{2.5cm}  }
		\rowcolor{tableHeadYellow}
		\textbf{Metrica}   & \textbf{Range \mbox{accettazione}} & \textbf{Range \mbox{ottimale}} \\ 
		\endhead
		Complessità ciclomatica                & 0 - 30      &      0 - 30 \\
		Numero di metodi                       & 2 - 10      &      3 - 8 \\
		Variabili non utilizzate               & 0           &      0 \\
		Numero di bug per linea                & 0 - 60      &      0 - 25 \\
		Rapporto linee di codice e commento    & \textgreater { 0.20 }      & SV \textgreater { 0.30 } \\
		\rowcolor{white}
		\caption{Tabella riassuntiva delle metriche e dei range}
	\end{longtable}
\end{center}
