\clearpage
\section{Introduzione}
\subsection{Scopo del documento}
Lo scopo del seguente documento consiste nel presentare le norme utilizzate dal Gruppo \emph{duckware} adottate per la \markg{verifica} e markg{validazione} dei prodotti e dei \markg{processi}. Per raggiungere lo scopo prefissato e il risultato desiderato, i \markg{processi} e i prodotti realizzati verranno sottoposti a \markg{verifica} continua affinché non vengano introdotti errori che ne influiscano il risultato finale in maniera negativa con l'uso di strategie e metriche di seguito descritte.
\subsection{Scopo del prodotto}
L'obiettivo del prodotto è la realizzazione di un'applicazione per smartphone, nello specifico per la piattaforma \markg{Android} OS, che permetta la creazione di \markg{workflow} per l'assistente vocale \markg{Amazon} \markg{Alexa}.\newline
Il \markg{back-end} sarà realizzato in \markg{Java} e opportunamente integrato con le \markg{API} di \markg{Amazon Web Services}, per il \markg{front-end} verrà utilizzato \markg{XML} per stabilire i layout e \markg{Java} per gestirne il comportamento. Si parlerà del \markg{front-end} dell'assistente vocale riferendosi a \markg{VUI} (voice user interface).
\subsection{Glossario}
Nel documento sono presenti termini che possono assumere significati ambigui a seconda del contesto o termini non conosciuti. Per ovviare a questa problematica è stato creato un Glossario contente tali termini con il loro significato specifico. Un termine è presente all'interno del \emph{Glossario v1.0.0} se seguito da una G corsiva a pedice.
\subsection{Riferimenti}
\subsubsection{Riferimenti Normativi}
\begin{itemize}
	\item Norme di Progetto v1.0.0.
\end{itemize}
\subsubsection{Riferimenti Informativi}
\begin{itemize}
	\item \href{https://www.math.unipd.it/~tullio/IS-1/2018/Dispense/L16.pdf}{Verifica e Validazione: introduzione - Slide del corso di Ingegneria del Software}
	\item \href{https://www.math.unipd.it/~tullio/IS-1/2018/Dispense/L13.pdf}{Qualità del Software - Slide del corso di Ingegneria del Software}
	\item \href{https://www.math.unipd.it/~tullio/IS-1/2018/Dispense/L14.pdf}{Qualità di Processo - Slide del corso di Ingegneria del Software}
	\item \href{https://www.math.unipd.it/~tullio/IS-1/2018/Dispense/L03.pdf}{Processi SW - Slide del corso di Ingegneria del Software}
	\item \markg{ISO}/IEC
	\begin{itemize} 
		\item \href{http://www.colonese.it/SviluppoSw_Standard_ISO15504.html}{ISO/IEC 15504}
	\end{itemize}
	\item \href{https://it.wikipedia.org/wiki/Indice_Gulpease}{Indice di Gulpease}
\end{itemize}
