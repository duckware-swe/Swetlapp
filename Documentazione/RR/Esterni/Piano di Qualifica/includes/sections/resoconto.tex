\clearpage
\appendix
\section{Resoconto attività di \markg{verifica}}
\subsection{Revisione dei \markg{requisiti}}
\subsubsection{Qualità di \markg{processo}}
Nella presente sezione si riassumono gli esiti delle attività di \markg{verifica} che è stata svolta su tutti i documenti che vengono consegnati nelle varie revisioni di progetto e sul prodotto \markg{software} che è in sviluppo.
\subsubsection{Qualità di prodotto}
In questa fase del progetto le metriche di prodotto istanziate sono quelle riguardanti i documenti.\\[0.4cm]
\textbf{Errori ortografici}\\[0.4cm]
Tutti i documenti dopo un rigoroso e attento lavoro di controllo dei verificatori ed un \markg{feedback} positivo rilasciato dallo strumento di controllo de \markg{error}i ortografici dell'ambiente \markg{TexStudio}, risultano privi di errori ortografici e raggiungono il valore accettabile ed ottimale della metrica  \textbf{MPRD002 Correttezza orografica}.
\\[0.4cm]\textbf{Indice di Gulpease}\\[0.4cm]
Per mezzo di alcuni script automatici è stato possibile istanziare la metrica  \textbf{MPRD001 Indice di Gulpease}.\\
Nella tabella sottostante è mostrato il risultato ottenuto per i principali documenti prodotti.
\begin{center}
	\centering
	\renewcommand{\arraystretch}{1.5}
	\rowcolors{3}{tableLightYellow}{}
	\begin{longtable}{  p{5cm}  p{5cm} p{3cm}  }
		\rowcolor{tableHeadYellow}
		\textbf{Nome documento}   & \textbf{Indice di \mbox{Gulpease}} & \textbf{Esito} \\ 
		\endhead
		Studio di Fattibilità     & 89                                 & Ottimo \\
		Norme di Progetto         & 97                                 & Ottimo \\
		Analisi dei Requisiti     & 80                                 & Ottimo \\
		Piano di Progetto         & 100                                & Ottimo \\
		Piano di Qualifica        & 96                                 & Ottimo \\
		\rowcolor{white}
		\caption{Resoconto delle misurazioni sulla metrica MPRD001 - Indice di Gulpease}
	\end{longtable}
\end{center}
\subsection{Revisione di Progettazione}
La sezione verrà implementata alla fine del periodo indicato.
\subsection{Revisione di Qualifica}
La sezione verrà implementata alla fine del periodo indicato.
\subsection{Revisione di Accettazione}
La sezione verrà implementata alla fine del periodo indicato.
