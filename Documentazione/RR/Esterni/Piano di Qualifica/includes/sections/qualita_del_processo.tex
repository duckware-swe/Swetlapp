\definecolor{white}{rgb}{1.0, 1.0, 1.0}
\clearpage
\section{Qualità del \markg{processo}}
\subsection{Scopo}
Per garantire la qualità del prodotto finale è necessario perseguire la qualità dei \markg{processi} che lo definiscono. Si è deciso di seguire un'organizzazione interna dei \markg{processi} incentrata sul principio del miglioramento continuo: \markg{PDCA} (Plan, Do, Check, Act) e di adottare lo standard ISO/IEC 15504, conosciuto come \markg{SPICE} (Software Process Improvement and Capability Determination), contenente un modello di riferimento che definisce una dimensione del \markg{processo} ed una dimensione della capacità.
\subsection{Competenze}
Lo standard stila una serie di regole molto dettagliate per coloro che si occupano della qualità dei \markg{processi}. Durante la realizzazione del progetto ci saranno frequenti cambi di ruolo necessari per gli scopi didattici. Risulta di conseguenza difficile applicare nel dettaglio l'intera regolamentazione. Il Gruppo \emph{duckware} si impegna a rispettare tali norme riportate nei documenti prodotti nei limiti delle conoscenze acquisibili nel tempo limitato.
\subsection{Processo di pianificazione i progetto, impostazione e controllo di \markg{processi}}
Il seguente \markg{processo} ha lo scopo di produrre dei piani di sviluppo studiati per il progetto che comprende la scelta del modello di ciclo di vita del prodotto, la descrizione delle attività e dei compiti da svolgere, la pianificazione temporale del lavoro e dei costi da sostenere e misurazioni per lo stato del progetto rispetto la pianificazione.\\
Vengono utilizzate le seguenti metriche definite nelle Norme di progetto:
\begin{itemize}
	\item \textbf{\markg{SPICE}}
	\item \textbf{Schedule Variance - SV}
	\item \textbf{Budget Variance - BV}
\end{itemize}
\clearpage
\subsection{Tabella riassuntiva delle metriche e degli obbiettivi}
Di seguito viene riportata la tabella riassuntiva delle metriche e degli obbiettivi riconosciuti durante la qualità dei \markg{processi}.
\begin{center}
	\renewcommand{\arraystretch}{1.5}
	\rowcolors{3}{tableLightYellow}{}
		\begin{longtable}{  p{2cm}  p{5cm} p{2.5cm}  p{2.5cm}  }
			\rowcolor{tableHeadYellow}
			\textbf{Metrica}   & \textbf{Obbiettivo} & \textbf{Valori \mbox{accettati}} & \textbf{Valori \mbox{ottimali}}\\
			\textbf{MPRC001} SPICIE & Miglioramento continuo & $x \eqslantgtr \text{livello 2}$ & $x \eqslantgtr \text{livello 4}$  \\
			\textbf{MPRC002} SV & Monitoraggio costo & $x \eqslantless \text{- 5 giorni}$ & 0 giorni \\
			\textbf{MPRC003} BV & Monitoraggio coso & $x \eqslantless -10$\% & 0\% \\
			\rowcolor{white}
			\caption{Tabella riassuntiva delle metriche e degli obbiettivi}
		\end{longtable}
\end{center}
