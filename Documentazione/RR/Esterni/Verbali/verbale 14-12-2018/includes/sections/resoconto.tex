\clearpage
\section{Resoconto}
	\subsection{Riassunto non dettagliato sul incontro di Zero12}
	Durante l'incontro di informazione il proponente ha fatto una breve introduzione spiegado come, nei tempi odierni, sia molto importante lo studio e la ricerca di un ottima interfaccia grafica nelle applicazioni, siano queste per smartphone o web. 
	Di seguito ha presentato l'ultimo prodotto di \markg{Amazon} entrato in commerco, \markg{Amazon} \markg{Alexa}. Quest'ultimo si presenta come un assistente personale intelligente in grado di restituire un output da un comando vocale impartito dall'utente. 
	Come esempio di funzionamento il \emph{CEO Stefano Dindo} ha utilizzato una \markg{skill} di \emph{GialloZafferano} in cui sono state notate le seguenti particolarità:
	\begin{itemize}
		\item Per poter impartire un secondo comando durante l'esecuzione del primo è necessario che la \markg{skill} sia predisposta per questa features.
		\item La gestione di errori durante l'esecuzione del comando è compito dei programmatori della \markg{skill}, non vi è presente una gestione standard di errori da parte di \markg{Alexa}.
		\item Fare particolare attenzione ed evitare contenuti lessicali indesiderati quali parolacce o altro.
		\item Il dispositivo è un assistente vocale, di conseguenza viene consigliato di struttureare la gestione del menù e l'attesa di un comando da parte della \markg{skill} il più accativante possibilie in modo da non annoiare l'utente e uscire dalla soglia di attenzione.
	\end{itemize}
	
	\subsection{Elenco domande e risposte fatte al CEO Stefano Dindo di Zero12}
		\begin{itemize}
			\item Il team desidererebbe una risposta esplicita per quanto riguarda le esigenze del committente: nonostante le richieste di chiarimenti  da parte di Duckware, non è ancora perfettamente chiaro se la piattaforma dovrà  essere prettamente web, mobile o entrambe le soluzioni. Nel particolare se possibile l'uso di linguaggio \emph{cross-platform nativo} – Delphi.
			\begin{itemize}
				\item \textbf{Risposta CEO S. Dindo}: l'obbiettivo di \emph{Zero12} in questo capitolato e a scopo sperimentale, vedere le idee e proposte che emergono nel implementare \emph{\markg{workflow}} per \emph{\markg{Alexa}}. Non è parametro di giudizio la scelta del linguaggio di utilizzo.
			\end{itemize}
			\item Come testare l’App o Web con \markg{Alexa}.
			\begin{itemize}
				\item \textbf{Risposta CEO S. Dindo}: \emph{Zero12} mette a disposizione un loro dispositivo per dei test nella loro sede. Ci viene comunicato inoltre che è stata fatta richiesta ad \markg{Amazon} la concessione di un \emph{\markg{Amazon} \markg{Alexa}} per scopo didattico.
			\end{itemize}
			\item All'affermazione \emph{“You pass back a graphical response”} è possibile interpretare la richiesta con una notifica di risposta al \markg{workflow}?
			\begin{itemize}
				\item \textbf{Risposta CEO S. Dindo}: si, può essere implementata questa richiesta con una notifica sul proprio device.
			\end{itemize}
			\item Il login e la registrazione avvengono con un account \markg{Amazon} \markg{Alexa} oppure con un nuovo account che verrà in seguito collegato ad \markg{Amazon} \markg{Alexa}?
			\begin{itemize}
				\item \textbf{Risposta CEO S. Dindo}: si consiglia una registrazione alla prorpia applicazione da cui poi associare l'account \emph{\markg{Amazon}}.
			\end{itemize}
			\item Database: è messo a disposizione un database \markg{AWS} o un altro DBMS?
			\begin{itemize}
				\item \textbf{Risposta CEO S. Dindo}: è messo a disposzione un databese \markg{AWS}.
			\end{itemize}
			\item Come vengono gestiti gli errori se un \markg{workflow} non termina correttamente?
			\begin{itemize}
				\item \textbf{Risposta CEO S. Dindo}: gli errori del \markg{workflow} e della \markg{skill} è compito dei programmatori, \markg{Alexa} non predispone alcun strumento o ruoutine standard per la loro gestione.
			\end{itemize}
			\item Cosa succede se il comando è ambiguo e non viene interpretato?
			\begin{itemize}
				\item Non viene fatta la domanda per problemi di tempo. Si rimanda il quesito all'incontro successivo.
			\end{itemize}
		\end{itemize}

