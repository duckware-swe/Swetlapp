\clearpage
\section{Modello di sviluppo}
Come modello di sviluppo per il ciclo di vita del \markg{software} è stato deciso di adottare il \emph{\markg{modello iterativo}}, per permettere una maggiore flessibilità e adattamento al gruppo nel suo ciclo evolutivo.\\[0.5cm]
Il modello iterativo aiuta il dialogo con gli \markg{stakeholder} in caso ogni aspetto del sistema non sia compreso fin dall'inizio, consentendo maggiore capacità di adattamento, ciò però comporta rischio di non convergenza poiché ogni iterazione comporta un ritorno all'indietro nel tempo.
Dunque l'approccio migliore è quello di decomporre la realizzazione del sistema, identificando le componenti più critiche in modo da limitare superiormente il numero delle iterazioni.\\
Durante ogni iterazione viene effettuato, se necessario, l'incremento dei documenti prodotti dal gruppo: questo permette di adattarsi ai requisiti, in modo evolutivo, sulla base del feedback delle iterazioni precedenti.
\subsection{Modello \markg{agile}}
Il modello \markg{agile} è un modello altamente dinamico, con cicli iterativi e incrementali, basato su principi fondanti per svincolarsi dall'eccessiva rigidità.\\ Esso si basa su quattro principi:
	\begin{itemize}
		\item L'interazione con gli \markg{stakeholder} va incentivata, questa viene infatti considerata la miglior risorsa disponibile durante lo sviluppo del progetto;
		\item È più importante avere \markg{software} funzionante che documentazione rigorosa;
		\item È importante la collaborazione con i \markg{client}i, in quanto essa produce risultati migliori rispetto ai soli rapporti contrattuali;
		\item Bisogna essere pronti a rispondere ai cambiamenti oltre che aderire alla pianificazione.
	\end{itemize}
L'idea di base per implementare tali principi è l'utilizzo della {"\markg{user story}"}, il lavoro viene quindi suddiviso in piccoli incrementi a valore aggiunto che vengono sviluppati indipendentemente in una sequenza continua dall'analisi all'integrazione.\\ Gli obiettivi strategici sono quindi:
	\begin{itemize}
		\item Poter costantemente e in ogni momento dimostrare al cliente ciò che è stato fatto;
		\item Verificare l'avanzamento tramite il progresso reale;
		\item Soddisfare e motivare gli sviluppatori con risultati immediati;
		\item Assicurare e dimostrare una buona \markg{verifica} e integrazione dell'intero prodotto \markg{software}. 
	\end{itemize}
I \textbf{vantaggi} principali di questo modello, sopra citati, sono quindi la facilità di adattamento in base alle esigenze da parte del gruppo, creando delle suddivisioni con piccoli incrementi di valore aggiunto con la possibilità di iterare per far sì che venga a pieno il soddisfacimento delle richieste del proponente.
