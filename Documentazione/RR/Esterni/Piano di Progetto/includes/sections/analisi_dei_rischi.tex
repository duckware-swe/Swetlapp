\clearpage
\section{Analisi dei rischi}
Questa sezione elenca i possibili rischi in cui il gruppo \emph{duckware} può incorrere durante la realizzazione del prodotto.\\
\\
I rischi rilevati possono essere raggruppati nelle seguenti categorie: 
\begin{itemize}
	\item A livello tecnologico; 
	\item A livello dei componenti del gruppo \emph{duckware};
	\item A livello di valutazione dei costi;
	\item A livello di organizzazione del lavoro;
	\item A livello dei \markg{requisiti}.
\end{itemize}
~
Ogni rischio possiede:
	
\begin{itemize}
	\item Descrizione;
	\item Probabilità di occorrenza;
	\item Grado di pericolosità;
	\item Identificazione;
	\item Controllo;
	\item Piano di contingenza.
\end{itemize}

\subsection{A livello tecnologico}
	
\subsubsection{Tecnologie da usare}
\begin{itemize}
			\item \textbf{Descrizione}: una tecnologia fondamentale per lo svolgimento del progetto è la piattaforma \markg{AWS} di \markg{Amazon}. Allo stato attuale i componenti del gruppo non hanno particolari conoscenze riguardo a questa piattaforma. Di conseguenza, i tempi di apprendimento per l'uso corretto di quest'ultima non possono essere ben quantificabili a priori, e possono variare di persona in persona;
			\item \textbf{Probabilità di occorrenza}: media;
			\item \textbf{Grado di pericolosità}: alto;
			\item \textbf{Identificazione}: il Responsabile deve \markg{verifica}re il grado di preparazione di ogni membro del gruppo relativo alle tecnologie utilizzate;
			\item \textbf{Controllo}: ogni componente del gruppo deve, in maniera autonoma, studiare tutte le tecnologie necessarie per la realizzazione del prodotto facendo uso dei documenti forniti dall'Amministratore;
			\item \textbf{Piano di contingenza}: se si presentasse il caso per cui uno o più membri del gruppo non fossero in grado di portare a termine il proprio lavoro a causa di lacune circa l'utilizzo delle tecnologie, il carico di lavoro dovrà essere ridistribuito fra gli altri membri con l'accortezza di non far slittare le \markg{milestone} fissate.
\end{itemize}
		
\subsubsection{Problemi hardware}
\begin{itemize}
			\item \textbf{Descrizione}: ogni membro del gruppo ha a disposizione un proprio computer portatile di tipo non professionale, il rischio di rottura di uno di questi è possibile ed è da tenere in considerazione. Un altro rischio di fallibilità hardware è quello del \markg{server} usato per ospitare il progetto, un malfunzionamento su tale macchina mette a rischio il lavoro dell'intero gruppo;
			\item \textbf{Probabilità di occorrenza}: bassa;
			\item \textbf{Grado di pericolosità}: medio;
			\item \textbf{Identificazione}: ogni membro è \markg{responsabile} della cura del proprio computer ed è suo compito notare eventuali comportamenti anomali;
			\item \textbf{Controllo}: per prevenire situazioni spiacevoli, i membri del gruppo devono di volta in volta salvare il proprio lavoro sul \markg{repository} remoto in \markg{GitLab};
			\item \textbf{Piano di contingenza}: se si presentasse il caso di una perdita di dati, questi dovranno essere ripristinati al più presto in locale dal membro del gruppo che ha causato la perdita.
\end{itemize}



\subsection{A livello del personale}
\subsubsection{Problemi tra componenti del team}
\begin{itemize}
			\item \textbf{Descrizione}: per la maggior parte dei componenti del gruppo, questo progetto è la prima esperienza di lavoro in un gruppo di grandi dimensioni. Tale fattore potrebbe causare problemi di collaborazione causando squilibri interni, provocando così dei ritardi nella consegna del prodotto;
			\item \textbf{Probabilità di occorrenza}: bassa;
			\item \textbf{Grado di pericolosità}: alto;
			\item \textbf{Identificazione}: la comunicazione costante con il Responsabile può far sì che quest'ultimo monitori ogni tipo di problematica sul nascere;
			\item \textbf{Controllo}: la divisione del lavoro e il controllo da parte del \markg{responsabile} assicurano che ogni componente sappia in ogni momento quale parte del lavoro sia compito suo svolgere;
			\item \textbf{Piano di contingenza}: nel caso si verificassero degli squilibri tra i membri del gruppo, sarà compito del \markg{responsabile} intervenire per portare alla normalità la situazione.
\end{itemize}
		
\subsubsection{Problemi dei componenti del gruppo}
\begin{itemize}
			\item \textbf{Descrizione}: ogni membro del gruppo ha impegni personali e necessità proprie. Risulta inevitabile il verificarsi di problemi organizzativi in seguito a sovrapposizioni di tali impegni;
			\item \textbf{Probabilità di occorrenza}: medio;
			\item \textbf{Grado di pericolosità}: medio;
			\item \textbf{Identificazione}: il \markg{responsabile}, controllando le scadenze delle varie fasi del progetto, chiederà al gruppo di avvertirlo per tempo nel caso in cui si presentassero imprevisti;
			\item \textbf{Controllo}: notificando il \markg{responsabile} è possibile coprire il lavoro mancante in tempi brevi da parte dei componenti disponibili al momento;
			\item \textbf{Piano di contingenza}: ad esclusione di impegni improvvisi o malattie, ogni componente si prenderà la responsabilità di rifare, in tempi brevi parti del lavoro di chi lo ha coperto durante l'assenza creatasi.
\end{itemize}
		
\subsection{A livello di valutazione dei costi}
		
\subsubsection{Sottostima dei costi}
\begin{itemize}
			\item \textbf{Descrizione}: durante la pianificazione è possibile che i tempi per l'esecuzione di alcune attività vengano calcolati in modo errato o che alcune tecnologie non si riesca ad implementarle;
			\item \textbf{Probabilità di occorrenza}: media;
			\item \textbf{Grado di pericolosità}: alta;
			\item \textbf{Identificazione}: verranno svolte comunicazioni periodiche con i proponenti e ad ogni implementazione di una tecnologia proposta si andrà a controllare il tempo impiegato per aggiungerla;
			\item \textbf{Controllo}: ad ogni fase del progetto, il gruppo svolgerà delle considerazioni sul da farsi per chiarire se la cosa rientra nelle tempistiche/costi preventivati;
			\item \textbf{Piano di contingenza}: ad ogni revisione se il proponente lo riterrà necessario si potrà ridiscutere le proposte formulate dal gruppo, specialmente quelle considerate più lunghe da implementare.
\end{itemize}

\subsection{A livello dei \markg{requisiti}}
	
\subsubsection{Variazioni nei \markg{requisiti}}
\begin{itemize}
			\item \textbf{Descrizione}: è possibile che alcuni \markg{requisiti} individuati dal capitolato vengano interpretati in modo erroneo o in maniera incompleta da parte degli Analisti rispetto alle aspettative del proponente. Sarà inoltre possibile che alcuni \markg{requisiti} vengano tolti, aggiunti o modificati durante il corso del progetto;
			\item \textbf{Probabilità di occorrenza}: media;
			\item \textbf{Grado di pericolosità}: alto;
			\item \textbf{Identificazione}: per ridurre al minimo la probabilità che si verifichi un tale errore nella prima fase di analisi vi è stato un continuo confronto con il proponente sia in via telematica che con incontri personali;
			\item \textbf{Controllo}: tramite degli incontri con il proponente si è cercato di definire con chiarezza ogni \markg{requisito} necessario al corretto sviluppo del progetto, inoltre alcuni dubbi che gli Analisti avevano sono stati chiariti;
			\item \textbf{Piano di contingenza}: sarà indispensabile correggere eventuali errori o imprecisioni indicati dal committente al termine di ogni revisione.
\end{itemize}
