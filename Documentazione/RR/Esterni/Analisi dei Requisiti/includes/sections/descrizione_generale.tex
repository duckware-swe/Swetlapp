\clearpage
\section{Descrizione generale}
\subsection{Obiettivi del prodotto}
L'obiettivo principale del prodotto è la realizzazione di \markg{skills} per \markg{Amazon} \markg{Alexa} che verranno installate sul dispositivo e gestite attraverso l'applicazione smartphone con dei \markg{workflow}. La peculiarità sta nella creazione di queste \markg{skills} e la loro iterazione con l'utente che avviene attraverso input vocali.
\subsection{Funzioni del prodotto}
L'applicazione fornirà un'interfaccia grafica con la quale l'utente potrà configurare dei \markg{workflow} da far eseguire tramite apposito comando dal dispositivo \markg{Amazon} \markg{Alexa}. L'utente potrà configurare vari parametri di ogni \markg{workflow} in qualsiasi momento, l'unico vincolo consisterà nell'unicità del nome di ogni \markg{workflow}.
Le funzionalità del prodotto sono le seguenti: 
\begin{itemize}
	\item Gestire i \markg{workflow}:
	\begin{itemize}
		\item Creare nuovi \markg{workflow};
		\item Modificare le informazioni generali di ogni \markg{workflow};
		\item Aggiungere nuovi \markg{connettori};
		\item Modificare i \markg{connettori} preesistenti;
		\item Modificare l'ordine in cui i \markg{connettori} vengono eseguiti;
		\item Rimuovere i \markg{connettori} dal \markg{workflow};
		\item Rimuovere definitivamente un \markg{workflow}.
	\end{itemize}
	\item Utilizzare le seguenti \markg{skills}:
	\begin{itemize}
		\item Ricevere un messaggio di benvenuto associato all'esecuzione del \markg{workflow}
		\item Leggere feed \markg{RSS};
		\item Leggere mail ricevute;
		\item Leggere messaggi WhatsApp ricevuti;
		\item Ricevere informazioni meteo;
		\item Ricevere informazioni relative al palinsesto TV;
		\item Gestire alcune funzionalità del proprio telefono;
		\item Riprodurre musica;
		\item Riprodurre \markg{rumori bianchi};
		\item Leggere o pubblicare tweet;
		\item Leggere o aggiungere eventi al calendario;
		\item Interagire con \markg{Trello}.
	\end{itemize}
\end{itemize}
\subsection{Caratteristiche degli utenti}
Il prodotto è rivolto ad un utenza circoscritta di persone. Si richiede come pre\markg{requisito} essere in possesso di un dispositivo \markg{Android} e di un dispositivo \markg{Amazon} \markg{Alexa} per usufruire dell'applicativo commissionato. L'utente potrà, una volta registrato, utilizzare l'applicativo e creare i suoi \markg{workflow}.
\subsection{Piattaforma d'esecuzione}
%Il prodotto durante la sua creazione e \markg{processo} di realizzazione verrà eseguito su due diverse piattaforme:
%Ambiente di sviluppo \markg{IDE} tra cui IntelliJ \markg{IDEA} o \markg{Android} Studio ed \markg{Amazon} \markg{Alexa}.
%Il prodotto finale destinato all'utenza verrà invece eseguito su dispositivi \markg{Android} e su \markg{Amazon} \markg{Alexa}.
Il prodotto durante la fase di sviluppo, verrà eseguito sia all'interno di un emulatore \markg{Android} per PC che su smartphone \markg{Android} fisici; inoltre ci si interfaccerà con \markg{Amazon} \markg{Alexa} per la parte di \markg{VUI} (Voice User Interface).
Il prodotto finale destinato all'utenza verrà invece eseguito su dispositivi \markg{Android} e su \markg{Amazon} \markg{Alexa}.
\subsection{Vincoli generali}
L'utente per usufruire del servizio offerto dal prodotto realizzato dovrà avere un account di \markg{Amazon} funzionante e verificato, una connessione ad internet e i dispositivi di esecuzione sopra citati. 
