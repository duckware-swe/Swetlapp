\clearpage
\section{Introduzione}
\subsection{Scopo del documento}
Lo scopo del presente documento è di fornire una descrizione completa e dettagliata di tutti i \markg{requisiti} che sono stati individuati e dei casi d'uso riguardanti il progetto MegAlexa.
Tali informazioni sono state recuperate ed elaborate dal capitolato e dagli incontri, con le relative discussioni, con la proponente \emph{Zero12}.
\subsection{Scopo del prodotto}
L'obiettivo del prodotto è la realizzazione di un'applicazione per smartphone, nello specifico per la piattaforma \markg{Android} OS, che permetta la creazione di \markg{workflow} per l'assistente vocale \markg{Amazon} \markg{Alexa}. Il \markg{back-end} sarà realizzato in \markg{Java} opportunamente integrato con le \markg{API} di \markg{Amazon Web Services}, per il \markg{front-end} verrà usato \markg{XML} per stabilire i \markg{layout} e \markg{Java} per gestirne il comportamento. Si parlerà del \markg{front-end} dell'assistente vocale riferendosi a \markg{VUI} (voice user interface).
\subsection{Glossario}
Nel documento sono presenti termini che possono assumere significati ambigui a seconda del contesto o termini non conosciuti. Per ovviare a questa problematica è stato creato un Glossario contente tali termini con il loro significato specifico. Un termine è presente all'interno del \emph{Glossario v1.0.0} se seguito da una G corsiva in pedice.
\subsection{Riferimenti}
\subsubsection{Riferimenti normativi}
\begin{itemize}
	\item Norme di progetto v1.0.0;
	\item \href{https://www.math.unipd.it/~tullio/IS-1/2018/Progetto/C4.pdf} {Capitolato d’appalto C4 dell'anno accademico 2018/2019.}
\end{itemize}
\subsubsection{Riferimenti informativi}
\begin{itemize}
	\item \href{https://www.math.unipd.it/~tullio/IS-1/2018/Progetto/C4p.pdf} {Presentazione capitolato d’appalto C4 dell'anno accademico 2018/2019};
	\item \href{https://www.math.unipd.it/~tullio/IS-1/2018/} {Lucidi didattici utilizzati durante il corso di Ingegneria del Software};
	\begin{itemize}
		\item Diagrammi dei casi d’uso;
		\item Analisi dei \markg{requisiti}.
	\end{itemize}
\end{itemize}
