\clearpage
\definecolor{tableHeadYellow}{rgb}{1.0, 0.88, 0.21}
\definecolor{tableLightYellow}{rgb}{1.0, 0.97, 0.86}
\definecolor{white}{rgb}{1.0, 1.0, 1.0}

\section{Requisiti}
\subsection{Classificazione dei \markg{requisiti}}
I \markg{requisiti} vengono classificati ed assegnati attraverso l'utilizzo di un identificativo univoco secondo le regole riportate all'interno del documento Norme di progetto v1.0.0
\subsubsection{Requisiti funzionali}
Requisiti funzionali
\begin{center}
	\renewcommand{\arraystretch}{1.5}
	\rowcolors{3}{tableLightYellow}{}
	\begin{longtable}{  p{2.5cm}  p{2.1cm} p{7cm}  p{1.7cm} }
		\rowcolor{tableHeadYellow}
		\textbf{Identificativo}   & \textbf{Importanza} & \textbf{Descrizioni} & \textbf{Fonte} \\ 
		\endhead
		R2F1   & Obbligatorio & L'utente può registrarsi e creare una sua area personale nella quale inserire \markg{workflow}                & Capitolato e Verbale \\  
		R2F1.1 & Obbligatorio & La registrazione necessita di nome, cognome ed indirizzo email valido                               & Interno              \\  
		R2F2   & Obbligatorio & L'utente può effettuare il login alla sua area personale                                              & Capitolato           \\  
		R2F2.1 & Obbligatorio & L'utente può recuperare la password in caso di smarrimento                                            & Interno              \\  
		R2F2.2 & Obbligatorio & L'utente può effettuare il logout                                                                    & Interno              \\  
		R1F3   & Desiderabile & L'utente può leggere una breve guida all'uso dell'applicazione                                        & Interno              \\  
		R2F4   & Obbligatorio & L'utente può gestire tutti i \markg{workflow} che ha inserito                                                 & Capitolato           \\  
		R2F4.1 & Obbligatorio & Ogni utente avrà i suoi \markg{workflow} personali che non saranno visibili ad altri utenti                   & Capitolato           \\  
		R2F4.2 & Obbligatorio & Ogni \markg{workflow} dovrà avere un nome identificativo di almeno 4 caratteri                                & Interno              \\  
		R2F4.3 & Obbligatorio & L'utente può aggiungere un \markg{workflow} alla sua lista personale                                          & Capitolato           \\  
		R2F4.4 & Obbligatorio & L'utente può rimuovere un \markg{workflow} dalla sua lista personale                                          & Interno              \\  
		R2F4.5 & Obbligatorio & L'utente può modificare un \markg{workflow} dalla sua lista personale                                         & Interno              \\  
		R2F4.6 & Obbligatorio & L'utente può vedere un numero che indica quanti \markg{workflow} ha creato                                    & Interno              \\  
		R2F5   & Obbligatorio & L'utente può vedere una lista di \markg{connettori} da assegnare al \markg{workflow} su cui sta operando              & Capitolato           \\  
		R2F5.1 & Obbligatorio & L'utente può vedere una lista di \markg{connettori} da rimuovere dal \markg{workflow} su cui sta operando             & Interno              \\  
		R2F5.2 & Obbligatorio & L'utente deve poter selezionare più di un \markg{connettore} per ciascun \markg{workflow}                             & Verbale              \\  
		R0F5.3 & Opzionale    & L'utente deve vedere un conteggio di quanti \markg{connettori} sono stati selezionati                         & Interno              \\  
		R2F5.4 & Obbligatorio & L'utente deve selezionare almeno un \markg{connettore} dalla lista per poter creare un \markg{workflow}               & Interno e Verbale    \\  
		R2F5.5 & Obbligatorio & L'utente deve poter selezionare lo stesso \markg{connettore} una sola volta per \markg{workflow}                                  & Interno              \\  
		R2F6   & Obbligatorio & L'applicazione dovrà supportare più lingue                                                            & Capitolato           \\  
		R1F6.1 & Desiderabile & Verrà caricata automaticamente la lingua corrispondente alle impostazioni di località del dispositivo & Interno              \\
		R2F7 & Obbligatorio	& L'utente deve poter avviare i \markg{workflow} creati tramite comando vocale impartito ad \markg{Amazon} \markg{Alexa}. & Capitolato \\
		R1F8	& Desiderabile	& L'utente deve poter conoscere tramite comando vocale impartito ad \markg{Amazon} \markg{Alexa} la lista dei \markg{workflow} personali creati in precedenza.	& Interno \\
		R1F9	& Desiderabile	& L'utente deve poter chiedere un aiuto per utilizzare la skill tramite comando vocale impartito ad \markg{Amazon} \markg{Alexa}.	& Interno \\
		R2F10	& Obbligatorio	& L'utente deve poter interrompere l'esecuzione della skill tramite comando vocale impartito ad \markg{Amazon} \markg{Alexa}.	& Esterno e Interno \\
		R0F11	& Opzionale	& L'utente deve poter terminare il \markg{connettore} attualmente in esecuzione per passare al successivo tramite comando vocale impartito ad \markg{Amazon} \markg{Alexa}.	& Interno \\
		\rowcolor{white}
		\caption{Tabella \markg{requisiti} funzionali}
	\end{longtable}
\end{center}

\subsubsection{Requisiti di qualità}
Requisiti di qualità
\begin{center}
	\renewcommand{\arraystretch}{1.5}
	\rowcolors{3}{tableLightYellow}{}
	\begin{longtable}{  p{2.5cm}  p{2.1cm} p{7cm}  p{1.7cm} }
		\rowcolor{tableHeadYellow}
		\textbf{Identificativo}   & \textbf{Importanza} & \textbf{Descrizioni} & \textbf{Fonte} \\ 
		\endhead
		R2Q1   & Obbligatorio & Codifica e progettazione devono seguire le norme riportate all'interno del documento Piano di qualifica v1.0.0                      & Interno    \\  
		R2Q2   & Obbligatorio & L'approccio al codice \markg{Java} dovrà seguire quanto riportato \href{https://google.github.io/styleguide/javaguide.html}{qui}                      & Capitolato \\  
		R2Q2.1 & Obbligatorio & Lo sviluppo del codice deve essere supportato dall'utilizzo di test di unità                                                        & Interno    \\  
		R2Q3   & Obbligatorio & Dovrà essere fornito un manuale utente in lingua italiana che tratterà l'uso dell'applicazione                                      & Verbale    \\  
		R2Q4   & Obbligatorio & Il codice sorgente deve essere caricato nella piattaforma \markg{GitLab} di duckware                                                        & Interno    \\  
		R2Q5   & Obbligatorio & Si dovranno utilizzare i servizi di \markg{AWS} quali \markg{API} Gateway, Lambda ed Aurora Serverless ove necessario                               & Capitolato \\  
		R2Q5.1 & Obbligatorio & Si dovrà creare un'\markg{architettura} REST per comunicare con l'applicazione                                                              & Interno    \\  
		R2Q6   & Obbligatorio & Per lo sviluppo del prodotto richiesto devono essere rispettate tutte le norme descritte nel documento Norme di Progetto v1.0.0 & Interno    \\  
		R2Q7   & Obbligatorio & Per lo sviluppo del prodotto richiesto devono essere rispettati tutti i \markg{processi} descritti nel documento Piano di Qualifica v1.0.0 & Interno    \\
		\rowcolor{white}
		\caption{Tabella \markg{requisiti} di qualità}
	\end{longtable}
\end{center}
\clearpage
\subsubsection{Requisiti di vincolo}
Requisiti di vincolo
\begin{center}
	\renewcommand{\arraystretch}{1.5}
	\rowcolors{3}{tableLightYellow}{}
	\begin{longtable}{  p{2.5cm}  p{2.1cm} p{7cm}  p{1.7cm} }
		\rowcolor{tableHeadYellow}
		\textbf{Identificativo}   & \textbf{Importanza} & \textbf{Descrizioni} & \textbf{Fonte} \\ 
		\endhead
		R1V1   & Desiderabile & L'applicazione dovrà essere sviluppata per dispositivi \markg{Android}                                                                             & Capitolato            \\  
		R2V1.1 & Obbligatorio & L'applicazione dovrà essere sviluppata con \markg{Java} 10 o superiore                                                                             & Interno               \\  
		R2V1.2 & Obbligatorio & L'applicazione dovrà supportare un livello di \markg{API} \markg{Android} che sia 26 o superiore                                                           & Interno               \\  
		R2V1.3 & Obbligatorio & Sarà necessario utilizzare JUnit per eseguire i test di unità                                                                              & Interno               \\  
		R2V2   & Obbligatorio & L'applicazione dovrà essere creata utilizzando \markg{IntelliJ IDEA}                                                                               & Interno               \\  
		R2V2.1 & Obbligatorio & Il deploy dell'applicazione avverrà per mezzo degli strumenti di build forniti da \markg{IntelliJ IDEA}                                            & Interno               \\  
		R2V2.2 & Obbligatorio & debug dell'applicazione dovrà essere eseguito sull'emulatore ufficiale fornito da Google utilizzando il toolkit AVD                        & Interno               \\  
		R1V2.3 & Desiderabile & Il debug dovrà avvenire in un reale dispositivo con dimensioni di schermo differenti da quelle impostate dall'emulatore creato tramite AVD & Interno               \\  
		R2V2.4 & Obbligatorio & L'applicazione deve avere come target minimo di SDK la versione di \markg{API} xx (dopo decido ma credo metteremo la                               & Interno               \\  
		R2V3   & Obbligatorio & Il collegamento con i servizi di \markg{AWS} avverrà tramite l'SDK \markg{AWS}  di amazon                                                                  & Capitolato            \\  
		R2V3.1 & Obbligatorio & Verrà utilizzata la versione 2.0 dell'SDK                                                                                                  & Interno               \\  
		R2V4   & Obbligatorio & Utilizzo di \markg{Amazon} \markg{API} Gateway per la realizzazione di \markg{API} per la comunicazione con l'applicazione                                         & Verbale               \\  
		R2V4.1 & Obbligatorio & Creazione di endpoint \markg{API} RESTful tramite \markg{Amazon} \markg{API} Gateway                                                                               & Interno               \\  
		R2V5   & Obbligatorio & Utilizzo di \markg{AWS} Lambda per l'esecuzione automatica di richieste HTTP create via \markg{API} Gateway                                                & Verbale               \\  
		R2V6   & Obbligatorio & Creazione di un database relazionale utilizzando \markg{Amazon} Aurora                                                                             & Capitolato ed Esterno \\  
		R2V7   & Obbligatorio & Un utente può creare \markg{workflow} solamente dopo aver effettuato il login con successo                                                         & Capitolato            \\
		\caption{Tabella \markg{requisiti} di vincolo}
	\end{longtable}
\end{center}
\subsubsection{Requisiti prestazionali}
Requisiti prestazionali
\begin{center}
	\renewcommand{\arraystretch}{1.5}
	\rowcolors{3}{tableLightYellow}{}
	\begin{longtable}{  p{2.5cm}  p{2.1cm} p{7cm}  p{1.7cm} }
		\rowcolor{tableHeadYellow}
		\textbf{Identificativo}   & \textbf{Importanza} & \textbf{Descrizioni} & \textbf{Fonte} \\ 
		R2P1 & Desiderabile & Il \markg{server} dovrebbe avere una latenza massima di 3 secondi, a meno di errori di connessione                     & Interno \\  
		R2P2 & Obbligatorio & L'applicazione deve sempre rimanere responsiva e mostrare adeguati indicatori di caricamento in caso di attesa & Interno \\
		\caption{Tabella \markg{requisiti} prestazionali}
	\end{longtable}
\end{center}

\subsection{Tracciamento}
\subsubsection{Tracciamento \markg{requisiti}-fonti}
\begin{center}
	\renewcommand{\arraystretch}{1.5}
	\rowcolors{3}{tableLightYellow}{}
	\begin{longtable}{  p{5cm} p{5cm} }
		\rowcolor{tableHeadYellow}
		\textbf{Requisiti} & \textbf{Fonti} \\
		\endhead 
		
		R2F1 & Capitolato \newline Verbale \newline UC2 \newline UC5 
				\newline UC5.1 \newline UC5.2\\
		R2F1.1 & Interno \\
		R2F2 & Capitolato \newline UC2 \newline UC3 \\
		R2F2.1 & Interno \newline UC6 \\
		R2F2.2 & Interno \newline UC4 \\
		R1F3 & Interno \newline UC1 \newline UC7 \\
		R2F4 & Capitolato \newline UC8.1 \newline UC8.2\\
		R2F4.1 & Capitolato \\
		R2F4.2 & Interno \\
		R2F4.3 & Capitolato \newline UC8 \\
		R2F4.4 & Interno \newline UC8.2 \\
		R2F4.5 & Interno \newline UC8.3 \\
		R2F4.6 & Interno \\
		R2F5 & Capitolato \newline UC9 \\
		R2F5.1 & Interno \newline UC9.2 \\
		R2F5.2 & Verbale \\
		R0F5.3 & Interno \\
		R2F5.4 & Interno \newline Verbale \\		
		R2F5.5 & Interno \\
		R2F6 & Capitolato \\
		R2F6.1 & Interno \\
		R2F7 & Capitolato \newline UC10 \\
		R1F8 & Interno \newline UC11 \\
		R1F9 & Interno \newline UC12 \\
		R2F10 & Interno \newline Esterno \newline UC13 \\
		R0F11 & Interno \newline UC14 \\
		R2Q1 & Interno \\
		R2Q2 & Capitolato \\
		R2Q2.1 & Interno \\
		R2Q3 & Verbale \\
		R2Q4 & Interno \\
		R2Q5 & Capitolato \\
		R2Q5.1 & Interno \\
		R2Q6 & Interno \\
		R2Q7 & Interno \\
		R1V1 & Capitolato \\
		R2V1.1 & Interno \\
		R2V1.2 & Interno \\
		R2V1.3 & Interno \\
		R2V2 & Interno \\
		R2V2.1 & Interno \\
		R2V2.2 & Interno \\
		R1V2.3 & Interno \\
		R2V2.4 & Interno \\
		R2V3 & Capitolato \\
		R2V3.1 & Interno \\
		R2V4 & Verbale \\
		R2V4.1 & Interno \\
		R2V5 & Verbale \\
		R2V6 & Capitolato \newline Esterno \\
		R2V7 & Capitolato \\
		R2P1 & Interno \\
		R2P2 & Interno \\
		\rowcolor{white}
	\caption{Tabella tracciamento \markg{requisiti}-fonti}
	\end{longtable}
\end{center}

\subsubsection{Tracciamento fonti-requisiti}
\begin{center}
	\centering
	\renewcommand{\arraystretch}{1.5}
	\rowcolors{3}{tableLightYellow}{}
	\begin{longtable}{  p{5cm} p{5cm} }
		\rowcolor{tableHeadYellow}
		\textbf{Fonti} & \textbf{Requisiti} \\
		\endhead  
		
		Capitolato & R2F1 \newline R2F2 \newline R2F4 \newline R2F4.1 \newline R2F4.3 \newline R2F5 \newline R2F6 \newline R2F7 \newline R2Q2 \newline R2Q5 \newline R1V1 \newline R2V3 \newline R2V6 \newline
					R2V7 \\
		Interno & R2F1.1 \newline R2F2.1 \newline R2F2.2 \newline R1F3 \newline R2F4.2 \newline R2F4.4 \newline R2F4.5 \newline R2F4.6 \newline R2F5.1 \newline R0F5.3 \newline R2F5.4 \newline R2F5.5 \newline R2F6.1 \newline R1F8 \newline R1F9 \newline R2F10 \newline R0F11 \newline R2Q1 \newline R2Q2.1 \newline R2Q4 \newline R2Q5.1 \newline R2Q6 \newline R2Q7 \newline R2V1.1 \newline R2V1.2 \newline R2V1.3 \newline R2V2 \newline R2V2.1 \newline R2V2.2 \newline R2V2.3 \newline R2V2.4 \newline R2V3.1 \newline R2V4.1 \newline R2P1 \newline R2P2 \\
		Esterno	& R2F10 \newline R2V6 \\
		UC1 & R1F3 \\
		UC2 & R2F2 \\
		UC3 & R2F2 \\
		UC4 & R2F2.2 \\
		UC5 & R2F1 \\
		UC5.1.1 & R2F1.1 \\
		UC5.1.2 & R2F1.1 \\
		UC5.2 & R2F1 \\
		UC5.3 & R2F1 \\
		UC6 & R2F2.1 \\
		UC6.2 & R2F2.3 \\
		UC7 & R1F3 \\
		UC8 & R2F4 \newline R2F4.3 \newline R2V7 \\
		UC8.1 & R2F4 \\
		UC8.2 & R2F4 \newline R2F4.4 \\
		UC8.3 & R2F4.5 \\
		UC9 & R2F5 \\
		UC9.2 & R2F5.1 \\
		UC10 & R2F7 \\
		UC11 & R1F8 \\
		UC12 & R1F9 \\
		UC13 & R2F10 \\
		UC14 & R0F11 \\
		\rowcolor{white}
		\caption{Tabella tracciamento fonti-requisiti}
	\end{longtable}
\end{center}

	
\subsection{Riepilogo}
Riepilogo
\begin{center}
	\renewcommand{\arraystretch}{1.5}
	\rowcolors{3}{tableLightYellow}{}
	\begin{longtable}{  p{1.8cm}  p{2cm} p{2.2cm}  p{1.8cm} p{1.2cm}}
		\rowcolor{tableHeadYellow}
		\textbf{Tipologia}   & \textbf{Obbligatori} & \textbf{Desiderabili} & \textbf{Opzionali} & \textbf{Totale}\\ 
		Funzionali    & 21          & 4            & 2         & 27     \\
		Di qualità    & 9           & 0            & 0         & 9      \\  
		Di vincolo    & 14          & 2            & 0         & 16     \\  
		Prestazionali & 1           & 1            & 0         & 2      \\  
		Totale        & 45          & 7            & 2         & 54     \\
		\rowcolor{white}
		\caption{Tabella \markg{requisiti} prestazionali}
	\end{longtable}
\end{center}
