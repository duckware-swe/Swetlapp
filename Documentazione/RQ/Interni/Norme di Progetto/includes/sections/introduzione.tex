\section{Introduzione}
\label{sec:intro}
\subsection{Scopo del documento}
Lo scopo di questo documento è quello di fissare tutte le regole e le metodologie da applicare durante la realizzazione del progetto di modo che il gruppo possa lavorare seguendo certi standard ben definiti. In particolare, verranno definiti gli strumenti da utilizzare e le procedure da applicare durante le varie fasi dello sviluppo del \markg{software}. In tal modo ci sarà una efficiente collaborazione tra i membri.
\subsection{Natura del documento}
Il presente documento non può essere considerato completo, in quanto sarà revisionato e incrementato nel suo contenuto ad ogni revisione di progettazione nelle rispettive sezioni durante i periodi di lavoro e sviluppo.
\subsection{Scopo del prodotto}
L'obiettivo del prodotto è la realizzazione di un'applicazione per smartphone, nello specifico per la piattaforma \markg{Android}, che permetta la creazione di \markg{workflow} per l'assistente vocale \markg{Amazon} \markg{Alexa}. Il \markg{back-end} sarà realizzato con \markg{Java} e \markg{Node.js} opportunamente integrati con le \markg{API} di \markg{AWS}, per il \markg{front-end} verrà utilizzato \markg{XML} per stabilire i layout e Java per gestirne il comportamento. Si parlerà del front-end dell'assistente vocale riferendosi a \markg{VUI}.
\subsection{Glossario}
Nel documento ci sono dei termini con un significato ambiguo a seconda del contesto nel quale sono stati utilizzati. Per ovviare a questo problema è presente il documento \emph{Glossario v2.0.0} che conterrà una lista di termini con la specifica descrizione del suo significato. La prima occorrenza di un qualsiasi termine presente nel glossario verrà indicata in questo documento scritta in corsivo con una G a pedice che seguirà la parola in questione.
\subsection{Riferimenti utili}
\subsubsection{Riferimenti informativi}
\begin{itemize}
	\item Sito del corso di Ingegneria del Software\footnote{\href{https://www.math.unipd.it/~tullio/IS-1/2018/}{https://www.math.unipd.it/~tullio/IS-1/2018/}}
	\item ISO/IEC15504\footnote{\href{http://www.colonese.it/SviluppoSw_Standard_ISO15504.html}{http://www.colonese.it/SviluppoSw\_{}Standard\_{}ISO15504.html}}
\end{itemize}
