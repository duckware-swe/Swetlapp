\clearpage
\section{Resoconto}
	\subsection{Elenco riassuntivo delle correzioni}
	\label{sec:elenco_riassunto}
		\begin{center}
			\renewcommand{\arraystretch}{1.5}
			\rowcolors{3}{tableLightYellow}{}
			\begin{longtable}{  p{3cm} p{11.2cm}  }
				\rowcolor{tableHeadYellow}
				\textbf{Identificativo correzione} & \textbf{Descrizione}\\
				C1 & 	 	Glossario
								\begin{itemize}
									\item correzione dell'indice del documento
									\item cambiare l'impaginatura creando una nuova pagina al termine di una lettera per quella successiva
									\item creare i \textit{Bookmark}
								\end{itemize}
				\\		
				C2 & 		Verbali
								\begin{itemize}
									\item rendere i verbali tracciabili tra loro catalogando le decisioni citate e/o prese 
									\item comunicare di più con il committente \textit{Zero12}
								\end{itemize}
				\\
				C3 & 		Registro delle modifiche
								\begin{itemize}
									\item correggere i changelog dei documenti nelle forme verbali e nelle citazioni dei capitoli
									\item creazione degli appositi comandi
								\end{itemize}
				\\
				C4 & 		Template dei documenti
								\begin{itemize}
									\item inserire il numero totale di pagine e non solo il numero di pagina
								\end{itemize}
				
				\\
				C5 & 		Studio di Fattibilità
								\begin{itemize}
									\item correzione del titolo nel capitolo §3 del documento
								\end{itemize}
				
				\\
				C6 & 		Analisi dei Requisiti
								\begin{itemize}
									\item correzione dei punti segnalati dalla valutazione della RR
								\end{itemize}
				\\
				C7 & 		Norme di Progetto
								\begin{itemize}
									\item correggere e inserire una norma per i titoli dei documenti
									\item inserire delle note a piè di pagina per spiegare alcuni punti nel documento
									\item correggere la parte relativa agli strumenti, che vanno citati e spiegati nei processi o nel contesto che li utilizzano
								\end{itemize}
				\\
				C8 & 		Piano di Progetto
								\begin{itemize}
									\item correzione del capitolo §2 del documento
									\item strutturare meglio la pianificazione delle revisioni future
									\item strutturare meglio la pianificazione dello sviluppo del prodotto
								\end{itemize}
				\\
				C9 & 		Piano di Qualifica
								\begin{itemize}
									\item aggiunta di nuove metriche per il software, e se necessario anche per i documenti
									\item correzione dei capitolo §3 e §4 del documento
									\item spiegare in maniera più precisa il modello utilizzato nel capitolo §5 del documento
								\end{itemize}
				\\
				
				\rowcolor{white}
				\caption{Tabella riassuntiva delle correzioni}
				\label{sec:tabella_riassuntiva1}
			\end{longtable}	
		\end{center}
	
\clearpage
	\subsection{Organizzazione e suddivisione del lavoro}
	\label{sec:oganizzazione_lavoro}
	Il lavoro è stato suddiviso e assegnato a membri del team, suddividendo il carico in modo equo, con l'obiettivo di migliorare i documenti correggendo le lacune segnalate in Revisione dei Requisiti coinvolgendo tutti i partecipanti del gruppo.

	\subsection{Scadenza}
	\label{sec:scadenza}
	A fine riunione è stata scelta una data per la conclusione di tale correzione dei documenti presentati in Revisione dei Requisiti e organizzata la data per la prossima riunione del gruppo \textit{duckware}.
	
	\begin{center}
		\renewcommand{\arraystretch}{1.5}
		\rowcolors{3}{tableLightYellow}{}
		\begin{longtable}{  p{6cm} p{5cm}  }
			\rowcolor{tableHeadYellow}
			\textbf{Obiettivo} & \textbf{Data di scadenza}\\
			Termine correzione dei documenti & 18-02-2019 \\
			Prossimo incontro per il gruppo & 18-02-2019 (da confermare) \\
			\rowcolor{white}
			\caption{Tabella riassuntiva delle scadenza}
			\label{sec:tabella_riassuntiva2}
		\end{longtable}
	\end{center}
