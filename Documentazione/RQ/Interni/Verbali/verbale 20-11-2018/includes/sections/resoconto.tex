\clearpage
\section{Resoconto}
\subsection{Discussione dei capitolati}
	\label{sec:discussione}
	Prima della riunione tutti i membri del gruppo hanno compilato una tabella elettronica, creata e condivisa dal membro del gruppo \emph{Matteo Pellanda}, indicando i capitolati che hanno suscitato interesse durante l'esposizione in aula il 16/11/2018. \\
	Durante la  riunione, visionando la tabella, sono stati analizzati tutti  i  capitolati proposti escludendo alcuni di essi motivandone le ragioni al fine di ridurre il campo di scelte. \\
	Alla termine della discussione è stata fatta una votazione unanime e la scelta é ricaduta sul \emph{capitolato C4} con il progetto \emph{Meg\markg{Alexa}} proposto dall’azienda \emph{Zero12}. \\ 
	Il gruppo ha ritenuto interessante e stimolante l’utilizzo della tecnologia \markg{AWS} di \markg{Amazon} sul suo ultimo prodotto lanciato in commercio, \markg{Amazon} \markg{Alexa}.
	\subsection{Nome e logo gruppo}
	\label{sec:nome_logo}
	Durante l'incontro sono state ascoltate alcune idee sul nome e logo del gruppo, per ora identificato come \emph{gruppo 3}. Sono emersi pareri discordanti e con scarsi risultati è stato deciso di rimandare la scelta di questa decisione alla riunione successiva.
	\subsection{Organizzazione}
	\label{sec:organizzazione}
	Per facilitare e coordinare al meglio il lavoro è stato scelto di adottare i seguenti strumenti organizzativi: 
	\begin{itemize}
		\item \textbf{\markg{Google Drive}} per condividere file e cartelle attraverso la piattaforma, il componente \emph{Matteo Pellanda} ha realizzato un foglio accessibili a tutti i membri del gruppo per organizzare le prime organizzazioni del gruppo;
		\item \textbf{\markg{Slack}} per comunicare con i membri del gruppo per mezzo di canali appositamente creati per la tematica affrontata;
		\item \textbf{\markg{GitLab}} per versionare i prodotti che il gruppo andrà a realizzare.
	\end{itemize}