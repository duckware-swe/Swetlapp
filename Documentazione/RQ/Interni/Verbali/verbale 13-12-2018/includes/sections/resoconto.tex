\clearpage
\section{Resoconto}
	\subsection{Elenco domande a Zero12}
	\label{sec:elencodomande}
		\begin{center}
			\renewcommand{\arraystretch}{1.5}
			\rowcolors{3}{tableLightYellow}{}
			\begin{longtable}{  p{3cm} p{11cm} }
				\rowcolor{tableHeadYellow}
				\textbf{Identificativo}&\textbf{Domanda}\\
				D1 & Il team desidererebbe una risposta esplicita per quanto riguarda le esigenze del committente: nonostante le richieste di chiarimenti  da parte di Duckware, non è ancora perfettamente chiaro se la piattaforma dovrà  essere prettamente web, mobile o entrambe le soluzioni.
				\begin{itemize}
					\item Nel particolare se possibile l'uso di linguaggio \emph{cross-platform nativo} – Delphi
				\end{itemize}
				\\
				D2 & Come testare l’App o Web con \markg{Alexa}\\
				D3 & Implementazione
				\begin{itemize}
					\item All'affermazione \emph{“You pass back a graphical response”} è possibile interpretare la richiesta con una notifica di risposta al \markg{workflow}?
					\item Il login e la registrazione avvengono con un account \markg{Amazon} \markg{Alexa} oppure con un nuovo account che verrà in seguito collegato ad \markg{Amazon} \markg{Alexa}?
					\item Database: è messo a disposizione un database \markg{AWS} o un altro DBMS?
					\item Come vengono gestiti gli \markg{error}i se un \markg{workflow} non termina correttamente?
					\item Cosa succede se il comando è ambiguo e non viene interpretato?
				\end{itemize}
				\\
				\rowcolor{white}
				\caption{Tabella riassuntiva domande}
				\label{sec:tabella_riassuntiva1}
			\end{longtable}	
		\end{center}
	
	\subsection{Email a Zero12}
	\label{sec:mail}
	Di seguito viene lasciato il contenuto della mail:
	\begin{quote}
		\textit{
	 	Grazie per la cortese risposta.
	 	Domani saremo presenti all'incontro con Zero12 "Da \markg{software} a voice interface" e, se disponibili a parlare, avremo delle domande da porvi sul progetto.\\[0.5cm]
	 	Vi ringraziamo \\
	 	Team Duckware
	 	}
	\end{quote}
	\clearpage
	\subsection{Tabella scadenza documenti - RR}
	\label{sec:scadenze}
	Viene formulato un tabellario con le scadenze per la stesura dei singoli documenti RR:
	\begin{center}
		\renewcommand{\arraystretch}{1.5}
		\rowcolors{3}{tableLightYellow}{}
		\begin{longtable}{  p{3cm} p{5cm} p{2cm} p{2cm} }
			\rowcolor{tableHeadYellow}
			\textbf{Identificativo}&\textbf{Documento}&\textbf{Inizio}&\textbf{Fine}\\
			SD1 & Analisi dei Requisiti & 13-12-2018 & 01-01-2019\\
			SD2 & Piano di Progetto & 01-01-2019 & 06-01-2019\\
			SD3 & Piano di Qualifica & 01-01-2019 & 10-01-2019\\
			\rowcolor{white}
			\caption{Tabella riassuntiva scadenze}
			\label{sec:tabella_riassuntiva3}
		\end{longtable}	
	\end{center}