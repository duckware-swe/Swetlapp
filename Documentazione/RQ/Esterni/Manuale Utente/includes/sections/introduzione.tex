\clearpage
\section{Introduzione}
\label{sec:sec_introduzione}
\subsection{Scopo del documento}
\label{sec:subsec_scopodocumento}
Lo scopo del presente documento è quello di dare delle indicazioni sul corretto uso del prodotto \textit{SwetlApp} realizzato dal team \textit{duckware}, disponibile per il sistema operativo \markg{Android} e per i dispositivi \markg{Amazon Alexa}. Si forniscono, inoltre, delle indicazioni sul modo in cui si può interagire con l'assistente vocale \textit{Amazon Alexa}.
Il presente documento verrà sviluppato e modificato seguendo lo sviluppo del prodotto finale \textit{MegAlexa}.
\subsection{Scopo del prodotto}
\label{sec:subsec_scopoprodotto}

%L'obiettivo del prodotto è la realizzazione di una Skill per l'assistente vocale Amazon Alexa in grado di eseguire dei \markg{workflow} e di un'applicazione per smartphone, nello specifico per la piattaforma \textit{Android}, che permetta la loro creazione e gestione. Il \markg{back-end} dell'applicazione per smartphone \textit{Android} sarà realizzato in \markg{Java} opportunamente integrato con le \markg{API} di \markg{Amazon Web Services}, per il \markg{front-end} verrà usato \markg{XML} per stabilire i \markg{layout} e \textit{Java} per gestirne il comportamento. Mentre il \textit{back-end} della \textit{Skill} è stato realizzato in \markg{NodeJS} con l'utilizzo di \markg{JSON} per lo scambio di informazioni. Si parlerà inoltre del \textit{front-end} dell'assistente vocale riferendosi a \markg{VUI} (voice user interface).

L'obiettivo del prodotto è la realizzazione di un'applicazione per smartphone, nello specifico per la piattaforma \markg{Android}, che permetta la creazione di \markg{workflow} per l'assistente vocale \markg{Amazon} \markg{Alexa}. Il \markg{back-end} sarà realizzato in \markg{Java} e \markg{Node.js} opportunamente integrati con le \markg{API} di \markg{AWS}, per il \markg{front-end} verrà utilizzato \markg{XML} per stabilire i layout e Java per gestirne il comportamento. Si parlerà del front-end dell'assistente vocale riferendosi a \markg{VUI}.

\subsection{Riferimenti}
\label{sec:subsec_riferimenti}
\begin{itemize}
	\item Android\\ \href{https://www.android.com/}{https://www.android.com/}
	\item BlueStacks\\ \href{https://www.bluestacks.com/it/index.html}{https://www.bluestacks.com/it/index.html}
%	\item Android Studio\\ \href{https://developer.android.com/studio}{https://developer.android.com/studio}
	\item Alexa Skill\\ \href{https://www.amazon.it/b?ie=UTF8\&node=13944605031}{https://www.amazon.it/b?ie=UTF8\&node=13944605031}
	%\item Alexa Skill Kit\\ \href{https://developer.amazon.com/alexa-skills-kit/}{https://developer.amazon.com/alexa-skills-kit/}
\end{itemize}