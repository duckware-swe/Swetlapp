\clearpage
\section{Introduzione}
\label{sec:intro}
\subsection{Scopo del documento}
Lo scopo del seguente documento consiste nel presentare le norme utilizzate dal Gruppo \emph{duckware} adottate per la \markg{verifica} e \markg{validazione} dei prodotti e dei \markg{processi}. Per raggiungere lo scopo prefissato e il risultato desiderato, i processi e i prodotti realizzati verranno sottoposti a verifica continua affinché non vengano introdotti errori che ne influiscano il risultato finale in maniera negativa con l'uso di strategie e metriche di seguito descritte.
\subsection{Natura del documento}
Il presente documento non può essere considerato completo, in quanto sarà revisionato e incrementato nel suo contenuto ad ogni revisione di progettazione nelle rispettive sezioni durante i periodi di lavoro e sviluppo.
\subsection{Scopo del prodotto}
L'obiettivo del prodotto è la realizzazione di un'applicazione per smartphone, nello specifico per la piattaforma \markg{Android}, che permetta la creazione di \markg{workflow} per l'assistente vocale \markg{Amazon} \markg{Alexa}.\newline
Il \markg{back-end} sarà realizzato in \markg{Java} e \markg{Node.js} opportunamente integrati con le \markg{API} di \markg{AWS}, per il \markg{front-end} verrà utilizzato \markg{XML} per stabilire i layout e Java per gestirne il comportamento. Si parlerà del front-end dell'assistente vocale riferendosi a \markg{VUI}.
\subsection{Glossario}
Nel documento sono presenti termini che possono assumere significati ambigui a seconda del contesto o termini non conosciuti. Per ovviare a questa problematica è stato creato un Glossario contente tali termini con il loro significato specifico. Un termine è presente all'interno del \emph{Glossario v3.0.0} se seguito da una G corsiva a pedice.
\subsection{Riferimenti}
\label{sec:ref}
\subsubsection{Riferimenti normativi}
\begin{itemize}
	\item Norme di Progetto v3.0.0
\end{itemize}
\subsubsection{Riferimenti informativi}
\begin{itemize}
	\item
		Verifica e Validazione: introduzione - Slide del corso di Ingegneria del Software\footnote{\href{https://www.math.unipd.it/~tullio/IS-1/2018/Dispense/L16.pdf}{https://www.math.unipd.it/~tullio/IS-1/2018/Dispense/L16.pdf}}
	\item
		Qualità del Software - Slide del corso di Ingegneria del Software\footnote{\href{https://www.math.unipd.it/~tullio/IS-1/2018/Dispense/L13.pdf}{https://www.math.unipd.it/~tullio/IS-1/2018/Dispense/L13.pdf}}	
	\item
		Qualità di Processo - Slide del corso di Ingegneria del Software\footnote{\href{https://www.math.unipd.it/~tullio/IS-1/2018/Dispense/L14.pdf}{https://www.math.unipd.it/~tullio/IS-1/2018/Dispense/L14.pdf}}
	\item
		Processi SW - Slide del corso di Ingegneria del Software\footnote{\href{https://www.math.unipd.it/~tullio/IS-1/2018/Dispense/L03.pdf}{https://www.math.unipd.it/~tullio/IS-1/2018/Dispense/L03.pdf}}
	\item 
		ISO/IEC 15504\footnote{\href{http://www.colonese.it/SviluppoSw_Standard_ISO15504.html}{http://www.colonese.it/SviluppoSw\_{}Standard\_{}ISO15504.html}}
	\item 
		Indice di Gulpease\footnote{\href{https://it.wikipedia.org/wiki/Indice_Gulpease}{https://it.wikipedia.org/wiki/Indice\_{}Gulpease}}
		\begin{itemize}
			\item Descrizione e formula di calcolo.
		\end{itemize}
\end{itemize}
