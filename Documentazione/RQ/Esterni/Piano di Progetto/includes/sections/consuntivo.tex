\clearpage
\section{Consuntivo di periodo e preventivo a finire}
\label{sec:consuntivo}
Di seguito vengono riportati i dettagli dei consuntivi che mostrano le ore lavorative e le spese effettive.
Ciascuna attività viene valutata in relazione al preventivo precedentemente stilato; per ciascuna viene mostrato un bilancio, che potrà essere:
\begin{enumerate}
	\item Positivo: il numero di ore lavorative o la spesa effettiva è stato inferiore a quello preventivato;
	\item Negativo: il numero di ore lavorative o la spesa effettiva è stato superiore a quello preventivato;
	\item In pari: il consuntivo rispetta il preventivo.
\end{enumerate}
\subsection{Analisi}
\label{sec:analisi}
\subsubsection{Consuntivo}
La tabella sottostante indica le ore di lavoro sostenute dai vari membri del gruppo durante il periodo di Analisi:
\begin{center}
	\renewcommand{\arraystretch}{1.5}
	\rowcolors{3}{tableLightYellow}{}
	\begin{longtable}[H]{ 	>{\RaggedRight}p{3.5cm}  
							>{\Centering}p{1.2cm} 
							>{\Centering}p{1.2cm}  
							>{\Centering}p{1.2cm} 
							>{\Centering}p{1.2cm}  
							>{\Centering}p{1.2cm} 
							>{\Centering}p{1.2cm}  
							>{\Centering}p{1.4cm}  
							}
		\rowcolor{tableHeadYellow}
		\textbf{Nome}   & \textbf{Re} & \textbf{Ad} & \textbf{An} & \textbf{Pj} & \textbf{Pr} & \textbf{Ve} & \textbf{TOT} \\ 
		\endhead

		Luca Stocco       & 0   	& 10    & 5 (-2) 	& 0   & 0   & 10 		& 25 (-2) \\  
		Alberto Miola     & 3   	& 10(-2)  	& 4   	& 0   & 0   & 8 (+3) 	& 25 (+1) \\  
		Andrea Pavin      & 3 (-1) 	& 5  	& 12(+2) 	& 0   & 0   & 8 (+1) 	& 25 (+2) \\  
		Sonia Menon       & 4  		& 9     & 3  (-3)	& 0   & 0   & 10(-1)  	& 25 (-4) \\  
		Pardeep Singh     & 8 (+3)  & 2     & 10  		& 0   & 0   & 4  		& 25 (+3) \\  
		Matteo Pellanda   & 4   	& 8 (-2)   & 7   	& 0   & 0   & 3 (-2)	& 25 (-4) \\
		Alessandro Pegoraro	& 0		& 10(-1)	& 9	  	& 0	  & 0	& 6	(-1)	& 25 (-2) \\

		\rowcolor{white}
		\caption{Tabella consuntivo Analisi}
	\end{longtable}
\end{center}
Di seguito è mostrato il prospetto economico consuntivo del periodo di Analisi:
\begin{center}
	\renewcommand{\arraystretch}{1.5}
	\rowcolors{3}{tableLightYellow}{}
	\begin{longtable}{  >{\RaggedRight}p{5.6cm}  
						>{\RaggedRight}p{3cm} 
						>{\RaggedRight}p{3cm}  
						}
		\rowcolor{tableHeadYellow}
		\textbf{Ruolo}   & \textbf{Ore} & \textbf{Costo (Euro)} \\ 
		\endhead

		Responsabile   & 22 (+2)	& 660,00 (+60) \euro \\
		Amministratore & 54 (-5)	& 1.080,00 (-100) \euro \\
		Analista       & 50 (-3)	& 1.250,00 (-75) \euro\\
		Progettista    & 0    		& 0,00 \euro \\
		Programmatore  & 0    		& 0,00 \euro \\
		Verificatore   & 49 	  	& 735,00 \euro\\
		Totale         & 175 (-6)  	& 3.725,00 (-115) \euro \\

		\rowcolor{white}
		\caption{Tabella consuntivo Analisi}
	\end{longtable}
\end{center}
\subsubsection{Conclusioni}
Nel periodo di Analisi ci sono state delle variazioni rispetto alle aspettative: sono risultate necessarie più ore per i ruoli di Amministratore e Analista dovute a una maggior necessità di ricerca ed approfondimenti per redigere i documenti; per il ruolo di Responsabile c'è stata invece una spesa lievemente minore a quella stimata. Nel complesso ci si è discostati di 6 ore lavorative dal preventivo. Questo scostamento non ha portato significativi cambiamenti o ritardi; la spesa aggiuntiva risultante dal consuntivo non influirà sul totale rendicontato al committente, trattandosi della fase di investimento.
\subsection{Consolidamento dei requisiti}
\label{sec:consolidamento_dei_requisiti}
\subsubsection{Consuntivo}
La tabella sottostante indica le ore di lavoro sostenute dai vari membri del gruppo durante il periodo di Consolidamento dei requisiti:
\begin{center}
	\renewcommand{\arraystretch}{1.5}
	\rowcolors{3}{tableLightYellow}{}
	\begin{longtable}[H]{ 	>{\RaggedRight}p{3.5cm}  
							>{\Centering}p{1.2cm} 
							>{\Centering}p{1.2cm}  
							>{\Centering}p{1.2cm} 
							>{\Centering}p{1.2cm}  
							>{\Centering}p{1.2cm} 
							>{\Centering}p{1.2cm}  
							>{\Centering}p{1.4cm}  
							}
		\rowcolor{tableHeadYellow}
		\textbf{Nome}   & \textbf{Re} & \textbf{Ad} & \textbf{An} & \textbf{Pj} & \textbf{Pr} & \textbf{Ve} & \textbf{TOT} \\ 
		\endhead

		Luca Stocco       & 6 & 1 & 0 & 0 & 0 & 0 & 7 \\  
		Alberto Miola     & 0 & 2 & 0 & 0 & 0 & 5 & 7 \\  
		Andrea Pavin      & 0 & 0 & 4 & 0 & 0 & 3 & 7 \\  
		Sonia Menon       & 0 & 5 & 2 & 0 & 0 & 0 & 7 \\  
		Pardeep Singh     & 0 & 0 & 0 & 0 & 0 & 7 & 7 \\  
		Matteo Pellanda   & 0 & 0 & 7 & 0 & 0 & 0 & 7 \\
		Alessandro Pegoraro	& 0	& 2 & 3	& 0	& 0	& 2	& 7 \\

		\rowcolor{white}
		\caption{Tabella consuntivo Consolidamento dei requisiti}
	\end{longtable}
\end{center}
Di seguito è mostrato il prospetto economico consuntivo del periodo di Consolidamento dei requisiti:
\begin{center}
	\renewcommand{\arraystretch}{1.5}
	\rowcolors{3}{tableLightYellow}{}
	\begin{longtable}{  >{\RaggedRight}p{5.6cm}  
						>{\RaggedRight}p{3cm} 
						>{\RaggedRight}p{3cm}  
						}
		\rowcolor{tableHeadYellow}
		\textbf{Ruolo}   & \textbf{Ore} & \textbf{Costo (Euro)} \\ 
		\endhead
		Responsabile   & 26	& 180,00 \euro \\
		Amministratore & 10 & 160,00 \euro \\
		Analista       & 16 & 325,00 \euro\\
		Progettista    & 0  & 0,00 \euro \\
		Programmatore  & 0 	& 0,00 \euro \\
		Verificatore   & 17 & 225,00 \euro \\
		Totale         & 49 & 890,00 \euro \\
		
		\rowcolor{white}
		\caption{Tabella consuntivo Consolidamento dei requisiti}
	\end{longtable}
\end{center}
\subsubsection{Conclusioni}
Nel periodo di Consolidamento dei requisiti il gruppo ha revisionato e migliorato l'\emph{Analisi dei Requisiti} e si è preparato per la presentazione della \emph{Revisione dei Requisti}. La spesa aggiuntiva risultante dal consuntivo non influirà sul totale rendicontato al committente, trattandosi della fase di investimento.
\subsection{Technology Baseline}
\label{sec:technology_baseline}
\subsubsection{Consuntivo}
La tabella sottostante indica le ore di lavoro sostenute dai vari membri del gruppo durante il periodo di technology baseline:
\begin{center}
	\renewcommand{\arraystretch}{1.5}
	\rowcolors{3}{tableLightYellow}{}
	\begin{longtable}[H]{ 	>{\RaggedRight}p{3.5cm}  
							>{\Centering}p{1.2cm} 
							>{\Centering}p{1.2cm}  
							>{\Centering}p{1.2cm} 
							>{\Centering}p{1.2cm}  
							>{\Centering}p{1.2cm} 
							>{\Centering}p{1.2cm}  
							>{\Centering}p{1.4cm}  
							}
		\rowcolor{tableHeadYellow}
		\textbf{Nome}   & \textbf{Re} & \textbf{Ad} & \textbf{An} & \textbf{Pj} & \textbf{Pr} & \textbf{Ve} & \textbf{TOT} \\ 
		\endhead

		Luca Stocco         & 2 (-2) & 0      & 0 	   & 7 (+1)	& 0 	& 5 (+2) & 14 (+1) \\  
		Alberto Miola       & 0      & 0      & 2  	   & 9 (+1) & 0   	& 1  	 & 12 (+1) \\  
		Andrea Pavin        & 3      & 4 (-1) & 0      & 4  	& 0   	& 4 (+1) & 15 \\  
		Sonia Menon         & 0      & 3      & 2      & 4 (-1) & 0  	& 2 (-2) & 11 (-3) \\  
		Pardeep Singh       & 0      & 2 (+1) & 0      & 7 (-2) & 0  	& 2 	 & 11 (-1) \\  
		Matteo Pellanda     & 3 (+2) & 0      & 2 (+1) & 6   	& 0  	& 3  	 & 14 (+3) \\ 
		Alessandro Pegoraro & 0      & 2 (-1) & 1	   & 7		& 0 	& 1 (-1) & 11 (-2) \\ 

		\rowcolor{white}
		\caption{Tabella consuntivo Technology Baseline}
	\end{longtable}
\end{center}
Di seguito è mostrato il prospetto economico consuntivo del periodo di technology baseline:
\begin{center}
	\renewcommand{\arraystretch}{1.5}
	\rowcolors{3}{tableLightYellow}{}
	\begin{longtable}{  >{\RaggedRight}p{5.6cm}  
						>{\RaggedRight}p{3cm} 
						>{\RaggedRight}p{3cm}  
						}
		\rowcolor{tableHeadYellow}
		\textbf{Ruolo}   & \textbf{Ore} & \textbf{Costo (Euro)} \\ 
		\endhead

		Responsabile   & 8       & 240,00 \\
		Amministratore & 11 (-1) & 220,00 \\
		Analista       & 7 (+1)  & 175,00 \\
		Progettista    & 44 (-1) & 968,00 \\
		Programmatore  & 0       & 0,00 \\
		Verificatore   & 18      & 270,00 \\
		Totale         & 81 (-1) & 1.873,00 \\

		\rowcolor{white}
		\caption{Tabella consuntivo Technology Baseline}
	\end{longtable}
\end{center}
\subsubsection{Conclusioni}
Nel periodo di Technology Baseline il gruppo ha definito l'architettura del prodotto, e ha migliorato la stesura di alcuni documenti come \emph{Analisi dei Requisiti}. Nel complesso il gruppo ha risparmiato 1 ora.%dovuto riorganizzare la distribuzione del lavoro dato che un componente non ha potuto fornire il suo contributo alla progettazione e programmazione del PoC(proof of concept). Il resto del gruppo ha quindi dovuto sopperire al codice mancante del membro, aumentando le ore che ogni altro componente ha impiegato per la Progettazione e Programmazione.
%Sono state rendicontate anche le ore impiegate dai Verificatori e dal Responsabile per controllare e approvare il codice prodotto dai Programmatori.
%Inoltre sono state necessarie più ore per i Progettisti e i Programmatori per la preparazione e lo studio dei documenti riguardanti la console AWS.
\subsection{Sprint 1}
\label{sec:sprint_1}
\subsubsection{Consuntivo}
La tabella sottostante indica le ore di lavoro sostenute dai vari membri del gruppo durante il periodo del primo sprint:
\begin{center}
	\renewcommand{\arraystretch}{1.5}
	\rowcolors{3}{tableLightYellow}{}
	\begin{longtable}[H]{ 	>{\RaggedRight}p{3.5cm}  
							>{\Centering}p{1.2cm} 
							>{\Centering}p{1.2cm}  
							>{\Centering}p{1.2cm} 
							>{\Centering}p{1.2cm}  
							>{\Centering}p{1.2cm} 
							>{\Centering}p{1.2cm}  
							>{\Centering}p{1.4cm}  
							}
		\rowcolor{tableHeadYellow}
		\textbf{Nome}   & \textbf{Re} & \textbf{Ad} & \textbf{An} & \textbf{Pj} & \textbf{Pr} & \textbf{Ve} & \textbf{TOT} \\ 
		\endhead

		Luca Stocco         & 3   & 0     	& 2   & 2      & 6 (+2)	& 7 (-2)   	& 20 \\  
		Alberto Miola       & 0   & 0     	& 2   & 0      & 12  	& 4  	    & 18 \\  
		Andrea Pavin        & 3   & 3 (+1) 	& 0   & 2      & 5   	& 7  	    & 20 (+1) \\  
		Sonia Menon         & 0   & 1     	& 0   & 1      & 8   	& 7 	    & 17 \\  
		Pardeep Singh       & 0   & 4 (+2)  & 0   & 0 (-1) & 0 (-4) & 8         & 12 (-3) \\  
		Matteo Pellanda     & 2   & 0     	& 1   & 0      & 7   	& 6 	    & 16 \\
		Alessandro Pegoraro & 0   & 1	  	& 2	  & 0      & 8 (+2)	& 5 	    & 16 (+2) \\ 

		\rowcolor{white}
		\caption{Tabella consuntivo Sprint 1}
	\end{longtable}
\end{center}
Di seguito è mostrato il prospetto economico consuntivo del periodo del primo sprint:
\begin{center}
	\renewcommand{\arraystretch}{1.5}
	\rowcolors{3}{tableLightYellow}{}
	\begin{longtable}{  >{\RaggedRight}p{5.6cm}  
						>{\RaggedRight}p{3cm} 
						>{\RaggedRight}p{3cm}  
						}
		\rowcolor{tableHeadYellow}
		\textbf{Ruolo}   & \textbf{Ore} & \textbf{Costo (Euro)} \\ 
		\endhead

		Responsabile   & 8       & 240,00 \\
		Amministratore & 9 (+3)  & 180,00 \\
		Analista       & 7   	 & 175,00 \\
		Progettista    & 5 (-1)  & 110,00 \\
		Programmatore  & 46  	 & 690,00 \\
		Verificatore   & 44 (-2) & 660,00 \\
		Totale         & 119 	 & 2.055,00 \\

		\rowcolor{white}
		\caption{Tabella consuntivo Sprint 1}
	\end{longtable}
\end{center}
\subsubsection{Conclusioni}
Nel periodo del primo Sprint il gruppo ha dovuto riorganizzare la distribuzione del lavoro dato che un componente non ha potuto fornire il suo contributo alla progettazione e programmazione del PoC(proof of concept). Il resto del gruppo ha quindi dovuto sopperire al codice mancante del membro, aumentando le ore che ogni altro componente ha impiegato per la Progettazione e Programmazione.
Sono state rendicontate anche le ore impiegate dai Verificatori per controllare e approvare il codice prodotto dai Programmatori.
Inoltre sono state necessarie più ore per gli Amministratori.
\subsection{Sprint 2}
\label{sec:sprint_2}
\subsubsection{Consuntivo}
La tabella sottostante indica le ore di lavoro sostenute dai vari membri del gruppo durante il periodo del secondo sprint:
\begin{center}
	\renewcommand{\arraystretch}{1.5}
	\rowcolors{3}{tableLightYellow}{}
	\begin{longtable}[H]{ 	>{\RaggedRight}p{3.5cm}  
							>{\Centering}p{1.2cm} 
							>{\Centering}p{1.2cm}  
							>{\Centering}p{1.2cm} 
							>{\Centering}p{1.2cm}  
							>{\Centering}p{1.2cm} 
							>{\Centering}p{1.2cm}  
							>{\Centering}p{1.4cm}  
							}
		\rowcolor{tableHeadYellow}
		\textbf{Nome}   & \textbf{Re} & \textbf{Ad} & \textbf{An} & \textbf{Pj} & \textbf{Pr} & \textbf{Ve} & \textbf{TOT} \\ 
		\endhead

		Luca Stocco         & 0      & 2     & 1 (+1) & 9      & 6 (+1) & 3   	 & 22 (+2) \\  
		Alberto Miola       & 2   	 & 2     & 1 (+1) & 8 (+1) & 6      & 5  	 & 24 (+2) \\  
		Andrea Pavin        & 0   	 & 0     & 2      & 9      & 7      & 6  	 & 24 \\  
		Sonia Menon         & 0   	 & 0     & 0      & 8      & 8      & 7 	 & 23 \\  
		Pardeep Singh       & 2 (+1) & 2     & 0 (-3) & 3 (-6) & 2 (-5) & 7 (+2) & 16 (-11) \\  
		Matteo Pellanda     & 0   	 & 1     & 2   	  & 8 (+1) & 6      & 4 	 & 21 (+1) \\
		Alessandro Pegoraro & 2 (-1) & 1	 & 2	  & 10     & 6	    & 4 	 & 25 (-1) \\ 

		\rowcolor{white}
		\caption{Tabella consuntivo Sprint 2}
	\end{longtable}
\end{center}
Di seguito è mostrato il prospetto economico consuntivo del periodo del secondo sprint:
\begin{center}
	\renewcommand{\arraystretch}{1.5}
	\rowcolors{3}{tableLightYellow}{}
	\begin{longtable}{  >{\RaggedRight}p{5.6cm}  
						>{\RaggedRight}p{3cm} 
						>{\RaggedRight}p{3cm}  
						}
		\rowcolor{tableHeadYellow}
		\textbf{Ruolo}   & \textbf{Ore} & \textbf{Costo (Euro)} \\ 
		\endhead

		Responsabile   & 6   	 & 180,00 \\
		Amministratore & 8   	 & 160,00 \\
		Analista       & 8 (-1)  & 200,00 \\
		Progettista    & 55 (-4) & 1.210,00 \\
		Programmatore  & 41 (-4)  & 615,00 \\
		Verificatore   & 36  & 540,00 \\
		Totale         & 154 (-9) & 2.905,00 \\

		\rowcolor{white}
		\caption{Tabella consuntivo Sprint 2}
	\end{longtable}
\end{center}
\subsubsection{Conclusioni}
Nel periodo del secondo Sprint il gruppo ha dovuto un altra volta riorganizzare la distribuzione del lavoro per la mancanza di un componente, da questa riorganizzazione sono avanzate 9 ore.
%Nel periodo di Progettazione della base tecnologica il gruppo ha dovuto riorganizzare la distribuzione del lavoro dato che un componente non ha potuto fornire il suo contributo alla progettazione e programmazione del PoC(proof of concept). Il resto del gruppo ha quindi dovuto sopperire al codice mancante del membro, aumentando le ore che ogni altro componente ha impiegato per la Progettazione e Programmazione.
%Sono state rendicontate anche le ore impiegate dai Verificatori e dal Responsabile per controllare e approvare il codice prodotto dai Programmatori.
%Inoltre sono state necessarie più ore per i Progettisti e i Programmatori per la preparazione e lo studio dei documenti riguardanti la console AWS.
\subsection{Sprint 3}
\label{sec:sprint_3}
\subsubsection{Consuntivo}
La tabella sottostante indica le ore di lavoro sostenute dai vari membri del gruppo durante il periodo del terzo sprint:
\begin{center}
	\renewcommand{\arraystretch}{1.5}
	\rowcolors{3}{tableLightYellow}{}
	\begin{longtable}[H]{ 	>{\RaggedRight}p{3.5cm}  
							>{\Centering}p{1.2cm} 
							>{\Centering}p{1.2cm}  
							>{\Centering}p{1.2cm} 
							>{\Centering}p{1.2cm}  
							>{\Centering}p{1.2cm} 
							>{\Centering}p{1.2cm}  
							>{\Centering}p{1.4cm}  
							}
		\rowcolor{tableHeadYellow}
		\textbf{Nome}   & \textbf{Re} & \textbf{Ad} & \textbf{An} & \textbf{Pj} & \textbf{Pr} & \textbf{Ve} & \textbf{TOT} \\ 
		\endhead

		Luca Stocco         & 0	& 2 & 0 & 7  & 12 & 3 & 24 \\  
		Alberto Miola       & 3	& 3	& 0	& 8	 & 5  & 8 & 27 \\  
		Andrea Pavin        & 0	& 0	& 1	& 7	 & 10 & 4 & 23 \\  
		Sonia Menon         & 0	& 0	& 0	& 8	 & 8  & 5 & 21 \\  
		Pardeep Singh       & 2	& 2	& 1	& 11 & 12 & 6 & 34 \\  
		Matteo Pellanda     & 0	& 3	& 1	& 11 & 8  & 6 & 29 \\
		Alessandro Pegoraro & 4	& 2	& 0	& 7	 & 7  & 7 & 27 \\ 

		\rowcolor{white}
		\caption{Tabella consuntivo Sprint 3}
	\end{longtable}
\end{center}
Di seguito è mostrato il prospetto economico consuntivo del periodo del terzo sprint:
\begin{center}
	\renewcommand{\arraystretch}{1.5}
	\rowcolors{3}{tableLightYellow}{}
	\begin{longtable}{  >{\RaggedRight}p{5.6cm}  
						>{\RaggedRight}p{3cm} 
						>{\RaggedRight}p{3cm}  
						}
		\rowcolor{tableHeadYellow}
		\textbf{Ruolo}   & \textbf{Ore} & \textbf{Costo (Euro)} \\ 
		\endhead

		Responsabile   & 9   & 270,00 \\
		Amministratore & 12  & 240,00 \\
		Analista       & 3   & 75,00 \\
		Progettista    & 59  & 1.298,00 \\
		Programmatore  & 62  & 930,00 \\
		Verificatore   & 39  & 585,00 \\
		Totale         & 178 & 3.398,00 \\

		\rowcolor{white}
		\caption{Tabella consuntivo Sprint 3}
	\end{longtable}
\end{center}
\subsubsection{Conclusioni}
Nel terzo Sprint il gruppo è riuscito a utilizzare le ore preventivate nella loro totalità senza andare incontro a rischi di alcun tipo.
%Nel periodo di Progettazione della base tecnologica il gruppo ha dovuto riorganizzare la distribuzione del lavoro dato che un componente non ha potuto fornire il suo contributo alla progettazione e programmazione del PoC(proof of concept). Il resto del gruppo ha quindi dovuto sopperire al codice mancante del membro, aumentando le ore che ogni altro componente ha impiegato per la Progettazione e Programmazione.
%Sono state rendicontate anche le ore impiegate dai Verificatori e dal Responsabile per controllare e approvare il codice prodotto dai Programmatori.
%Inoltre sono state necessarie più ore per i Progettisti e i Programmatori per la preparazione e lo studio dei documenti riguardanti la console AWS.
\subsection{Preventivo a finire}
\label{sec:preventivo_a_finire}
Viene di seguito presentata la tabella riportante il preventivo a finire. Il totale rendicontato, come detto in precedenza, non comprenderà i costi sostenuti nella fase di Analisi e Consolidamento, le quali verranno comunque riportate per chiarezza; i consuntivi non ancora definibili verranno considerati equivalenti al preventivo nel calcolo del totale.
\begin{center}
	\renewcommand{\arraystretch}{1.5}
	\rowcolors{3}{tableLightYellow}{}
	\begin{longtable}[H]{  	>{\RaggedRight}p{4cm}  
							>{\RaggedRight}p{2.5cm} 
							>{\RaggedRight}p{2.5cm}  
							}
		\rowcolor{tableHeadYellow}
		\textbf{Periodo}   & \textbf{Preventivo} & \textbf{Consuntivo} \\ 
		\endhead

		Analisi 	                      & 3.725,00 \euro  & 3.840,00 \euro \\
		Consolidamento dei Requisiti      & 890,00 \euro	& 890,00 \euro \\
		Technology Baseline               & 1.890,00 \euro	& 1.873,00 \euro \\
		Sprint 1						  & 2.047,00 \euro	& 2.055,00 \euro \\
		Sprint 2						  & 3.048,00 \euro	& 2.905,00 \euro \\
		Sprint 3						  & 3.398,00 \euro	& 3.398,00 \euro \\
		Sprint 4					      & 1.357,00 \euro 	& - \\
		Sprint 5					      & 1.330,00 \euro 	& - \\
		Totale                            & 17.685,00 \euro	& 14.961,00 \euro \\
		Totale Rendicontato	              & 13.070,00 \euro	& 10.231 \euro \\

		\rowcolor{white}
		\caption{Tabella preventivo a finire}
	\end{longtable}
\end{center}
