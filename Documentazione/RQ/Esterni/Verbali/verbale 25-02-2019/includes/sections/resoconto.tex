\clearpage
\section{Resoconto}
	\subsection{Elenco domande fatte al \emph{CEO Stefano Dindo di Zero12}}
	\label{sec:elenco}
	Nella tabella si riportano le domande preparate e le risposte ottenute dal \textbf{CEO S. Dindo} in risposta alla mail inviata.
		\begin{center}
			\renewcommand{\arraystretch}{1.5}
			\rowcolors{3}{tableLightYellow}{}
			\begin{longtable}{  p{2.5cm} p{11.7cm} }
				\rowcolor{tableHeadYellow}
				\textbf{Identificativo}&\textbf{Domanda e risposta}\\
				D1 & Come scelta implementativa per la registrazione dell'utente per l'uso della nostra app abbiamo scelto di fare una registrazione con una mail personale al nostro sistema. Si chiede una considerazione da parte vostra. Inoltre fatto ciò come è possibile successivamente collegare la mail con cui si è registrato l'utente e l'account personale di Amazon?
				\begin{itemize}
					\item \textbf{Risposta CEO S. Dindo}: Va bene la scelta che avete adottato. Successivamente per collegare le due mail basterà che l'utente acceda sul suo dispositivo Alexa con il suo account Amazon.
				\end{itemize}
				\\
				D2 & Abbiamo intenzione di utilizzare una chiave API Gateway unita ad un Piano di Utilizzo per poter limitare le richieste al secondo e le richieste totali al giorno. La scelta si pone sul vantaggio, rispetto a Cognito AWS, di poter gestire direttamente i dati degli utenti. Si chiede una considerazione da parte vostra.
				\begin{itemize}
					\item \textbf{Risposta CEO S. Dindo}: La scelta che proponete può andare bene, importante è che mantenete un livello accettabile di sicurezza.
				\end{itemize}
				\\
				D3 & Come prototipo del prodotto abbiamo implementato solo una parte delle funzionalità che la skill e l'app dovranno avere nel prodotto finale. L'applicazione mostra la base che intendiamo realizzare per la creazione e gestione dei workflow, mentre la skill presenta ciò che intendiamo per l'esecuzione di un workflow creato dall'app.
				\begin{itemize}
					\item \textbf{Risposta CEO S. Dindo}: Okay, cercate però dal prototipo di realizzare al meglio le funzionalità.
				\end{itemize}
				\\
				\rowcolor{white}
				\caption{Tabella riassuntiva domande e risposte}
			\end{longtable}	
		\end{center}
	
	
%\footnote{\href{https://developer.amazon.com/alexa}{https://developer.amazon.com/alexa}}
