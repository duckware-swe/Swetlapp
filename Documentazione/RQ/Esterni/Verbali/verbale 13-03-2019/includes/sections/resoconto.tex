\clearpage
\section{Resoconto}
	\subsection{Presentazione proseguimento dei lavori}
	\label{sec:presentazione_proseguimento_lavori}
	Il gruppo \emph{Duckware} ha presentato a \emph{CEO Stefano Dindo di Zero12} l'andamento dei lavori svolti alla data attuale. Il team ha spiegato ciò che è riuscito a realizzare mostrando il prototipo realizzato per il Proof of Concept avvenuto il 04-03-2019 con il Prof. Riccardo \textsc{Cardin}. Sono state fatte inoltre delle considerazioni sull'obbiettivo del progetto, il committente suggerisce di analizzare i limiti che la tecnologia \textit{Amazon Alexa} possiede, ed andare a realizzare una soluzione di gestione della \textit{skill} quanto più fruibile e apprezzabile dall'utente. 
	\subsection{Presentazione domande tecnologiche}
	\label{sec:presentazione_domande_tecnologiche}
	In presenza del collaboratore e CSD, \textbf{Simone Maratea}, i partecipanti del gruppo presenti all'incontro hanno esposto delle domande di natura tecnica per migliorare l'apprendimento e l'implementazione delle tecnologie \textit{AWS} richieste dalla proponente.  Le domande fatte sono sui seguenti argomenti:
	\begin{itemize}
		\item \textbf{Cognito}
			\begin{itemize}
				\item Tecniche e modalità di autenticazione:\\ \textit{viene spiegato \textit{User pool/Identity Pool} e \textit{Amplify}.}
				\item Come inviare i dati ad Alexa.
			\end{itemize}
		\item \textbf{Skill}
			\begin{itemize}
			\item Conferma di come gestire gli output vocali multipli:\\ \textit{viene suggerito di incapsulare tutto in un unico output in quanto è una limitazione di Alexa non poter restituire più di un output vocale.}
			\item Variabili di sessione e persistenti:\\ \textit{viene suggerito di usarle per gestire un dialogo con più interazioni tra Alexa e utente.}
			\end{itemize}
		\item \textbf{Connettori}
			\begin{itemize}
			\item Utilizzare siti terzi per l'autenticazione:\\ \textit{viene suggerito di gestire la funzionalità da applicazione.}
			\item quali connettori implementare:\\ \textit{viene suggerito di realizzare connettori fruibili che possono essere realizzabili attraverso le tecnologie AWS. Qualora dovesse risultare difficoltoso realizzare tale/i connettori fare presente ciò per pattuire una soluzione accettabile.}
			\end{itemize}
	\end{itemize}