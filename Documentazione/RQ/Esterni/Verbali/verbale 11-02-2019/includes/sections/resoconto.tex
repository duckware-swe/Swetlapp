\clearpage
\section{Resoconto}
	\subsection{Elenco domande fatte al \emph{CEO Stefano Dindo di Zero12}}
	\label{sec:elenco}
	Nella tabella si riportano le domande preparate e le risposte ottenute dal \textbf{CEO S. Dindo} in risposta alla mail inviata.
		\begin{center}
			\renewcommand{\arraystretch}{1.5}
			\rowcolors{3}{tableLightYellow}{}
			\begin{longtable}{  p{2.5cm} p{11.7cm} }
				\rowcolor{tableHeadYellow}
				\textbf{Identificativo}&\textbf{Domanda e risposta}\\
				D1 & Come scelta di implementazione è stato deciso di utilizzare come database \textit{Amazon DynamoDB}, un db non relazionale. Si chiede una considerazione da parte vostra sulla nostra scelta di utilizzare tale tecnologia.
				\begin{itemize}
					\item \textbf{Risposta CEO S. Dindo}: Si per le architetture \textit{serverless DynamoDB} è un buon database \textit{NoSQL} quindi avete fatto la scelta giusta. 
				\end{itemize}
				\\
				D2 & Per la realizzazione della skill per Alexa stiamo utilizzando il sito \href{https://developer.amazon.com/alexa}{https://developer.amazon.com/alexa}, dove creiamo la \textit{skill} vera e propria, inserendo gli \textit{intent} per i comandi e per abbozzare un discorso con Alexa;
				ed il sito \href{https://amzn.to/2UbRK0D}{https://amzn.to/2UbRK0D} per la creazione delle lambda che verrano poi eseguite dalla \textit{skill}.
				Si chiede se tale approccio per la realizzazione della \textit{skill} sia quello corretto, e/o sapere se presenti altre modalità più corrette per la creazione di tale \textit{skill}.
				\begin{itemize}
					\item \textbf{Risposta CEO S. Dindo}: Corretto.
				\end{itemize}
				\\
				D3 & Durante l'esecuzione di una lambda, è possibile, e se si come, collegare e far comunicare tra loro altre lambda? 
				\begin{itemize}
					\item \textbf{Risposta CEO S. Dindo}: Si, tratta eventi quindi potete fare in modo che una lambda chiami delle altre API così questa avvia delle lambda oppure usare \href{https://amzn.to/2tFT7co}{https://amzn.to/2tFT7co} se dovete fare in modo di creare una catena di lambda dove la prima passa il risultato alla seconda e così via. Terza strada sarebbe quella di integrare l'sdk sulla lambda e chiamare tramite un hook un'altra lambda... qui però bisogna stare attenti perché se la prima lambda deve attendere il tempo della seconda potenzialmente potrebbero aumentare i costi di esecuzione delle lambda perché pagate anche il tempo in cui la prima \textit{lambda} attende l'esecuzione della prima.
				\end{itemize}
				\\
				D4 & L'approccio attualmente utilizzato per effettuare chiamate al database dall'applicazione prevede di effettuare una richiesta \textit{HTTPS GET} che avvierà una lambda che si occuperà della comunicazione con \textit{DynamoDB}. Come è possibile risolvere le stesse richieste utilizzando invece una chiamata di tipo \textit{POST}?
				\begin{itemize}
					\item \textbf{Risposta CEO S. Dindo}: Dovete usare API gateway dove mappate le API rest e quindi potete definire delle post che poi chiamano una lambda. 
				\end{itemize}
				\\
				\rowcolor{white}
				\caption{Tabella riassuntiva domande e risposte}
			\end{longtable}	
		\end{center}
	
	
%\footnote{\href{https://developer.amazon.com/alexa}{https://developer.amazon.com/alexa}}
