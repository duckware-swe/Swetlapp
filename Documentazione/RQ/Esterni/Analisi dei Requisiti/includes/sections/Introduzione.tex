\clearpage
\section{Introduzione}
\label{sec:intro}
\subsection{Scopo del documento}
Lo scopo del presente documento è di fornire una descrizione completa e dettagliata di tutti i requisiti che sono stati individuati e dei casi d'uso riguardanti il progetto MegAlexa.
Tali informazioni sono state recuperate ed elaborate dal capitolato e dagli incontri, con le relative discussioni, con la proponente \emph{Zero12}.
\subsection{Scopo del prodotto}
L'obiettivo del prodotto è la realizzazione di un'applicazione per smartphone, nello specifico per la piattaforma Android OS, che permetta la creazione di workflow per l'assistente vocale Amazon Alexa. Il back-end sarà realizzato in Java opportunamente integrato con le API di Amazon Web Services, per il front-end verrà usato XML per stabilire i layout e Java per gestirne il comportamento. Si parlerà del front-end dell'assistente vocale riferendosi a VUI (voice user interface).
\subsection{Glossario}
Nel documento sono presenti termini che possono assumere significati ambigui a seconda del contesto o termini non conosciuti. Per ovviare a questa problematica è stato creato un Glossario contente tali termini con il loro significato specifico. Un termine è presente all'interno del \emph{Glossario v2.0.0} se seguito da una G corsiva in pedice.
\subsection{Riferimenti}
\subsubsection{Riferimenti normativi}
\begin{itemize}
	\item Norme di progetto v2.0.0;
	\item Capitolato d’appalto C4 dell'anno accademico 2018/2019\footnote{\href{https://www.math.unipd.it/~tullio/IS-1/2018/Progetto/C4.pdf}{https://www.math.unipd.it/~tullio/IS-1/2018/Progetto/C4.pdf}}.
\end{itemize}
\subsubsection{Riferimenti informativi}
\begin{itemize}
	\item Presentazione capitolato d’appalto C4 dell'anno accademico 2018/2019\footnote{\href{https://www.math.unipd.it/~tullio/IS-1/2018/Progetto/C4p.pdf}{https://www.math.unipd.it/~tullio/IS-1/2018/Progetto/C4p.pdf}};
	\item Lucidi didattici utilizzati durante il corso di Ingegneria del Software\footnote{\href{https://www.math.unipd.it/~tullio/IS-1/2018/}{https://www.math.unipd.it/~tullio/IS-1/2018/}}.
	\begin{itemize}
		\item Diagrammi dei casi d’uso;
		\item Analisi dei requisiti.
	\end{itemize}
\end{itemize}
