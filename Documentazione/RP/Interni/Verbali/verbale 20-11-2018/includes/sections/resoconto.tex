\clearpage
\section{Resoconto}
\subsection{Discussione dei capitolati}
	Prima della riunione tutti i membri del gruppo hanno compilato una tabella elettronica, creata e condivisa dal membro del gruppo \emph{Matteo Pellanda}, indicando i capitolati che hanno suscitato interesse durante l'esposizione in aula il 16/11/2018. \\
	Durante la  riunione, visionando la tabella, sono stati analizzati tutti  i  capitolati proposti escludendo alcuni di essi motivandone le ragioni al fine di ridurre il campo di scelte. \\
	Alla termine della discussione è stata fatta una votazione unanime e la scelta é ricaduta sul \emph{capitolato C4} con il progetto \emph{Meg\markg{Alexa}} proposto dall’azienda \emph{Zero12}. \\ 
	Il gruppo ha ritenuto interessante e stimolante l’utilizzo della tecnologia \markg{AWS} di \markg{Amazon} sul suo ultimo prodotto lanciato in commercio, \markg{Amazon} \markg{Alexa}.
	
	\subsection{Nome e logo gruppo}
	Durante l'incontro sono state ascoltate alcune idee sul nome e logo del gruppo, per ora identificato come \emph{gruppo 3}.
		\begin{center}
			\renewcommand{\arraystretch}{1.5}
			\rowcolors{3}{tableLightYellow}{}
			\begin{longtable}{  p{3cm} p{11.2cm}  }
				\rowcolor{tableHeadYellow}
				\textbf{Identificativo} & \textbf{Descrizione}\\
				NL1 & 	 	\textbf{Nome e logo del gruppo:}
				\begin{itemize}
					\item con scarsi risultati è stato deciso di rimandare la scelta di questa decisione alla riunione successiva.
				\end{itemize}
				\\					
				\rowcolor{white}
				\caption{Tabella riassuntiva tracciabilità}
			\end{longtable}	
		\end{center}
	\clearpage
	\subsection{Organizzazione}
	Per facilitare e coordinare al meglio il lavoro è stato scelto di adottare i seguenti strumenti organizzativi:
	
	\begin{center}
		\renewcommand{\arraystretch}{1.5}
		\rowcolors{3}{tableLightYellow}{}
		\begin{longtable}{  p{3cm} p{11.2cm}  }
			\rowcolor{tableHeadYellow}
			\textbf{Identificativo} & \textbf{Descrizione}\\
			ORG1 & 	 	\textbf{\markg{Google Drive}:}
			\begin{itemize}
				 \item sono stati utilizzati i file e le cartelle condivise dalla piattaforma \markg{Google Drive} personale del componenete \emph{Matteo Pellanda} accessibili a tutti i membri del gruppo.
			\end{itemize}
			\\			
			
			ORG2 & 	 	\textbf{\markg{Slack}:}
			\begin{itemize}
				\item piattaforma scelta per la comunicazione, essa permette la creazione di diversi canali e l’integrazione con \markg{Google Drive} e \markg{GitLab}.
			\end{itemize}
			\\			
			
			ORG3 & 	 	\textbf{\markg{GitLab}:}
			\begin{itemize}
				\item scelto come strumento per il versionamento.
			\end{itemize}
			\\			
					
			\rowcolor{white}
			\caption{Tabella riassuntiva tracciabilità}
		\end{longtable}	
	\end{center}