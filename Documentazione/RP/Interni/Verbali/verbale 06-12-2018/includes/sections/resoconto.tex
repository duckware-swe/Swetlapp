\clearpage
\section{Resoconto}
\subsection{Esaminazione Studio di Fattibilità}
	Il documento in questione viene letto da tutti i partecipanti del team duckware, vengono fatti notare alcuni punti di stesura, verificato e approvato.
	 
\subsection{Email a Zero12}
	Dopo un confronto collettivo il team ha deciso di inviare una mail a Zero12 S.p.A.\\
	Nella mail viene fatta una presentazione del gruppo, esposti alcuni dubbi e richiesto un incontro con il proponente del capitolato.\\
	Di seguito viene lasciato il contenuto della mail:
	\begin{quote}
	\emph{
	 	Alla cortese attenzione di Stefano Dindo.\\[0.25cm]
		Salve, con la presente mail il gruppo 3, del corso di Ingegneria del Software - Informatica dell'Università degli Studi di Padova, vuole presentarsi e proporsi per la realizzazione del \markg{capitolato d'appalto} proposto in sede universitaria il 16/11/2018.\\[0.25cm]
		Il gruppo di progetto, chiamato duckware, è composto da 7 studenti: Sonia, Matteo, Luca, Alberto, Andrea, Pardeep e Alessandro.\\[0.25cm]
		In merito ad un primo studio di fattibilità duckware desidererebbe chiarire alcuni punti:
		\begin{itemize}
		\item scelta sull'uso di tecnologie per la realizzazione dell'app smartphone e web, nel particolare se possibile realizzare le app \markg{Android} e iOS con un \markg{framework} crossplatform nativo. Esso permetterebbe la realizzazione di un'applicazione per entrambe le piattaforme
		\item come, quando necessario, testare il \markg{software}
		\end{itemize}
		Si chiede inoltre la disponibilità, se possibile, di un incontro, anche nella vostra sede, per poter discutere faccia faccia ed entrare nel dettaglio sulle scelte da adottare.\\[0.25cm]
		In attesa di una vostra risposta\\
		Vi ringraziamo \\
		Team Duckware\\
	}
	\end{quote}

\subsection{Scadenze e suddivisione documenti}
	Viene fissato un obiettivo comune e unanime da parte del team nel voler presentare tutti la documentazione del RR con scadenza di consegna il 14-01-2019.\\
	La suddivisione dei documenti viene assegnata nel seguente modo:
	\\[1.5cm]
		\begin{center}
		\textbf{Analisi dei Requisiti}\\[0.25cm]
		\renewcommand{\arraystretch}{1.5}
		\begin{longtable}{  p{5cm} p{4cm}  }
			
			\rowcolor{tableHeadYellow}
			\textbf{Punto del capitolo}&\textbf{AT(@)}\\
				\emph{1 - Introduzione} & \multirow{2}{*}{Luca Stocco}\\ \emph{2 - Descrizione generale}  \\
				\hline
				\multirow{4}{*}{\emph{3 - Casi d'Uso}} & Alessandro Pegoraro\\&Pardeep Singh\\&Andrea Pavin\\&Matteo Pellanda\\
				\hline
				\multirow{2}{*}{\emph{4 - Requisisti}} &  Alberto Miola\\&Sonia Menon\\
				\hline
				\rowcolor{white}
			\caption{Tabella riassuntiva suddivisione lavoro}
		\end{longtable}	
		\end{center}
	
		\begin{center}
			\textbf{Piano di Qualifica}\\[0.25cm]
			\renewcommand{\arraystretch}{1.5}
			\begin{longtable}{  p{6cm} p{4cm}  }
				\rowcolor{tableHeadYellow}
				\textbf{Punto del capitolo}&\textbf{AT(@)}\\
					\emph{1 - Introduzione} & Luca Stocco  \\ \emph{2 - Qualità di \markg{processo}} & Matteo Pellanda\\
				\hline
				\emph{4 - Specifiche dei test} & Andrea Pavin  \\ \emph{5 - Misure e metriche per i test} & Alberto Miola\\
				\hline
				\multirow{2}{*}{\emph{3 - Qualità di prodotto}} & Pardeep Singh\\&Alessandro Pegoraro\\
				\hline
				\emph{6 - Resoconto + Standard \markg{ISO}} & Sonia Menon\\
				\hline
				\rowcolor{white}
				\caption{Tabella riassuntiva suddivisione lavoro}
			\end{longtable}	
		\end{center}
		\clearpage
		\begin{center}
		\textbf{Piano di Progetto}\\[0.25cm]
		\renewcommand{\arraystretch}{1.5}
		\begin{longtable}{  p{8cm} p{4cm}  }
			\rowcolor{tableHeadYellow}
			\textbf{Punto del capitolo}&\textbf{AT(@)}\\
			\emph{1 - Introduzione} & Andrea Pavin  \\ \emph{2 - Analisi dei rischi} & Pardeep Singh\\ \emph{A - Organigramma} & Alessandro Pegoraro\\
			\hline
			\emph{4 - Modello di sviluppo} & Luca Stocco  \\ \emph{5 - Pianificazione} & Matteo Pellanda\\
			\hline
			\emph{5 - Suddivisione risorse} e preventivo & Alberto Miola\\ \emph{6 - Consuntivo di periodo e preventivo a finire} & Sonia Menon\\
			\hline
			\rowcolor{white}
			\caption{Tabella riassuntiva suddivisione lavoro}
		\end{longtable}	
		\end{center}
	
	
\subsection{Discussione linguaggi e tecnologie}
	Al team vengono proposti i seguenti linguaggi e tecnologie per la realizzazione delle piattaforme richieste dal committente.
	
		\begin{center}
		\renewcommand{\arraystretch}{1.5}
		\rowcolors{3}{tableLightYellow}{}
		\begin{longtable}{  p{3cm} p{6cm}  }
			\rowcolor{tableHeadYellow}
			\textbf{Identificativo} & \textbf{Descrizione}\\
			
			TEC1 & 	 	\textbf{App per iOS e \markg{Android}:}
			\begin{itemize}
				\item Delphi
				\item Delphi® - Overview \markg{IDE} 
				\item Firemonkey \markg{framework} 
			\end{itemize}
			\\			
			
			TEC2 & 	 	\textbf{Piattaforma Web:}
			\begin{itemize}
				\item HTML5 
				\item CSS3
				\item PHP
				\item Javasript
			\end{itemize}
			\\			
			
			\rowcolor{white}
			\caption{Tabella riassuntiva proposte tecnologiche}
		\end{longtable}	
		\end{center}

