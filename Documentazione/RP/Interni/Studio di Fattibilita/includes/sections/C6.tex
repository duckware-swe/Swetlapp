\clearpage
\section{Capitolato C6 - Soldino}
\label{sec:c6}
\subsection{Descrizione}
Il capitolato C6, Soldino, richiede di creare un sistema di gestione dell'imposta sul valore aggiunto (IVA), su piattaforma decentralizzata Ethereum. Gli utenti sono composti da:
\begin{itemize}
\item Produttori 
\item Grossisti 
\item Dettaglianti 
\item Consumatori finali.
\end{itemize}
Le transazioni tra questi saranno eseguite attraverso Ethereum e tramite lo stesso verrà versato allo stato l'incasso netto dell'IVA accumulato nelle transazioni tra le prime tre tipologie di utenti elencati.

\subsection{Dominio Applicativo}
Creare un sistema di gestione dell'IVA che attraverso un sito web permetta di automatizzare transazioni e versamenti appoggiandosi alla blockchain Ethereum.

\subsection{Dominio Tecnologico}
\begin{description}
\item[Smart contract:] Procedure che facilitano, verificano, o fanno rispettare la negoziazione o l'esecuzione di un contratto.
Sono definiti dal loro creatore, ma la loro esecuzione e, di conseguenza, il servizio che offrono, sono dettati dal network Ethereum.
Una volta messi in atto esistono finchè esiste l'intero network, scompaiono solo se programmati per autodistruggersi.\\
Link: \href{https://www.investopedia.com/terms/s/smart-contracts.asp}{https://www.investopedia.com/terms/s/smart-contracts.asp}

\item[Solidity:] Linguaggio contract-oriented che consente l'implementazione di smart contracts su Ethereum Virtual Machine (EVM).\\
Link: \href{https://solidity.readthedocs.io/en/v0.4.25/}{https://solidity.readthedocs.io/en/v0.4.25/}

\item[Ropsten:] Rete di test utilizzante gli stessi protocolli di Ethereum utile a testare dApp(Ethereum enabled distributed applications).\\
Link: \href{https://ethereum.stackexchange.com/questions/13534/what-is-actually-ropsten-what-is-a-new-network}{https://ethereum.stackexchange.com/questions/13534/what-is-actually-ropsten-what-is-a-new-network}

\item[React:] Libreria JavaScript per la creazione di interfacce utente riutilizzabili.\\
Link: \href{https://reactjs.org/}{https://reactjs.org/}

\item[Surge:] Servizio che permette la messa in uso di web-app.\\
Link: \href{https://surge.sh/}{https://surge.sh/}

\item[MetaMask:] Estensione per browser che permette di allacciarsi a servizi presenti su blockchain Ethereum (Ethereum enabled distributed applications).\\
Link: \href{https://metamask.zendesk.com/hc/en-us}{https://metamask.zendesk.com/hc/en-us}

\item[Raiden:] Blockchain complementare a Ethereum che permette pagamenti istantanei.\\
Link: \href{https://raiden.network/}{https://raiden.network/}
\end{description}

\subsection{Valutazione Finale}
Il capitolato propone tecnologie moderne e pone un obiettivo sfumato ma avveniristico. Lascia un ampio margine di libertà nell'implementazione: le direttive ci sono sembrate sfocate ed è difficile immaginare la forma del prodotto finito. La presentazione è stata vaga sulla richiesta del committente e non è stato ancora capito a fondo l'obiettivo richiesto. Il team di Duckware ha quindi espresso poca preferenza verso questo capitolato.
