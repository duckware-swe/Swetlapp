\clearpage
\section{Capitolato C4 - MegAlexa}
\label{sec:c4}
\subsection{Descrizione}
Il capitolato C4 mira a creare un'applicazione per dispositivi mobile o un'interfaccia web che sia in grado di creare delle routine personalizzabili dagli utenti.
Queste ultime dovranno essere gestibili utilizzando \markg{Alexa}, l'assistente personale intelligente creato da \markg{Amazon}. Ogni utente potrà quindi creare una propria sequenza di azioni preferite scegliendo fra quelle già disponibili ed interagendo con \markg{Alexa}.

\subsection{Dominio Applicativo}
Creare una skill di \markg{Alexa} che, tramite sito web o applicazione mobile, possa aiutare l'utente a creare un \markg{workflow} nel quale vengono specificate una serie di azioni da compiere. 
Sarà disponibile una serie di \markg{connettori}, ovvero delle azioni predefinite che l'utente sarà in grado di inserire nel suo \markg{workflow}.

\subsection{Dominio Tecnologico}
\begin{description}
\item[\markg{Java}:] Linguaggio di programmazione ad alto livello, orientato agli oggetti e a tipizzazione statica, specificamente progettato per essere il più possibile indipendente dalla piattaforma di esecuzione.\\
Link: \href{https://docs.oracle.com/javase/8/}{https://docs.oracle.com/javase/8/}
	
	\item[Bootstrap:] Framework per HTML, CSS, Javascript per creare siti web responsivi e mobile-first\\
Link: \href{https://www.javascript.com/}{https://www.javascript.com/}

	\item[PHP:]	Linguaggio di scripting che consente la creazione di pagine web dinamiche ed interazioni con i database
	
	\item[\markg{Amazon Web Services}:]	Servizio di computazione su cloud fornito da \markg{Amazon} che propone diverse soluzioni fra le quali: \markg{AWS} Lambda per l'esecuzione in cloud di servizi, \markg{AWS} API Gateway per la creazione di \markg{API} (potenzialmente per infrastrutture REST) e \markg{AWS} DynamoDB per il supporto ai database non relazionali\\
Link: \href{https://aws.amazon.com/it/lambda/}{https://aws.amazon.com/it/lambda/}\\
Link: \href{https://aws.amazon.com/it/api-gateway/}{https://aws.amazon.com/it/api-gateway/}\\
Link: \href{https://aws.amazon.com/it/dynamodb/}{https://aws.amazon.com/it/dynamodb/}
\end{description}

\subsection{Valutazione Finale}
Questo capitolato è sembrato molto stimolante al gruppo in quanto propone una vasta gamma di tecnologie da utilizzare e consente di approcciarsi ai dispositivi personali intelligenti più recenti. Si è ritenuto fosse particolarmente interessante poter gestire da vicino il dispositivo utilizzando prima un'app sul proprio telefono e poi inviando comandi vocali.
Un fattore importante nella scelta di questo capitolato è stato l'uso dei servizi di \markg{Amazon} poichè \markg{AWS} è uno dei leader mondiali del \markg{cloud computing} e si sta affermando sempre di più sul mercato. 
