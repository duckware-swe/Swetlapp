\clearpage
\section{Capitolato C2 - Mivoq}
\label{sec:c2}
\subsection{Descrizione}
Lo scopo del capitolato C2, è quello di sviluppare una piattaforma web di tipo collaborativo di raccolta dati, in cui gli utenti possano predisporre e/o svolgere esercizi di grammatica. Grazie a questa piattaforma i dati raccolti verrano utilizzati da sviluppatori al fine di insegnare ad un elaboratore a svolgere esercizi mediante tecniche di apprendimento automatico.

\subsection{Dominio Applicativo}
L'obiettivo finale del capitolato è quello di realizzare una piattaforma che raccolga dati da tre diverse tipologie di utenti: insegnanti, allievi, sviluppatori.
Per gli insegnanti la piattaforma deve poter:
\begin{itemize}
		\item Predisporre di esercizi di analisi grammaticale
		\item Inserire nuove frasi nel sistema della piattaforma e poterle etichettare in modo automatico con informazioni relative all'analisi
		\item Correggere gli svolgimenti
\end{itemize}
Al fine di agevolare il lavoro di preparazione viene consigliato l'utilizzo di \markg{software} di terze parti. Poiché tali \markg{software} sono soggetti ad errori l'insegnante dovrà poter correggere i risultati prodotti da esso.
Per gli allievi la piattaforma deve poter:
\begin{itemize}
		\item Svolgere gli esercizi proposti e/o predisposti e ricevere una valutazione immediata
\end{itemize}

Per gli sviluppatori la piattaforma deve poter:
\begin{itemize}
		\item Prelevare ed accedere ai dati che sono raccolti nella piattaforma e poter visionare più versioni di analisi di ogni frase svolta al fine di dedurre quale sia l'annotazione più corretta
		\item Conoscere lo storico dei dati
		\item Accedere ai modelli realizzati
\end{itemize}

\subsection{Dominio Tecnologico}
\begin{description}
\item[MaryTTS e FreeLing:] Sono piattaforme di sintesi vocale multilingua e open-source. Questi \markg{software} implementano il part-of-speech tagger, cioè etichettare le parti del discorso con tag riferiti alle classi grammaticali.

\item[Firebase:] è una piattaforma di sviluppo di applicazioni web e mobili sviluppata da Google nel 2014. Offre funzionalità che ruotano attorno ai servizi cloud, consentendo agli utenti di salvare e recuperare i dati ai quali potranno accedere da qualsiasi dispositivo o browser. Nel caso in questione il suo utilizzo è quello di analizzare e immagazzinare al meglio i dati estratti dalla piattaforma richiesta.
\end{description}

Per l'obiettivo richiesto il gruppo 3 preventiva l'utilizzo di tecnologie web come:
\begin{description}		
		\item[Html e CSS:] Sono linguaggi che hanno applicazione in pagine/siti web, nati per la formattazione e impaginazione di documenti ipertestuali. Tramite questi linguaggi sarà possibile realizzare un'interfaccia, la più smart e friendly possibile, fra l'utente e la piattaforma.

		\item[Javascript:] E' un linguaggio di scripting orientato agli oggetti e agli eventi, comunemente utilizzato nella programmazione lato \markg{client} per la creazione di applicazioni e piattaforme web. Il suo ambito in questo capitolato sarà quello di interfacciarsi con le piattaforme comunicateci (MaryTTS o FreeLing e Firebase) per poter implementare le features richieste.
\end{description}

\subsection{Valutazione Finale}
Il gruppo Duckware ha espresso pareri positivi riguardo la richiesta di utilizzo della piattaforma Firebase, tuttavia il capitolato presentato non ha richiamato l'attenzione di tutti i partecipanti del nostro gruppo, i quali non lo hanno trovato appetibile rispetto alle tendenze tecnologiche odierne e poco stimolante nell'auto-apprendimento.
La presentazione è stata interessante anche se a primo impatto risultava vaga sulla richiesta del committente, solo dopo un'ulteriore analisi infatti si è inquadrato a pieno l'obiettivo richiesto.
Infine dopo aver analizzato tutti i capitolati disponibili e le specifiche per essi richieste, il team di Duckware ha deciso di non considerare questo progetto in quanto non presenta tecnologie di interesse, che siano sufficientemente innovative e forniscano nuove capacità e conoscenze.
