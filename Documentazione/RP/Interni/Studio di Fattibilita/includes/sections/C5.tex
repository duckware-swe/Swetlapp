\clearpage
\section{Capitolato C5 - GaiaGo}
\label{sec:c5}
\subsection{Descrizione}
Il capitolato C5 ha lo scopo di creare un'applicazione di Car Sharing per dispositivi \markg{Android}. L'app deve permetta agli utenti di definire un calendario per segnare i giorni in cui è disposto a prestare il proprio veicolo. Quest'ultimo verrà munito del dispositivo GaiaBox con cui il proprietario potrà geolocalizzarlo e controllare i percorsi effettuati dalla macchina. Ogni singolo utente potrà quindi chiedere in prestito tutte le macchine registrate nelle sue vicinanze e solo nei giorni concordati.

\subsection{Dominio Applicativo}
Creare un supporto applicativo che possa permettere all'utente di creare un calendario nel quale vengono specificati i giorni in cui pone a disposizione il proprio veicolo.
Creare un servizio specifico Peer To Peer che organizzi il calendario e calcoli l'effettiva disponibilità di ogni macchina per ogni utente.
Implementare nell'applicativo almeno 5 dei core drive del \markg{framework} di Octalysis per incentivare gli utenti ad usare Gaiago.

\subsection{Dominio Tecnologico}
Non vengono imposti vincoli nella scelta delle tecnologie per implementare il servizio Peer To Peer e la creazione dell'applicazione \markg{Android}, mentre per la Gamification viene richiesto l'utilizzo del \markg{framework} Octalysis.
\begin{description}
	\item[Javascript:] Linguaggio di scripting orientato agli oggetti e agli eventi, comunemente utilizzato nella programmazione web lato \markg{client} per la creazione di applicazioni.
Il suo ambito in questo capitolato sarà quello di interfacciarsi tra l'applicazione e il \markg{server} contenente il calendario.\\
Link: \href{https://www.javascript.com/}{https://www.javascript.com/}

	\item[PHP:]	Linguaggio di scripting utilizzato per le interazioni con il database.\\
Link: \href{http://www.php.net/}{http://www.php.net/}
						
	\item[Framework Octalysis:]	Servizio di Gamification ideato da Yu-kai Chou; la Gamification viene intesa come spinta motivazionale e agonistica che mantiene l'utente fedele ad un dato servizio. Il \markg{framework} si divide in otto core principali e ci viene chiesto di sviluppare il progetto implementandone almeno 5.\\ 
Link: \href{https://yukaichou.com/gamification-examples/octalysis-complete-gamification-framework/}{https://yukaichou.com/gamification-examples/octalysis-complete-gamification-framework/}\\
Link: \href{https://octalysisgroup.com/}{https://octalysisgroup.com/}\\
Link: \href{https://www.yukaichou.com/octalysis-tool/}{https://www.yukaichou.com/octalysis-tool/}
\end{description}

\subsection{Valutazione Finale}
Questo capitolato inizialmente si è presentato come il più appetibile per il gruppo date le conoscenze personali già affermate per la maggior parte dei componenti di Duckware, ma un'analisi più attenta ha messo in luce una limitata possibilità di aumentare il proprio bagaglio conoscitivo e di competenze rispetto ad altri progetti.
Il capitolato non è più stato la scelta predominante del gruppo ma ha guadagnato il secondo posto come progetto da poter sviluppare.
	
