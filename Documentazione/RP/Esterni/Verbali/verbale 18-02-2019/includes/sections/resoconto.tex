\clearpage
\section{Resoconto}
	\subsection{Elenco domande fatte al \emph{CEO Stefano Dindo di Zero12}}
	Nella tabella si riportano le domande preparate e le risposte ottenute dal \textbf{CEO S. Dindo} in risposta alla mail inviata.
		\begin{center}
			\renewcommand{\arraystretch}{1.5}
			\rowcolors{3}{tableLightYellow}{}
			\begin{longtable}{  p{2.5cm} p{11.7cm} }
				\rowcolor{tableHeadYellow}
				\textbf{Identificativo}&\textbf{Domanda e risposta}\\
				D1 & Come scelta di sviluppo è stato pensato di utilizzare, per l'applicazione Android, \textit{Andorid Studio}, un IDE pensato principalmente per sviluppare applicazioni per sistemi operativi Andorid OS. Si chiede una considerazione da parte vostra sulla nostra scelta di utilizzare tale tecnologia.
				\begin{itemize}
					\item \textbf{Risposta CEO S. Dindo}: La scelta del IDE è personale, Android Studio è un valido IDE per sviluppare l'app.
				\end{itemize}
				\\
				D2 & Per la realizzazione dell'applicazione Android si chiede se ci sono particolari richieste nel frontend, come animazioni, scelte di colori o altro nella GUI del prodotto.
				\begin{itemize}
					\item \textbf{Risposta CEO S. Dindo}: No, non ci sono particolari richieste, le scelte del frontend sono libere. Fate particolare attenzione al gradimento visivo ed intuitivo che andate a realizzare.
				\end{itemize}
				\\
				D3 & Il nome dell'applicazione e il nome della skill associata che stiamo realizzando deve essere lo stesso o possono avere normi diversi?
				\begin{itemize}
					\item \textbf{Risposta CEO S. Dindo}: È irrilevante, il nome dei due prodotti possono avere tranquillamente nomi diversi. 
				\end{itemize}
				\\
				D4 & L'applicazione Android che andremo a realizzare avrà come limitazione il fatto di essere installata solo su determinare versioni del OS. Si chiede una considerazione da parte vostra su tale considerazione.
				\begin{itemize} 
					\item \textbf{Risposta CEO S. Dindo}: La vostra applicazione deve poter girare sulle versioni 4.4.x e successive di Android.
				\end{itemize}
				\\
				\rowcolor{white}
				\caption{Tabella riassuntiva domande e risposte}
			\end{longtable}	
		\end{center}
	
	
%\footnote{\href{https://developer.amazon.com/alexa}{https://developer.amazon.com/alexa}}
