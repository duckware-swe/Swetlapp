\clearpage
\section{Introduzione}
\label{sec:intro}
\subsection{Scopo del documento}
Lo scopo di questo documento è quello di delineare la pianificazione del gruppo \emph{duckware} per lo sviluppo del progetto \emph{Meg\markg{Alexa}}. Il documento riporterà un'analisi dei costi e dei rischi collegati allo sviluppo di tale progetto. Nel dettaglio il documento tratterà:
\begin{itemize}
	\item Analisi del modello di sviluppo per il progetto;
	\item Analisi dei rischi relativi al progetto;
	\item Una pianificazione dettagliata dei tempi e delle attività;
	\item Una stima preventiva dell'utilizzo delle risorse disponibili.
\end{itemize}
\subsection{Scopo del prodotto}
L'obiettivo del prodotto è la realizzazione di un'applicazione per smartphone, nello specifico per la piattaforma \markg{Android} OS, che permetta la creazione di \markg{workflow} per l'assistente vocale \markg{Amazon} \markg{Alexa}. Il \markg{back-end} sarà realizzato in \markg{Java} opportunamente integrato con le \markg{API} di \markg{Amazon Web Services}, per il \markg{front-end} verrà utilizzato \markg{XML} per stabilire i layout e \markg{Java} per gestirne il comportamento. Si parlerà del \markg{front-end} dell'assistente vocale riferendosi a \markg{VUI}(voice user interface).
\subsection{Glossario}
Nel documento sono presenti termini che possono assumere significati ambigui a seconda del contesto o termini non conosciuti. Per ovviare a questa problematica è stato creato un Glossario contente tali termini con il loro significato specifico. Un termine è presente all'interno del Glossario v2.0.0 se seguito da una G corsiva a pedice.
\subsection{Riferimenti}
\subsubsection{Riferimenti normativi}
\begin{itemize}
	\item Norme di Progetto v2.0.0;
	\item Capitolato d'appalto C4 - MegAlexa\footnote{\href{https://www.math.unipd.it/~tullio/IS-1/2018/Progetto/C4.pdf}{https://www.math.unipd.it/~tullio/IS-1/2018/Progetto/C4.pdf}};
	\item Regole organigramma e specifica tecnica-economica\footnote{\href{https://www.math.unipd.it/~tullio/IS-1/2018/Progetto/RO.html}{https://www.math.unipd.it/~tullio/IS-1/2018/Progetto/RO.html}}.
\end{itemize}
\subsubsection{Riferimenti informativi}
\begin{itemize}
	\item Gestione di progetto - Slide del corso di Ingegneria del Software\footnote{\href{https://www.math.unipd.it/~tullio/IS-1/2018/Dispense/L06.pdf}{https://www.math.unipd.it/~tullio/IS-1/2018/Dispense/L06.pdf}};
	\item Regole del Progetto didattico - Slide del corso di Ingegneria del Software\footnote{\href{https://www.math.unipd.it/~tullio/IS-1/2018/Dispense/P01.pdf}{https://www.math.unipd.it/~tullio/IS-1/2018/Dispense/P01.pdf}}.
\end{itemize}
\clearpage
\hypertarget{scadenze}{}
\subsection{Scadenze}
Il gruppo \emph{duckware} decide di rispettare le seguenti scadenze temporali, su cui si basa per la pianificazione e per lo svolgimento del progetto:
\begin{itemize}
	\item \textbf{Revisione dei Requisiti}: 21-01-2019;
	\item \textbf{Revisione di Progettazione}: 15-03-2019;
	\item \textbf{Revisione di Qualifica}: 19-04-2019;
	\item \textbf{Revisione di Accettazione}: 17-05-2019.
\end{itemize}
