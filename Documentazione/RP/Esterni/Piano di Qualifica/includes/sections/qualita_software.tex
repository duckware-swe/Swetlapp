\textbf{Software}\\
Di seguito viene riportata la tabella riassuntiva delle metriche e degli obiettivi riconosciuti, con il range di accettazione e di ottimalità, per le misure effettuate nel \markg{software}.
\label{sec:qualita_software}
\begin{center}
	\centering
	\renewcommand{\arraystretch}{1.5}
	\rowcolors{3}{tableLightYellow}{}
	\begin{longtable}{ >{\RaggedRight}p{2.8cm} >{\RaggedRight}p{5cm}  >{\RaggedRight}p{3cm} >{\RaggedRight}p{2.5cm}  }
		\rowcolor{tableHeadYellow}
		\textbf{Metrica}  & \textbf{Obiettivo} & \textbf{Range \mbox{accettazione}} & \textbf{Range \mbox{ottimale}} \\ 
		%\endhead
		\textbf{MPRDS001} & Copertura requisiti obbligatori & 100\% & 100\% \\
		\textbf{MPRDS002} & Copertura requisiti accettati & 60\% - 100\% & 80\% - 100\% \\
		\textbf{MPRDS003} & Percentuale di failure & 0\% - 5\% & 0\% \\
		\textbf{MPRDS004} & Blocco operazioni non corrette & 80\% - 100\% & 100\% \\
		\textbf{MPRDS005} & Comprensibilità delle funzioni offerte & 80\% -100\% & 85\% - 100\% \\
		\textbf{MPRDS006} & Facilità di apprendimento delle funzionalità & 0 - 30 min & 0-15 min \\
		\textbf{MPRDS007} & Tempo di risposta & 0 - 4 sec & 0 - 2 sec \\
		\textbf{MPRDS008} & Impatto delle modifiche & 0\% - 20\% & 0\% - 10\% \\
		\textbf{MPRDS009} & Complessità ciclomatica                & 0 - 30      &      0 - 30 \\
		\textbf{MPRDS010} & Numero di metodi                       & 2 - 10      &      3 - 8 \\
		\textbf{MPRDS011} & Variabili non utilizzate               & 0           &      0 \\
		\textbf{MPRDS012} & Numero di bug per linea                & 0 - 60      &      0 - 25 \\
		\textbf{MPRDS013} & Rapporto linee di codice e commento    & \textgreater { 0.20 }      & SV \textgreater { 0.30 } \\
		\rowcolor{white}
		\caption{Tabella delle metriche della qualità del software}
	\end{longtable}
\end{center}
