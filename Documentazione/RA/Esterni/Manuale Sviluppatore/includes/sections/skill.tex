\pagebreak
\section{Architettura Skill}

L'architettura della skill che risiede in cloud in AWS Lambda è rappresentata dal seguente diagramma:

\begin{figure}[H]
	\begin{center}
		\includegraphics[width=\textwidth, keepaspectratio]{../includes/pics/Skill.png}
		\caption{Overview architetturale della skill.}
	\end{center}
\end{figure}
%TODO descrizione e dire che la struttura generale è visibile nella figura, aggiungere quindi il diagramma della skill completa

\subsection{Back-end}
La logica della skill è gestita dal file \emph{index.js} al quale Alexa si appoggerà anche per l'invio delle richieste derivanti dall'interazione con l'utente.
Ogni richiesta inviata incapsula una serie di dati, quali il tipo di \markg{\gls{intent}} che la attiva e i parametri forniti dall'utente, denominati \emph{slot}, e sarà gestita da uno degli handler presenti nel file \emph{index.js}.
Per ognuno di questi verrà eseguito il metodo \textit{canHandle()} e, in caso di ritorno positivo, sarà lanciato il relativo metodo \textit{handle()}, interrompendo la ricerca dell'handler adeguato.
Nel caso in cui non venga trovato alcun gestore appropriato, la skill ha un comportamento non definito che porta al suo arresto improvviso; pertanto la \markg{\gls{best practice}} prevede l'implementazione di un handler di default che viene eseguito all'occorrenza di errori.

\subsubsection{Handler}
L'interfaccia \emph{RequestHandler} è fornita dalle librerie di AWS incluse nel package.
Espone un metodo \textit{canHandle(input : Input)} che ritornerà un booleano: \textit{true} nel caso in cui l'handler che lo ridefinisce dovrà gestire la richiesta passatagli; \textit{false} altrimenti.
Nel caso di ritorno positivo, verrà eseguito il metodo \textit{handle(input : Input)} fornito dall'interfaccia; la sua ridefinizione dovrà elaborare la richiesta in input per poter fornire un ritorno \textit{Output} comprensibile da Alexa.
Gli handler sono strutturati con uno strategy pattern descritto nel seguente diagramma:

\begin{figure}[H]
	\begin{center}
		\includegraphics[width=0.9\textwidth, keepaspectratio]{../includes/pics/Strategy-Pattern.png}
		\caption{Diagramma della classe Handler}
	\end{center}
\end{figure}


I principali handler da noi definiti sono:
\begin{itemize}
	\item \textit{LaunchRequestHandler}: gestisce il lancio iniziale della skill controllando che l'utente sia autenticato; in caso contrario lo invita ad effettuare il login con una notifica vocale e una push nell'applicazione Amazon Alexa dell'utente;
	\item \textit{RunWorkflowHandler}: viene lanciato quando l'utente richiede l'avvio di un workflow; si occupa di effettuare una richiesta al database per ottenere i dati del workflow richiesto, quando disponibile, e cominciare la sua esecuzione;
	\item \textit{StopIntentHandler}: elabora la richiesta di arresto della skill da parte dell'utente;
	\item \textit{ErrorHandler}: è il gestore di default che verrà eseguito quando nessuno dei precedenti handler è stato avviato, o nel caso in cui si siano verificati errori durante l'esecuzione della skill;
\end{itemize}

\subsubsection{Action}
I workflow sono composti da azioni;
la loro struttura è definita da \emph{Action}, questa è una classe astratta ed espone un metodo \textit{run()} che verrà chiamato per l'effettiva esecuzione dell'azione specifica.
La sua ridefinizione delinea il comportamento dell'azione concreta.\\
Le azioni ridefinite nella Skill SwetlApp sono:
\begin{itemize}
	\item \textit{CustomMessageAction}: consente ad Alexa di leggere un messaggio definito dall'utente;
	\item \textit{ReadFeedRSSAction}: consente ad Alexa di ricevere e leggere un feed rss impostato dall'utente;
	\item \textit{TVScheduleAction}: consente ad Alexa di informare l'utente sulla programmazione quotidiana dei suoi canali TV preferiti;
	\item \textit{TwitterReadAction}: permette ad Alexa di leggere i tweet di un account Twitter definito dall'utente;
	\item \textit{TwitterWriteAction}: permette ad Alexa di postare un tweet personalizzato nella bacheca dell'utente;
\end{itemize}

\subsubsection{ActionFactory}
Si occupa della costruzione di un oggetto concreto di tipo derivante da Action.
Nasconde all'esterno il modo in cui viene scelto quale oggetto costruire e permette di estendere, manutenere velocemente e semplicemente il codice, in quanto per aggiungere una nuova Action sarà sufficiente che questa implementi l'interfaccia, e che venga aggiunto un controllo sul factory perchè questa sia usabile all'esterno.
Il controllo per capire qual è il tipo di azione corretto è fatto a partire dal nome dell'azione che viene chiesto come parametro, su questo sarà svolto uno switch case che ritorna l'azione corretta.
Si tratta di un \markg{\gls{factory pattern}}.

\begin{figure}[H]
	\begin{center}
		\includegraphics[width=0.9\textwidth, keepaspectratio]{../includes/pics/Factory-Pattern.png}
		\caption{Diagramma della classe Action}
	\end{center}
\end{figure}

\subsubsection{Interazione con il database}
La connessione al database è gestita con un pattern singleton che quindi impone l'esistenza di un'unica istanza di \textit{DatabaseInteractor} in un dato momento. L'accesso di più client contemporaneamente al database è possibile perchè AWS si occupa poi di gestire le multiple richieste di accesso a DynamoDB.
Fornisce un metodo di supporto \textit{query(params : JSON)} per gestire tutte le interazioni con il database direttamente da DatabaseInteractor.


\begin{figure}[H]
	\begin{center}
		\includegraphics[width=0.75\textwidth, keepaspectratio]{../includes/pics/Singleton-Pattern.png}
		\caption{Diagramma della classe DatabaseInteractor}
	\end{center}
\end{figure}
