\pagebreak
\section{Installazione}
\label{sec:installazione}
\subsection{Requisiti software}
\begin{itemize}
    \item Java Development Kit (jdk) 1.8.151 o superiore;
    \item Android Studio 3.3 o superiore;
    \item NodeJS 10.15 o superiore;
    \item Mozilla Firefox v.53 o superiore;
    \item AWS Amplify.
\end{itemize}
\textbf{Note}. La console di \textit{Amplify} può essere integrata in \textit{Android Studio} e consente di generare tutti gli SDK necessari senza accedere alla console di AWS. Il team duckware ha utilizzato questo approccio perché automatizza l'import ed il setup delle librerie all'interno dell'applicazione.\\[0.25cm]
Tuttavia è anche possibile utilizzare la versione standalone della console ma in questo caso è richiesto il setup manuale delle librerie e l'importazione dei file necessari.\\[0.25cm]Di seguito vengono riportati i requisti per l'installazione del software sopra elencato per le distribuzioni di Windows e Mac OSX:
\subsection{Requisiti per Windows}
\begin{itemize}
    \item Sistema operativo: Windows 10, 32 o 64 bit;
    \item RAM: 8GB di RAM;
    \item Disco fisso: 4GB di spazio libero richiesto;
    \item Connessione ad internet richiesta.
\end{itemize}
\subsection{Requisiti per Mac OSX}
\begin{itemize}
    \item Sistema operativo: MacOS 10.14 Mojave o superiore, 64 bit;
    \item RAM: 8GB di RAM;
    \item Disco fisso: 4GB di spazio libero richiesto;
    \item Connessione ad internet richiesta.
\end{itemize}

\subsection{Installazione}
Creare una cartella per il progetto \textit{SwetlApp} ed utilizzando Git da command line fare una copia del seguente comando:
\begin{verbatim}
    git clone https://github.com/Andreapava/swetlAPP.git
\end{verbatim}
altrimenti fare un clone del progetto da client GUI di Git copiando l'indirizzo HTTPS \href{https://github.com/Andreapava/swetlAPP.git}{https://github.com/Andreapava/swetlAPP.git}.\\Creare poi una cartella per il codice sorgente delle Skill per Alexa e, utilizzando nuovamente Git con CLI oppure il client GUI, utilizzando il seguente comando:
\begin{verbatim}
    git clone https://github.com/duckware-swe/Swetlapp.git
\end{verbatim}
altrimenti fare il clone del progetto copiando l'indirizzo HTTPS \href{https://github.com/duckware-swe/Swetlapp.git}{https://github.com/duckware-swe/Swetlapp.git}

\subsection{Esecuzione}
\subsubsection{Applicazione Android}
Per poter utilizzare e testare \textit{SwetlApp} è sufficiente eseguire il comando \emph{Build and Run} da Android Studio e fare il deploy dell'applicazione su un dispositivo reale o emulato. Nel secondo caso, sono sufficienti le impostazioni di default generate dall'AVD manager.
\subsubsection{Skill Alexa}
Per eseguire la \Gls{skill} per \Gls{Alexa}, è necessario seguire questi passi per poter eseguire il codice su AWS:
\begin{enumerate}
    \item Accedere a \url{https://developer.amazon.com/it/alexa-skills-kit/} ed effettuare il login;
    \item Navigare su \emph{Le mie console Alexa} e cliccare su \emph{Skills};
    \item Utilizzare i bottoni proposti dall'interfaccia per creare, modificare e cancellare skill.
\end{enumerate}