\pagebreak

\section{Test}
\label{sec:test}
\subsection{Scopo del paragrafo}
Questo paragrafo ha lo scopo di indicare agli sviluppatore come controllare le azioni del proprio codice e la sua sintassi.

\subsection{Test in Android Studio}
In Android Studio è possibile eseguire test sia per l'applicazione, con Java e \markg{\Gls{JUnit}}, sia per la Skill utilizzando Amplify ed il suo framework interno.
\begin{itemize}
    \item I test per il codice dell'applicazione sono presenti in Android Studio all'interno dell'apposito progetto \emph{test}. Questi sono stati creati con il supporto della libreria jUnit 5 e possono essere eseguiti direttamente dall'editor di Android Studio cliccando sul pulsante verde di \emph{Run}.
    \item I test per il \emph{GraphQL} sono integrati nella console di Amplify.
\end{itemize}

\subsection{Test in ambiente AWS}
Per testare la Skill è necessario accedere al portale sviluppatori\footnote{\href{https://developer.amazon.com/alexa/console/ask}{https://developer.amazon.com/alexa/console/ask}} di Alexa e selezionare la Skill che si vuole testare. In quest'area si potranno eseguire dei test preimpostati e vedere i risultati su:
\begin{itemize}
    \item \textbf{Logger:} una finestra che mostra un resoconto di tutte le azioni eseguite dal Back-end e le risposte ricevute da Alexa:
    \item \textbf{Alexa:} se un dispositivo fisico Alexa è disponibile, verranno eseguiti direttamente i test su di esso.
\end{itemize}