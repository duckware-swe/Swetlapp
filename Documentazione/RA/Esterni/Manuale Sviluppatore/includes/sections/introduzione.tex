\pagebreak
\section{Introduzione}
\label{sec:intro}
\subsection{Scopo del documento}
Lo scopo di questo documento è quello di fornire tutte le informazioni necessarie per estendere, migliorare e correggere \textit{SwetlApp}. Ci saranno ulteriori informazioni riguardanti la configurazione dell'ambiente di sviluppo che consentiranno di lavorare nelle stesse condizioni del team \textit{duckware}. 
Questa guida è stata scritta tenendo in considerazione i sistemi operativi Microsoft Windows, MacOS Mojave e sistemi UNIX. Nel caso venissero utilizzati altri sistemi operativi, potrebbero esserci dei problemi di compatibilità ma si prevede che le differenze siano minime o nulle.
\subsection{Scopo del prodotto}
Lo scopo di questo prodotto è quello di creare un'applicazione, ovvero una \textit{\Gls{skill}}, per l'assistente vocale \textit{\Gls{Amazon} \Gls{Alexa}} in grado di eseguire dei \textit{\gls{workflow}} personalizzati dall'utente per mezzo di un applicazione \Gls{Android}.
Il \gls{front-end} del sistema consiste di in un'applicazione nativa creata in \Gls{Java}. Il \gls{back-end} è stato implementato con i servizi di Amazon Web Services: Amplify, Cognito e AppSync; le skill di Alexa sono state realizzate con NodeJS.
\subsection{Riferimenti}
\begin{itemize}
    \item \textbf{Git}\\ \href{https://git-scm.com/}{https://git-scm.com/}
    \item \textbf{Node}\\ \href{https://nodejs.org/en/}{https://nodejs.org/en/}
    \item \textbf{Java}\\ \href{https://www.java.com/en/}{https://www.java.com/en/}
    \item \textbf{Android}\\ \href{https://developer.android.com/}{https://developer.android.com/}
    \item \textbf{Amplify}\\ \href{https://aws.amazon.com/it/amplify/}{https://aws.amazon.com/it/amplify/}
    \item \textbf{AppSync}\\ \href{https://aws.amazon.com/it/appsync/}{https://aws.amazon.com/it/appsync/}
    \item \textbf{GraphQL}\\ \href{https://graphql.org/}{https://graphql.org/}
    
\end{itemize}