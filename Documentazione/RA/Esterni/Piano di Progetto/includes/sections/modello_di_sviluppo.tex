\clearpage
\section{Modello di sviluppo}
\label{sec:modello_sviluppo}
Come modello di sviluppo per il ciclo di vita del \markg{software} è stato deciso di adottare il \markg{modello agile}, che permette una maggiore flessibilità e adattamento al gruppo nel suo ciclo evolutivo.\\
Il modello agile consente e induce all'iteratività, favorendo il dialogo con gli \markg{stakeholder} nel caso in cui ogni aspetto del sistema non sia compreso fin dall'inizio, consentendo maggiore capacità di adattamento, ciò però comporta un rischio di non convergenza poiché ogni iterazione comporta un ritorno ad uno stadio di sviluppo precedente.
Dunque l'approccio migliore è quello di decomporre la realizzazione del sistema, identificando le componenti più critiche in modo da limitare superiormente il numero delle iterazioni.\\
Durante ogni ciclo iterativo verranno effettuate delle migliorie nei documenti e dei prodotti realizzati dal gruppo: questo permette di adattarsi ai \markg{requisiti}, basandosi sul feedback delle iterazioni precedenti.
\subsection{Modello agile}
\label{sec:modello_agile}
Il modello agile è un modello altamente dinamico, costituito da cicli iterativi, ideato con l'intento di svincolarsi dall'eccessiva rigidità.\\ Esso si basa su quattro principi:
	\begin{itemize}
		\item L'interazione con gli stakeholder va incentivata, questa viene infatti considerata la miglior risorsa disponibile durante lo sviluppo del progetto;
		\item È più importante avere software funzionante che documentazione rigorosa;
		\item È importante la collaborazione con i clienti, in quanto essa produce risultati migliori rispetto ai soli rapporti contrattuali;
		\item Bisogna essere pronti a rispondere ai cambiamenti oltre che aderire alla pianificazione.
	\end{itemize}
L'idea di base per implementare tali principi è l'utilizzo della \markg{user story}, il lavoro viene quindi suddiviso in piccoli incrementi a valore aggiunto che vengono sviluppati indipendentemente in una sequenza continua dall'analisi all'integrazione.\\ Gli obiettivi strategici sono quindi:
	\begin{itemize}
		\item Poter costantemente e in ogni momento dimostrare al cliente ciò che è stato fatto;
		\item Verificare l'avanzamento tramite il progresso reale;
		\item Soddisfare e motivare gli sviluppatori con risultati immediati;
		\item Assicurare e dimostrare una buona verifica e integrazione dell'intero prodotto software. 
	\end{itemize}
I \textbf{vantaggi} principali di questo modello, sopra citati, sono quindi la facilità di adattamento in base alle esigenze da parte del gruppo, creando delle suddivisioni con piccoli incrementi di valore aggiunto con la possibilità di iterare per far sì che venga a pieno il soddisfacimento delle richieste del proponente.
\subsubsection{Modalità realizzativa}
Nel contesto del progetto in questione, il modello di sviluppo agile sarà adottato dai membri del gruppo secondo i seguenti punti cardine:
\begin{itemize}
	\item \textbf{Sprint}: Il periodo di sviluppo software del prodotto sarà caratterizzato dal susseguirsi di sprint di circa 3 settimane. Ogni sprint dovrà essere inizialmente pianificato sulla base delle attività con priorità maggiore, e dovrà concludersi con un incremento del valore aggiunto del prodotto. Durante l'ultimo giorno di sprint, il team dovrà quantificare e valutare obiettivamente il lavoro svolto in quel periodo di tempo, oltre a dover proporre migliorie o altre caratteristiche da implementare. 
	\item \textbf{Riunioni settimanali}: I membri del team si impegnano ad effettuare un incontro settimanale, nella quale ci si aggiorna sullo stato di avanzamento dei lavori o su eventuali problematiche incontrate durante lo sviluppo.
	\item \textbf{Coinvolgimento degli stakeholder}: Per l'attuazione del modello di sviluppo agile è fondamentale avere un feedback costante da parte degli stakeholder e/o dal cliente. Per questo motivo, ogni scelta implementativa e realizzativa presa dal gruppo, dovrà essere confermata dagli stakeholder.
	\item \textbf{Aggiornamento giornaliero}: Ogni membro di \emph{duckware} si impegna a fornire ai propri collaboratori aggiornamenti quotidiani sullo stato di lavorazione della parte assegnatagli, così da diminuire la probabilità di rischi correlati alla poca trasparenza.
	\item \textbf{Discussione feedback}: Dopo la pubblicazione di ogni valutazione e a seguito di ogni confronto con gli stakeholder, il gruppo si riserva una giornata nella quale verranno discussi gli esiti e le critiche mosse al prodotto realizzato fino a quel momento, nonchè le contromisure da adottare per far fronte a tali critiche. 
\end{itemize}