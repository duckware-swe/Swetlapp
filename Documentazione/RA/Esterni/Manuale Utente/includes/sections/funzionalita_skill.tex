\clearpage
\section{Funzionalità Skill Amazon Alexa}
\label{sec:sec_interazione_amazon_alexa}
Per poter usufruire delle funzionalità previste dalla skill \textit{MegAlexa} è necessario far riferimento ad un dispositivo fisico (o virtuale) con abilitata la tecnologia Amazon Alexa.
\subsection{Avvio della Skill}
Per avviare l'esecuzione della skill è necessario pronunciare il nome "swetlapp" che identifica la skill creata dal gruppo \textit{Duckware}. Un esempio di avvio della skill, quindi, potrebbe essere il seguente:
\begin{itemize}
	\item Utente: "Alexa, esegui swetlapp".
	\item Alexa: "Ciao \textit{nome utente}. Benvenuto in swetlapp".
\end{itemize}
Dopo che la skill è stata avviata si possono impartire comandi per l'esecuzione di un particolare workflow.
\subsubsection{Esecuzione di un workflow}
Un esempio di dialogo per l'esecuzione di un workflow potrebbe essere il seguente:
\begin{itemize}
	\item Alexa: \textit{Alexa attende un'ordine da parte dell'utente}.
	\item Utente: "Avvia \textit{nome del workflow}".
\end{itemize}
\subsubsection{Riesecuzione di un workflow}
Nel caso in cui l'utente volesse ripetere l'esecuzione di un workflow è possibile impartire un comando del seguente tipo:
\begin{itemize}
	\item Alexa: \textit{Alexa attende un'ordine da parte dell'utente}.
	\item Utente: "Ripeti \textit{nome del workflow}" oppure "Ripeti workflow \textit{nome del workflow}".
\end{itemize}
\subsubsection{Cancellazione esecuzione di un workflow}
Per interrompere l'esecuzione di un workflow è necessario impartire ad Alexa un comando del genere:
\begin{itemize}
	\item Alexa: \textit{Alexa sta eseguendo un workflow}.
	\item Utente: "Cancella".
\end{itemize}
Alexa quindi interromperà l'esecuzione del workflow attuale e tornerà in attesa di un altro comando, restando comunque all'interno della Skill.
\subsubsection{Stop esecuzione di un workflow}
Per interrompere completamente l'esecuzione della Skill è necessario fare uso di un comando del seguente tipo:
\begin{itemize}
	\item Alexa: \textit{Alexa attende un'ordine da parte dell'utente}.
	\item Utente: "Ferma" oppure "Stop".
\end{itemize}
In questo caso, oltre ad interrompere l'esecuzione del workflow corrente, Alexa chiuderà anche la sessione della Skill.
\subsubsection{Aiuto riguardo alla Skill}
Nel caso l'utente desiderasse ottenere aiuto riguardo al modo di funzionamento della skill, è necessario far uso di una richiesta del genere:
\begin{itemize}
	\item Alexa: \textit{Alexa attende un'ordine da parte dell'utente}.
	\item Utente: "Aiutami" oppure "Ho bisogno di aiuto".
\end{itemize}

\subsection{Connettori che necessitano la registrazione a siti esterni}
Di seguito sono elencati i connettori che poter essere usati necessitano di un account propriamente verificato:
\begin{itemize}
	\item \textbf{Connettori per operazioni su Trello}: è necessario quindi essere registrati sulla piattaforma \href{https://trello.com}{Trello};
	\item \textbf{Connettori per lettura/scrittura di Tweet}: è necessario quindi essere registrati sulla piattaforma \href{https://twitter.com/}{Twitter}.
\end{itemize}


\subsection{Interazione Utente - Alexa per alcuni connettori}
Durante l’esecuzione del workflow alcuni connettori potrebbero necessitare di un’ulteriore
interazione vocale tra l’utente e l’assistente vocale. Di seguito viene esplicato in che modo
è strutturato il dialogo tra l’utente e Amazon Alexa.
%\subsubsection{Connettore Meteo}
%All'esecuzione del connettore Meteo la risposta di Alexa sarà l'esposizione delle previsioni meteorologiche. Non è prevista una frase o una tipologia di frase standard per questo tipo di connettore.
\subsubsection{Connettore TV Schedule}
All'esecuzione del connettore TV Schedule, Alexa chiederà all'utente di fornirle il nome del canale di cui desidera conoscere il palinsesto. Una volta ottenuta la risposta, Alexa necessiterà anche di un'orario indicativo da cui iniziare a esporre la programmazione, così da rispondere in maniera più precisa alla richiesta dell'utente.
Sono disponibili 72 canali, tra i principali della televisione italiana, in chiaro e satellitare.
\subsubsection{Connettore Pubblicazione Twitter}
Per la pubblicazione di un Tweet l'utente dovrà aver collegato il proprio account Twitter all'applicazione.
Durante l'esecuzione il dispositivo andrà a chiedere all'utente il testo del tweet da pubblicare nel seguente modo: "Qual è il corpo del messaggio da inviare?", dopo aver ricevuto una risposta, verrà chiesta una conferma di invio che, se positiva, consentirà ad Alexa di pubblicare il tweet. Viceversa, se la risposta è negativa, verrà chiesto di comunicare nuovamente il testo del tweet.
\subsubsection{Connettore Lettura Twitter}
Per la lettura degli ultimi tweet di un account Alexa risponderà leggendo gli ultimi 3 tweet pubblicati dall'account precedentemente scelto nel seguente modo: "Ultimi tweet di \textbf{Utente Twitter}: ... ".
\subsubsection{Connettore lettura schede da Trello}
Per poter usare questo connettore, è necessario effettuare il login al proprio account personale di Trello.
\begin{itemize}
	\item \textbf{Obiettivo di questo connettore}: questo connettore permette di leggere le prime tre schede assegnate all'utente di una lista facente parte di una bacheca a scelta dell'utente.
	\item \textbf{Interazione utente-Alexa}: un modo in cui potrebbe avvenire l'interazione tra l'utente e il dispositivo Alexa è il seguente:
	\begin{itemize}
		\item Alexa: "Dimmi il nome della bacheca di Trello da dove vuoi leggere le tue schede";
		\item Utente: "{\it nome bacheca esistente}".
		\begin{itemize}
			\item Alexa: "Ok adesso dimmi il nome della lista da dove leggere le tue schede".
			\item Utente: "{\it nome lista esistente}".
			\begin{itemize}
				\item Alexa: "{\it lettura delle schede}".
			\end{itemize}
			\item Utente: "{\it nome lista inesistente}".
			\begin{itemize}
				\item Alexa: "Riprova a dirmi il nome della lista da dove leggere le tue schede".
			\end{itemize}			
		\end{itemize}
		\item Utente: "{\it nome bacheca inesistente}".
		\begin{itemize}
			\item Alexa: "Riprova a dirmi il nome della bacheca di Trello da dove vuoi leggere le schede".
		\end{itemize}
	\end{itemize}
\end{itemize}

\subsubsection{Connettore aggiunta di una scheda su Trello}
Per poter usare questo connettore, è necessario effettuare il login al proprio account personale di Trello.
\begin{itemize}
	\item \textbf{Obiettivo di questo connettore}: permette di aggiungere una singola scheda ad una lista di una bacheca scelta dall'utente
	\item \textbf{Interazione utente-Alexa}: essendo un'operazione di pubblicazione di una scheda, durante l'esecuzione della skill per ogni scelta dell'utente il dispositivo Alexa chiederà conferma all'utente. A questo punto l'utente potrà dire {\it "si/certo/ok"} per confermare la scelta, altrimenti {\it "no" } per rifiutare la scelta. Se l'utente dice {\it "no" } allora avrà la possibilità di rifare l'ultima azione, quindi ad esempio potrà ridire il titolo della scheda da aggiungere se ha sbagliato a pronunciare al primo colpo.\newline
	Un modo in cui potrebbe avvenire l'interazione tra l'utente e il dispositivo Alexa è il seguente:
	\begin{itemize}
		\item Alexa: "Dimmi il nome della bacheca di Trello dove vuoi aggiungere la scheda";
		\item Utente: "{\it nome bacheca esistente}";
		\begin{itemize}
			\item Alexa: "Ok adesso dimmi il nome della lista dove aggiungere la scheda";
			\item Utente: "{\it nome lista esistente}";
			\begin{itemize}
				\item Alexa: "Ok adesso dimmi il titolo della scheda da aggiungere";
				\item Utente: "{\it titolo della scheda}";
				\item Alexa: "Ok. Dimmi la descrizione della scheda da aggiungere";
				\item Utente: "{\it descrizione della scheda}";
				\item Alexa: "{\it La scheda è stata aggiunta corretamente}";
			\end{itemize}
			\item Utente: "{\it nome lista inesistente}";
			\begin{itemize}
				\item Alexa: "Riprova a dirmi il nome della lista dove aggiungere la scheda".
			\end{itemize}		
		\end{itemize}
		\item Utente: "{\it nome bacheca inesistente}";
		\begin{itemize}
			\item Alexa: "Ok. Riprova a dirmi il nome della bacheca di Trello dove vuoi aggiungere la scheda";
		\end{itemize}
	\end{itemize}
\end{itemize}



