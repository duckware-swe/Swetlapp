\clearpage
\section{Resoconto}
	\subsection{Collaudo del prodotto}
	\label{sec:collaudo}
	I membri del gruppo andati all'incontro hanno presentato, con l'obbiettivo di eseguire un pre-collaudo, il prodotto in fase di raffinamento e conclusione al \emph{CEO Stefano Dindo di Zero12}. L'incontro è stato concordato per la data di mercoledì 08 Maggio presso la loro sede, inviando una mail con il seguente testo:
	\begin{quote}
		\emph{Buongiorno,\\[0.25cm]vorremmo chiedere disponibilità per un incontro per mostrare lo stato di avanzamento del progetto e ricevere un feedback in previsione della Revisione di Accettazione che avremo il 17 Maggio.\\Pertanto chiediamo se possiamo venire presso la vostra sede per mostrare lo stato dei lavori possibilmente entro giovedì 9 maggio.\\Cordiali saluti.
		}
	\end{quote}
	Durante l'incontro i membri hanno esposto l'applicazione per ambiente Android OS su un dispositivo smartphone spiegando, in maniera colloquiale, le funzionalità che l'app presenta. Allo stesso tempo è stato dimostrato il funzionamento delle funzionalità dell'applicativo. Inoltre per l'esposizione della skill è sta attuata la stessa modalità, spiegando e dimostrando il funzionamento delle sue funzionalità.
	\subsection{Risultati e considerazione del collaudo}
	\label{sec:feedback}
	Alla termine del pre-collaudo il \emph{CEO Stefano Dindo di Zero12} ha espresso un giudizio positivo e soddisfatto sulla realizzazione del prodotto, dicendo che le funzionalità implementate sono interessanti e soddisfacenti, inoltre ha apprezza molto il tempismo nel comunicare e organizzare gli incontro fatti. Sono stati fatti notare alcuni appunti migliorativi sulla Skill per renderla più dinamica e appetibile all'uso.
