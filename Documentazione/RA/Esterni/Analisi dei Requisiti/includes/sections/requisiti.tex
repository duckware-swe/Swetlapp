\clearpage
\definecolor{tableHeadYellow}{rgb}{1.0, 0.88, 0.21}
\definecolor{tableLightYellow}{rgb}{1.0, 0.97, 0.86}
\definecolor{white}{rgb}{1.0, 1.0, 1.0}

\section{Requisiti}
\label{sec:requisiti}
\subsection{Classificazione dei requisiti}
\label{sec:classificazione_requisiti}
I requisiti vengono classificati ed assegnati attraverso l'utilizzo di un identificativo univoco secondo le regole riportate all'interno del documento Norme di progetto v4.0.0
\subsubsection{Requisiti funzionali}
\begin{center}
	\renewcommand{\arraystretch}{1.5}
	\rowcolors{3}{tableLightYellow}{}
	\begin{longtable}{  >{\RaggedRight}p{2.5cm}  
						>{\RaggedRight}p{2.1cm} 
						>{\RaggedRight}p{7cm}  
						>{\RaggedRight}p{1.7cm} 
						}
		\rowcolor{tableHeadYellow}
		\textbf{Identificativo}   & \textbf{Importanza} & \textbf{Descrizioni} & \textbf{Fonte} \\ 
		\endhead
		R2F1   & Obbligatorio & L'utente può registrarsi creando una sua area personale nella quale inserire workflow                & Capitolato e Verbale \\  
		R2F1.1 & Obbligatorio & La registrazione necessita di un indirizzo email valido                               & Interno              \\  
		R2F2   & Obbligatorio & L'utente può effettuare il login alla sua area personale                                              & Capitolato           \\  
		R2F2.1   & Obbligatorio & L'utente può effettuare il login con username e password creati dopo la registrazione               & Capitolato e Interno      \\  
		R2F2.2 & Obbligatorio & L'utente può recuperare la password in caso di smarrimento                                            & Interno              \\  
		R2F2.3 & Obbligatorio & L'utente può effettuare il logout                                                                    & Interno              \\  
		R1F3   & Desiderabile & L'utente può leggere una breve guida all'uso dell'applicazione                                        & Interno              \\  
		R2F4   & Obbligatorio & L'utente può gestire tutti i workflow che ha inserito                                                 & Capitolato           \\  
		R2F4.1 & Obbligatorio & L'utente può aggiungere un workflow alla sua lista personale                                          & Capitolato           \\  
		R2F4.2 & Obbligatorio & L'utente può rimuovere un workflow dalla sua lista personale                                          & Interno              \\  
		R2F4.3 & Obbligatorio & L'utente può modificare un workflow dalla sua lista personale                                         & Interno              \\  
		R2F4.4 & Obbligatorio & L'utente può vedere un numero che indica quanti workflow ha creato                                    & Interno              \\  
		R2F5   & Obbligatorio & Ogni utente avrà i suoi workflow personali che non saranno visibili ad altri utenti                   & Capitolato           \\  
		R2F6   & Obbligatorio & Ogni workflow dovrà avere un nome identificativo di almeno 4 caratteri                                & Interno              \\
		R2F7   & Obbligatorio & L'utente può vedere una lista di connettori da assegnare al workflow su cui sta operando              & Capitolato           \\  
		R2F7.1 & Obbligatorio & L'utente può vedere una lista di connettori da rimuovere dal workflow su cui sta operando             & Interno              \\  
		R2F7.2 & Obbligatorio & L'utente deve poter selezionare più di un connettore per ciascun workflow                             & Verbale              \\  
		R0F7.3 & Opzionale    & L'utente deve vedere un conteggio di quanti connettori sono stati selezionati                         & Interno              \\  
		R2F7.4 & Obbligatorio & L'utente deve aggiungere almeno un connettore dalla lista per poter avviare il workflow               & Interno e Verbale    \\  
		R2F7.5 & Obbligatorio & L'utente può aggiungere lo stesso connettore una sola volta per workflow                                  & Interno              \\  
		R2F8   & Obbligatorio & L'applicazione dovrà supportare più lingue                                                            & Capitolato           \\  
		R1F8.1 & Desiderabile & L'applicazione caricherà automaticamente la lingua corrispondente alle impostazioni di località del dispositivo & Interno              \\
		R2F9   & Obbligatorio	& L'utente deve poter avviare i workflow creati tramite comando vocale impartito ad Amazon Alexa & Capitolato \\
		R1F10	& Desiderabile	& L'utente deve poter conoscere tramite comando vocale impartito ad Amazon Alexa la lista dei workflow personali creati in precedenza	& Interno \\
		R1F11	& Desiderabile	& L'utente deve poter chiedere un aiuto per utilizzare la skill tramite comando vocale impartito ad Amazon Alexa	& Interno \\
		R2F12	& Obbligatorio	& L'utente deve poter interrompere l'esecuzione della skill tramite comando vocale impartito ad Amazon Alexa	& Esterno e Interno \\
		R0F13	& Opzionale	& L'utente deve poter terminare il connettore attualmente in esecuzione per passare al successivo tramite comando vocale impartito ad Amazon Alexa	& Interno \\
		R0F14	& Opzionale & L'utente deve poter ripetere dall'inizio il workflow appena eseguito tramite comando vocale impartito ad Amazon Alexa. & Interno \\ 
		\rowcolor{white}
		\caption{Tabella requisiti funzionali}
	\end{longtable}
\end{center}

\subsubsection{Requisiti di qualità}
\begin{center}
	\renewcommand{\arraystretch}{1.5}
	\rowcolors{3}{tableLightYellow}{}
	\begin{longtable}{  >{\RaggedRight}p{2.5cm}  
						>{\RaggedRight}p{2.1cm} 
						>{\RaggedRight}p{7cm}  
						>{\RaggedRight}p{1.7cm} 
						}
		\rowcolor{tableHeadYellow}

		\textbf{Identificativo}   & \textbf{Importanza} & \textbf{Descrizioni} & \textbf{Fonte} \\ 
		\endhead

		R2Q1   & Obbligatorio & Per lo sviluppo del prodotto richiesto devono essere rispettati tutti i processi descritti nel documento Piano di Qualifica v4.0.0 & Interno    \\
		R2Q2   & Obbligatorio & L'approccio al codice Java dovrà seguire quanto riportato al seguente link: https://google.github.io/styleguide/javaguide.html                & Capitolato \\
		R2Q2.1 & Obbligatorio & Lo sviluppo del codice deve essere supportato dall'utilizzo di test di unità                                                        & Interno    \\  
		R2Q3   & Obbligatorio & Dovrà essere fornito un manuale utente in lingua italiana che tratterà l'uso dell'applicazione                                      & Verbale    \\  
		R2Q4   & Obbligatorio & Il codice sorgente deve essere caricato nella piattaforma Github di duckware                                                        & Interno    \\  
		R2Q5   & Obbligatorio & Per lo sviluppo del prodotto richiesto devono essere rispettate tutte le norme descritte nel documento Norme di Progetto v4.0.0 & Interno    \\  
		R2Q6   & Obbligatorio & L'applicazione dovrà essere creata utilizzando IntelliJ IDEA                                                                               & Interno               \\  
		R2Q6.1 & Obbligatorio & Il deploy dell'applicazione avverrà per mezzo degli strumenti di build forniti da IntelliJ IDEA                                            & Interno               \\  
		R2Q6.2 & Obbligatorio & Il debug dell'applicazione dovrà essere eseguito sull'emulatore ufficiale fornito da Google utilizzando il toolkit AVD                        & Interno               \\  
		R1Q6.3 & Desiderabile & Il debug dovrà avvenire in un reale dispositivo con dimensioni di schermo differenti da quelle impostate dall'emulatore creato tramite AVD & Interno               \\

		\rowcolor{white}
		\caption{Tabella requisiti di qualità}
	\end{longtable}
\end{center}
\clearpage
\subsubsection{Requisiti di vincolo}
\begin{center}
	\renewcommand{\arraystretch}{1.5}
	\rowcolors{3}{tableLightYellow}{}
	\begin{longtable}{  >{\RaggedRight}p{2.5cm}  
						>{\RaggedRight}p{2.1cm} 
						>{\RaggedRight}p{7cm}  
						>{\RaggedRight}p{1.7cm} 
						}
		\rowcolor{tableHeadYellow}

		\textbf{Identificativo}   & \textbf{Importanza} & \textbf{Descrizioni} & \textbf{Fonte} \\ 
		\endhead

		R1V1   & Desiderabile & L'applicazione dovrà essere sviluppata per dispositivi Android                                                                             & Capitolato            \\  
		R2V1.1 & Obbligatorio & L'applicazione dovrà essere sviluppata con Java 10 o superiore                                                                             & Interno               \\  
		R2V1.2 & Obbligatorio & L'applicazione dovrà supportare un livello di API Android che sia 26 o superiore                                                           & Interno               \\  
		R2V1.3 & Obbligatorio & Sarà necessario utilizzare JUnit per eseguire i test di unità                                                                              & Interno               \\  
		R2V2   & Obbligatorio & Il collegamento con i servizi di AWS avverrà tramite l'SDK AWS  di Amazon                                                                  & Capitolato            \\  
		R2V2.1 & Obbligatorio & Verrà utilizzata la versione 2.0 dell'SDK                                                                                                  & Interno               \\  
		R2V3   & Obbligatorio & Utilizzo di Amazon Amplify per la realizzazione di API per la comunicazione con l'applicazione                                         & Verbale               \\  
		R2V3.1 & Obbligatorio & Creazione di endpoint API RESTful tramite Amazon Amplify & Interno               \\  
		R2V4   & Obbligatorio & Utilizzo di AWS Lambda per l'esecuzione automatica di richieste HTTP create via Amplify & Verbale               \\  
		R2V5   & Obbligatorio & Creazione di un database non relazionale utilizzando Amazon DynamoDB                                                                       & Capitolato ed Esterno \\  
		R2V6   & Obbligatorio & Un utente può creare workflow solamente dopo aver effettuato il login con successo                                                         & Capitolato            \\
		R2V7   & Obbligatorio & Si dovrà creare un'architettura REST per comunicare con l'applicazione                                                              & Interno    \\
		R2V8	& Obbligatorio & Utilizzo di  Alexa Developer Console per la definizione del modello di interazione della skill	& Interno \\
		R2V9	& Obbligatorio & Utilizzo di AWS Lambda per l'esecuzione automatica di richieste da parte della skill	& Interno \\
		R1V10	& Desiderabile & La skill dovrà essere sviluppata con Node.js 8.10 o superiore	& Interno \\
		R2V11	& Obbligatorio	& La skill dovrà essere testata su Alexa Developer Console o con dispositivo fisico con supporto ad Amazon Alexa	& Capitolato ed Interno \\
		R2V12	& Obbligatorio	& Un utente potrà accedere alle funzionalità complete della skill solo dopo aver collegato un account Amazon valido	& Interno \\
		
		\rowcolor{white}		
		\caption{Tabella requisiti di vincolo}
	\end{longtable}
\end{center}
\subsubsection{Requisiti prestazionali}
\begin{center}
	\renewcommand{\arraystretch}{1.5}
	\rowcolors{3}{tableLightYellow}{}
	\begin{longtable}{  >{\RaggedRight}p{2.5cm}  
						>{\RaggedRight}p{2.1cm} 
						>{\RaggedRight}p{7cm}  
						>{\RaggedRight}p{1.7cm} 
						}

		\rowcolor{tableHeadYellow}

		\textbf{Identificativo}   & \textbf{Importanza} & \textbf{Descrizioni} & \textbf{Fonte} \\ 

		R2P1 & Desiderabile & Il server deve avere una latenza massima di 3 secondi, a meno di errori di connessione                     & Interno \\  
		
		\rowcolor{white}
		\caption{Tabella requisiti prestazionali}
	\end{longtable}
\end{center}

\subsection{Tracciamento}
\label{sec:tracciamento}
\subsubsection{Tracciamento requisiti-fonti}
\begin{center}
	\renewcommand{\arraystretch}{1.5}
	\rowcolors{3}{tableLightYellow}{}
	\begin{longtable}{  p{5cm} p{5cm} }
		\rowcolor{tableHeadYellow}
		\textbf{Requisiti} & \textbf{Fonti} \\
		\endhead 
		
		R2F1 & Capitolato \newline Verbale \newline UC4 \newline UC5 \\
		R2F1.1 & Interno \newline UC4 \\
		R2F2 & Capitolato \newline UC2 \newline UC2.1 \\
		R2F2.1 & Capitolato \newline Interno \newline UC2 \newline UC2.1 \newline UC4 \\
		R2F2.2 & Interno \newline UC2 \newline UC2.1 \newline UC4 \newline UC5 \\
		R2F2.3 & Interno \newline UC3 \\
		R1F3 & Interno \newline UC1 \newline UC6 \\
		R2F4 & Capitolato \newline UC7 \newline UC8 \newline UC9 \newline UC10 \newline UC11 \newline UC12 \\
		R2F4.1 & Capitolato \newline UC7 \\
		R2F4.2 & Interno \newline UC9 \\
		R2F4.3 & Interno \newline UC8 \newline UC10 \newline UC11 \newline UC12 \\
		R2F4.4 & Interno \newline UC7 \newline UC8 \newline UC9 \\
		R2F5 & Capitolato \newline UC7 \newline UC8 \newline UC9 \\
		R2F6 & Interno \newline UC7 \newline UC8 \\
		R2F7 & Capitolato \newline UC8 \newline UC10 \newline UC11 \\
		R2F7.1 & Interno \newline UC8 \newline UC12 \\
		R2F7.2 & Verbale \newline UC8 \newline UC10 \newline UC11 \\
		R0F7.3 & Interno \newline UC8 \newline UC10 \newline UC11 \\
		R2F7.4 & Interno \newline Verbale \newline UC7 \newline UC8 \\
		R2F7.5 & Interno \newline UC8 \newline UC10 \\
		R2F8 & Capitolato \newline UC2 \newline UC2.1 \newline UC4 \\
		R1F8.1 & Interno \newline UC2 \newline UC2.1 \newline UC4 \\
		R2F9 & Capitolato \newline UC19 \\
		R1F10 & Interno \newline UC20 \\
		R1F11 & Interno \newline UC21 \\
		R2F12 & Esterno \newline Interno \newline UC22 \\
		R0F13 & Interno \newline UC23 \\
		R0F14 & Interno \newline UC19 \\
		R2Q1 & Interno \\
		R2Q2 & Capitolato \\
		R2Q2.1 & Interno \\
		R2Q3 & Verbale \\
		R2Q4 & Interno \\
		R2Q5 & Interno \\
		R2Q6 & Interno \\
		R2Q6.1 & Interno \\
		R2Q6.2 & Interno \\
		R2Q6.3 & Interno \\
		R1V1 & Capitolato \\
		R2V1.1 & Interno \\
		R2V1.2 & Interno \\
		R2V1.3 & Interno \\
		R2V2 & Capitolato \\
		R2V2.1 & Interno \\
		R2V3 & Verbale \\
		R2V3.1 & Interno \\
		R2V4 & Verbale \\
		R2V5 & Capitolato \newline Esterno \\
		R2V6 & Capitolato \\
		R2V7 & Interno \\
		R2V8 & Interno \\
		R2V9 & Interno \\
		R1V10 & Interno \\
		R2V11 & Capitolato \newline Interno \\
		R2V12 & Interno \\		
		R2P1 & Interno \\
		\rowcolor{white}
	\caption{Tabella tracciamento requisiti-fonti}
	\end{longtable}
\end{center}

\subsubsection{Tracciamento fonti-requisiti}
\begin{center}
	\centering
	\renewcommand{\arraystretch}{1.5}
	\rowcolors{3}{tableLightYellow}{}
	\begin{longtable}{  p{5cm} p{5cm} }
		\rowcolor{tableHeadYellow}
		\textbf{Fonti} & \textbf{Requisiti} \\
		\endhead  
		
		Capitolato & R2F1 \newline R2F2 \newline R2F2.1 \newline R2F4 \newline R2F4.1 \newline R2F5 \newline R2F7 \newline R2F8 \newline R2F9 \newline R2Q2 \newline R1V1 \newline R2V2 \newline R2V5 \newline R2V6 \newline R2V11 \\
		Interno & R2F1.1 \newline R2F2.1 \newline R2F2.2 \newline R2F2.3 \newline R1F3 \newline R2F4.2 \newline R2F4.3 \newline R2F4.4 \newline R2F6 \newline R2F7.1 \newline R0F7.3 \newline R2F7.4 \newline R2F7.5 \newline R1F8.1 \newline R1F10 \newline R1F11 \newline R2F12 \newline R0F13 \newline R0F14 \newline R2Q1 \newline R2Q2.1 \newline R2Q4 \newline R2Q5 \newline R2Q6 \newline R2Q6.1 \newline R2Q6.2 \newline R2Q6.3 \newline R2V1.1 \newline R2V1.2 \newline R2V1.3 \newline R2V2.1 \newline R2V3.1 \newline R2V7 \newline R2V8 \newline R2V9 \newline R1V10 \newline R2V11 \newline R2V12 \newline R2P1 \\
		Esterno	& R2F12 \newline R2V5 \\
		UC1 & R1F3 \\
		UC2 & R2F2 \newline R2F2.1 \newline R2F2.2 \newline R2F8 \newline R2F8.1 \\
		UC2.1 & R2F2 \newline R2F2.1 \newline R2F2.2 \newline R2F8 \newline R2F8.1 \\
		UC3 & R2F2.3 \\
		UC4 & R2F1 \newline R2F2.1 \newline R2F2.1 \newline R2F2.2 \newline R2F8 \newline R2F8.1 \\
		UC5 & R2F1 \newline R2F2.2 \\
		UC6 & R1F3 \\
		UC7 & R2F4 \newline R2F4.1 \newline R2F4.4 \newline R2F5 \newline R2F6 \newline R2F7.4 \\
		UC8 & R2F4 \newline R2F4.3 \newline R2F4.4 \newline R2F5 \newline R2F6 \newline R2F7 \newline R2F7.1 \newline R2F7.2 \newline R2F7.3 \newline R2F7.4 \newline R2F7.5 \\
		UC9 & R2F4 \newline R2F4.2 \newline R2F4.4 \newline R2F5 \\
		UC10 & R2F4 \newline R2F4.3 \newline R2F7 \newline R2F7.2 \newline R2F7.3 \newline R2F7.5 \\
		UC11 & R2F4 \newline R2F4.3 \newline R2F7 \newline R2F7.2 \newline R2F7.3 \\
		UC12 & R2F4 \newline R2F4.3 \newline R2F7.1 \\
		\rowcolor{white}
		\caption{Tabella tracciamento fonti-requisiti}
	\end{longtable}
\end{center}

	
\subsection{Riepilogo}
\label{sec:riepilogo}
\begin{center}
	\renewcommand{\arraystretch}{1.5}
	\rowcolors{3}{tableLightYellow}{}
	\begin{longtable}{  >{\RaggedRight}p{1.8cm}  
						>{\Centering}p{2cm} 
						>{\Centering}p{2.2cm}  
						>{\Centering}p{1.8cm} 
						>{\Centering}p{1.2cm}
						}
		\rowcolor{tableHeadYellow}
		\textbf{Tipologia}   & \textbf{Obbligatori} & \textbf{Desiderabili} & \textbf{Opzionali} & \textbf{Totale}\\ 
		Funzionali    & 21          & 4            & 3         & 28     \\
		Di qualità    & 9           & 1            & 0         & 10      \\  
		Di vincolo    & 15          & 2            & 0         & 17     \\  
		Prestazionali & 0           & 1            & 0         & 1      \\  
		Totale        & 45         	& 8            & 3         & 56     \\
		\rowcolor{white}
		\caption{Tabella requisiti prestazionali}
	\end{longtable}
\end{center}
