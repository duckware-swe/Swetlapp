\clearpage
\section{Qualità del prodotto}
\label{sec:qualita_prodotto}
\subsection{Scopo}
\label{sec:qualita_prodotto_scopo}
Basandosi sullo standard ISO/IEC 9126, sono state individuate le qualità che secondo il gruppo \emph{duckware} risultano importanti nell'arco del ciclo di vita del prodotto.
\subsection{Prodotti}
\subsubsection{Qualità dei documenti}
\label{sec:qualita_documenti}
I documenti prodotti dal gruppo \emph{duckware} dovranno essere leggibili, comprensibili e corretti dal punto di vista ortografico, sintattico, logico e semantico.
\paragraph{Comprensione}
\begin{itemize}
	\item \textbf{Leggibilità}: i documenti prodotti devono essere leggibili e comprensibili a persone con almeno licenza di scuola superiore di primo grado;
	\item \textbf{Correttezza ortografica}: i documenti prodotti non devono presentare errori ortografici.
\end{itemize}
Verranno utilizzate le seguenti metriche definite nelle \textit{Norme di progetto} alla §3.6:
\begin{itemize}
	\item \hyperref[sec:qprodotto_tabella_metriche_obiettivi]{MPRDD001} Indice di Gulpease;
	\item \hyperref[sec:qprodotto_tabella_metriche_obiettivi]{MPRDD002} Correttezza ortografica.
\end{itemize}
\subsubsection{Qualità del software}
\label{sec:qualita_software_parag}
\paragraph{Funzionalità}\mbox{}\\
Il prodotto deve fornire tutte le funzionalità che sono state individuate durante la redazione del documento \emph{Analisi dei requisiti}.\\[0.4cm]
\textbf{Obiettivi di qualità} \ Il gruppo \emph{duckware} si impegna a perseguire:
\begin{itemize}
	\item \textbf{Adeguatezza}: le funzionalità offerte dal prodotto risultano conformi rispetto alle aspettative;
	\item \textbf{Accuratezza}: il prodotto fornisce i risultati attesi, soddisfacendo il livello di dettaglio richiesto;
	\item \textbf{Sicurezza}: il prodotto assicura la protezione dei dati e delle informazioni che gli verranno forniti, affinché non sia permesso nè l'accesso nè la modifica a utenti o sistemi non autorizzati.
\end{itemize}
Verranno utilizzate le seguenti metriche definite nelle \textit{Norme di progetto} alla §3.7:
\begin{itemize}
	\item \hyperref[sec:qualita_software]{MPRDS001} Copertura \markg{requisiti obbligatori};
	\item \hyperref[sec:qualita_software]{MPRDS002} Copertura \markg{requisiti accettati}.
\end{itemize}

\paragraph{Affidabilità}\mbox{}\\[0.4cm]
Il prodotto software deve svolgere correttamente le sue funzioni durante il suo utilizzo, anche in caso in cui si presentino situazioni non previste (anomale).\\ 
\textbf{Obiettivi di qualità} \ L'esecuzione del prodotto dovrà avere le seguenti caratteristiche:
\begin{itemize}
	\item \textbf{Maturità:} principalmente si vuole evitare che si verifichino dei malfunzionamenti in seguito a difetti del software;
	\item \textbf{Tolleranza agli errori:} nel caso in cui si verifichino degli errori, dovuti a guasti o ad un uso scorretto dell'applicativo, questi devono essere gestiti correttamente.
\end{itemize}
Verranno utilizzate le seguenti metriche definite nelle \textit{Norme di progetto} alla §3.7:
\begin{itemize}
	\item \hyperref[sec:qualita_software]{MPRDS003} Percentuale di \markg{failure};
	\item \hyperref[sec:qualita_software]{MPRDS004} Blocco operazioni non corrette.
\end{itemize}

\paragraph{Usabilità}\mbox{}\\[0.4cm]
Rappresenta la capacità del prodotto finale di poter essere usato e compreso facilmente, in ogni sua parte, da qualsiasi utente che lo voglia usare.\\[0.4cm]
\textbf{Obiettivi di qualità} \ Il prodotto dovrà puntare ai seguenti obiettivi di usabilità:
\begin{itemize}
	\item \textbf{Comprensibilità:} l'utente deve essere in grado di riconoscere le funzionalità che sono offerte dal software, e deve poter comprendere le sue modalità di utilizzo per raggiungere i risultati attesi;
	\item \textbf{Apprendibilità:} all'utente viene data la possibilità di poter imparare le funzionalità offerte dal software;
	\item \textbf{Operabilità:} le funzioni presenti devono essere coerenti con le aspettative dell'utente;
	\item \textbf{Attrattiva:} l'utilizzo del software deve risultare piacevole per l'utente.
\end{itemize}
Verranno utilizzate le seguenti metriche definite nelle \textit{Norme di progetto} alla §3.7:
\begin{itemize}
	\item \hyperref[sec:qualita_software]{MPRDS005} Comprensibilità delle funzioni offerte;
	\item \hyperref[sec:qualita_software]{MPRDS006} Facilità di apprendimento delle funzionalità.
\end{itemize}

\textbf{Misurazione}
Queste metriche di usabilità saranno misurate tramite alcune sessioni di prova con utenti esterni, così da ottenere \markg{feedback} reali e misurazioni attendibili. Per questa procedura non è ancora possibile stabilire che metriche usare per misurarne l'usabilità del prodotto. Verrà decisa in successiva revisione dopo un'analisi approfondita con la proponente.

\paragraph{Efficienza}\mbox{}\\[0.4cm]
Attraverso questa metrica è possibile determinare la capacità del prodotto di eseguire le funzionalità offerte nel minor tempo possibile.
Inoltre con la misurazione dell'efficienza si vuole anche ridurre il numero di risorse usate dal software per eseguire le funzionalità offerte.\\[0.4cm]
\textbf{Obiettivi di qualità} \ Il prodotto deve essere il più efficiente possibile secondo i seguenti criteri:
\begin{itemize}
	\item \textbf{Comportamento rispetto al tempo:} il software deve eseguire le funzionalità che offre in tempi adeguati;
	\item \textbf{Utilizzo delle risorse:} il software, per eseguire le sue funzionalità, deve avvalersi di un appropriato numero e tipo di risorse.
\end{itemize}
Verranno utilizzate le seguenti metriche definite nelle \textit{Norme di progetto} alla §3.7:
\begin{itemize}
	\item \hyperref[sec:qualita_software]{MPRDS007} Tempo di risposta.
	\item \hyperref[sec:qualita_software]{MPRDS009} Complessità ciclomatica.
\end{itemize}

\paragraph{Manutenibilità}\mbox{}\\[0.4cm]
Questa metrica indica la capacità del software di poter essere modificato, adattato o migliorato a seconda delle esigenze.\\[0.4cm]
\textbf{Obiettivi di qualità} \ Per misurare la misurabilità si andranno a valutare le seguenti caratteristiche del software:
\begin{itemize}
	\item \textbf{Stabilità:} a seguito di modifiche del software non devono insorgere effetti non voluti;
	\item \textbf{Testabilità:} si deve poter facilmente testare il software;
	\item \textbf{Modificabilità:} il software deve poter essere modificato in alcune delle parti che lo compongono;
	\item \textbf{Analizzabilità:} si deve poter identificare facilmente le possibili cause di eventuali errori/malfunzionamenti.
\end{itemize}
Verranno utilizzate le seguenti metriche definite nelle \textit{Norme di progetto} alla §3.7:
\begin{itemize}
	\item \hyperref[sec:qualita_software]{MPRDS008} Impatto delle modifiche;
	\item \hyperref[sec:qualita_software]{MPRDS010} Numero di metodi;
	\item \hyperref[sec:qualita_software]{MPRDS011} Variabili non utilizzate.
	\item \hyperref[sec:qualita_software]{MPRDS013} Rapporto linee di codice e commento.
\end{itemize}
\clearpage
\subsection{Tabella riassuntiva delle metriche e degli obiettivi}
\label{sec:qprodotto_tabella_metriche_obiettivi}
\textbf{Documenti}\\
Di seguito viene riportata la tabella riassuntiva delle metriche e degli obiettivi riconosciuti, con il range di accettazione e di ottimalità, per le misure effettuate nei documenti.
\label{sec:qprodotto_tabella_metriche_obiettivi}
\begin{center}	\renewcommand{\arraystretch}{1.5}
	\rowcolors{3}{tableLightYellow}{}
	\begin{longtable}{  >{\RaggedRight}p{2.8cm}  >{\RaggedRight}p{5cm} >{\RaggedRight}p{2.5cm}  >{\RaggedRight}p{2.5cm}  }
		\rowcolor{tableHeadYellow}
		\textbf{Metrica}   & \textbf{Obiettivo} & \textbf{Valori \mbox{accettati}} & \textbf{Valori \mbox{ottimali}}\\
		\textbf{MPRDD001} Indice di Gulpease & Leggibilità del documento & 50 < x < 100 & 60 < x < 100 \\
		\textbf{MPRDD002} Correttezza ortografica & Documenti privi di errori & 95\% privi & 100\% privi \\
		\rowcolor{white}
		\caption{Tabella delle metriche della qualità di documenti}
	\end{longtable}
\end{center}
\pagebreak
\textbf{Software}\\
Di seguito viene riportata la tabella riassuntiva delle metriche e degli obiettivi riconosciuti, con il range di accettazione e di ottimalità, per le misure effettuate nel software.
\label{sec:qualita_software}
\begin{center}
	\centering
	\renewcommand{\arraystretch}{1.5}
	\rowcolors{3}{tableLightYellow}{}
	\begin{longtable}{ >{\RaggedRight}p{2.8cm} >{\RaggedRight}p{5cm}  >{\RaggedRight}p{3cm} >{\RaggedRight}p{2.5cm}  }
		\rowcolor{tableHeadYellow}
		\textbf{Metrica}  & \textbf{Obiettivo} & \textbf{Range \mbox{accettazione}} & \textbf{Range \mbox{ottimale}} \\ 
		%\endhead
		\textbf{MPRDS001} & Copertura requisiti obbligatori & 100\% & 100\% \\
		\textbf{MPRDS002} & Copertura requisiti accettati & 60\% - 100\% & 80\% - 100\% \\
		\textbf{MPRDS003} & Percentuale di failure & 0\% - 5\% & 0\% \\
		\textbf{MPRDS004} & Blocco operazioni non corrette & 80\% - 100\% & 100\% \\
		\textbf{MPRDS005} & Comprensibilità delle funzioni offerte & 80\% -100\% & 85\% - 100\% \\
		\textbf{MPRDS006} & Facilità di apprendimento delle funzionalità & 0 - 30 min & 0-15 min \\
		\textbf{MPRDS007} & Tempo di risposta & 0 - 4 sec & 0 - 2 sec \\
		\textbf{MPRDS008} & Impatto delle modifiche & 0\% - 20\% & 0\% - 10\% \\
		\textbf{MPRDS009} & Complessità ciclomatica                & 0 - 30      &      0 - 30 \\
		\textbf{MPRDS010} & Numero di metodi                       & 2 - 10      &      3 - 8 \\
		\textbf{MPRDS011} & Variabili non utilizzate               & 0           &      0 \\
		\textbf{MPRDS012} & Numero di bug per linea                & 0 - 60      &      0 - 25 \\
		\textbf{MPRDS013} & Rapporto linee di codice e commento    & \textgreater { 0.20 }      & SV \textgreater { 0.30 } \\
		\rowcolor{white}
		\caption{Tabella delle metriche della qualità del software}
	\end{longtable}
\end{center}