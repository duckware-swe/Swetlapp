\pagebreak

\section{Preparazione ambiente di lavoro}
\label{sec:ambientelavoro}
\subsection{Scopo del paragrafo}
Questo paragrafo spiega come configurare l'ambiente di lavoro di modo che possa essere replicato da chiunque per operare nelle stesse condizioni del team \textit{duckware}. Tutti i tool che verranno elencati non sono obbligatori ma evitano l'insorgere di problemi legati alla compatibilità.

\subsection{Node.js}
Per la creazione della Skill di Alexa è necessario installare Node.js, disponibile sia per Windows che per Mojave. Successivamente, bisognerà aprire la console di Node.js ed eseguire:
\begin{verbatim}
    npm install aws-sdk
\end{verbatim}
In questo modo verranno installati tutti i pacchetti necessari per poter lavorare sul codice delle Skill. Si consiglia l'aggiornamento frequente della libreria \emph{aws-sdk} che viene settimanalmente aggiornata.

\subsection{Android Studio}
L'IDE scelto per lo sviluppo di \textit{SwetlApp} è Android Studio, il tool ufficiale di Google scaricabile gratuitamente e disponibile sia per Microsoft Windows sia per MacOS Mojave.\\[0.25cm]
Dal sito ufficiale è disponibile anche una versione per Linux ma questa non risulta essere stabile in tutte le distribuzioni e si possono riscontrare diversi problemi nella configurazione dei dispositivi virtuali.

\subsection{Amplify}
Amplify è un framework di AWS che facilita l'integrazione dei servizi di AWS nelle applicazioni Android e iOS. Nel progetto \emph{SwetlApp} è stato utilizzato per aggiungere all applicazione il servizio di autenticazione Cognito e per generare le API.\\[0.25cm]
L'installazione di Amplify avviene all'interno della CLI di Android Studio eseguendo il seguente comando, premesso che Node.js sia stato installato:
\begin{verbatim}
    npm install -g @aws-amplify/cli
\end{verbatim}
Per configurare l'ambiente al primo avvio e scaricare le dipendenze necessarie bisognerà utilizzare sempre nella CLI questo comando:
\begin{verbatim}
    amplify configure
\end{verbatim}
Per impostare permessi e ruoli e collegare il progetto al backend su cloud AWS che si aggiungerà sarà necessario spostarsi nella root del progetto e dare il comando
\begin{verbatim}
    amplify init
\end{verbatim}
Per modificare le API (riferite ad un endpoint GraphQL) sarà necessario modificare il file SwetlApp \textbackslash amplify \textbackslash backend \textbackslash api \textbackslash SwetlApp \textbackslash schema.graphql e dare il comando 
\begin{verbatim}
    amplify api push
\end{verbatim}
Per aggiungere plugin (per una spiegazione in dettaglio su questi riferirsi alla sezione Architecture del link di Amplify fornito) è necessario dare il comando:
\begin{verbatim}
    amplify <category> add
\end{verbatim}
dove <category> si riferisce al tipo di plugin.

\subsection{Java Development Kit}
Con l'installazione di Android Studio viene automaticamente installata la versione più recente del JDK, compatibile con l'IDE. È anche possibile installare la propria versione del kit anche se ci potrebbero essere dei problemi di compatibilità. Su Windows andrà installato \emph{JDK} mentre su MacOS \emph{OpenJDK}.