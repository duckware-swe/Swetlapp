\pagebreak

\section{Estensione delle funzionalità}
Il seguente paragrafo ha lo scopo di aiutare lo sviluppatore che intenda apportare modifiche all'architettura su AWS o estendere il codice sorgente.

\subsection{Architettura AWS}
L'intera architettura viene automaticamente generata dal tool \emph{Amplify} dopo aver interpretato il file GraphQL. Una volta modificato tale file, la console di Amplify genererà il database e l'intera infrastruttura per la comunicazione con l'applicazione Android e la Skill.\\[0.25cm]

\subsection{Android}
\subsubsection{Estensione delle risorse}
Per estendere le risorse sarà necessario:
\begin{itemize}
\item Localizzare il file swetlAPP/amplify/backend/api/testcognito/schema.graphql
\item Modificare questo file per adattarlo al modello necessario
\item Salvare il nuovo file
\item Da CLI su root di progetto digitare e inviare \textbf{amplify api push}
\item Attendere la sincronizzazione con il cloud AWS
\end{itemize}

\subsubsection{Accesso a nuove risorse}
Le classi e i metodi di accesso a queste risorse saranno disponibili nella cartella generatedJava, il comando \textbf{clean} eliminerà eventuali  classi associate a risorse non più esistenti. \\
Le query verranno create con i metodi generati producendo, se necessario, prima un input e poi passandolo come parametro alla query, gli eventi successivi a una query saranno gestiti da una callback con i metodi onResponse() e onFailure(). \\
Si consiglia di mettere gli update della view dentro questi due metodi.

\subsubsection{Estensione Front-end}
Sarà necessario agire sui seguenti file XML:
\begin{itemize}
\item \textbf{layout} per dichiarare e modificare viste e le posizioni delle stesse
\item \textbf{strings} per dichiarare e modificare stringhe utilizzate dall'applicazione
\item \textbf{colors} per dichiarare e modificare i colori utilizzati dalle viste
\item \textbf{styles} per dichiarare e modificare gli stili utilizzati dalle viste
\end{itemize}

\subsubsection{Cognito}
Per modificare le policy e il front-end della drop-in UI di AWS Cognito sarà necessario accedere alla console AWS Cognito da browser e navigare tra le categorie, nella categoria policy sarà possibile modificare la sicurezza necessaria per la password e l'obbligatorietà di vari attributi, nella categoria interfaccia si potranno modificare gli elementi della vista della drop-in UI tramite direttive CSS.

\subsubsection{Implementazione nuovi connettori}
Se si vuole aggiungere un connettore i cui parametri necessitano all'utente solo di digitare in campi di testo sarà sufficiente dichiarare il connettore (tramite il builder) nel metodo onCreate di ConnectorActivity, per come è implementato, il sistema provvederà ad aggiungerlo nelle opportune RecyclerView e a costruire il comportamento di inserimento dei parametri e loro archiviazione nel database.
Per connettori i cui input sono differenti da normali field testuali sarà necessario, oltre ad effettuare il processo suddetto, creare una activity che lo gestisca.

\subsection{Skill}
Todo \textbf{SONIA}
