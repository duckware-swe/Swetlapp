\clearpage
\section{Resoconto}
	\subsection{Discussione e pianificazione delle correzioni}
	\label{sec:correzioni}
	Il team in questo incontro ha esaminato e verificato il feedback ricevuto dopo la presentazione della Revisione di Qualifica avvenuta il 19-04-2019. Sono stati quindi riportati tutte le correzioni necessarie per migliorare i prodotti in vista dell'ultima revisione, la Revisione di Accettazione in data 17-05-2019. Di seguito vengono riportati i documenti che subiranno correzioni e integrazioni:
	\begin{itemize}
		\item Verbali;
		\item Norme di Progetto;
		\item Analisi dei Requisiti;
		\item Piano di Progetto;
		\item Piano di Qualifica;
		\item Manuali
			\begin{itemize}
				\item Manuale Utente;
				\item Manuale Sviluppatore.
			\end{itemize}
	\end{itemize}
	Il team terrà presente anche le correzioni e accuratezze che sono state fatte sul prodotto software dopo la \textit{Product Baseline} il 10-04-2019.
	\subsection{Modiche della pianificazione}
	\label{sec:pianificazione}
	È stata fatta una ripianificazione dei planning in funzione del feedback ricevuto dalla Revisione di Qualifica. Di seguito si riportano le date riformulate e le nuove aggiunte.
	\begin{center}
		\textbf{Sprint da RQ a RA}
		\renewcommand{\arraystretch}{1.5}
		\rowcolors{3}{tableLightYellow}{}
		\begin{longtable}{  p{2.5cm} p{6cm} }
			\rowcolor{tableHeadYellow}
			\textbf{Data}&\textbf{Tipo di incontro}\\
			30-04-2019 & Sprint retrospective e sprint review \\
			02-05-2019 & Sprint planning\\
			13-05-2019 & Discussione post consegna\\
			\rowcolor{white}
			\caption{Ruoli verbali}
			\label{sec:tabella2}
		\end{longtable}	
	\end{center}