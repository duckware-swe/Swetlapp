\clearpage
\section{Resoconto}
	\subsection{Considerazioni dello sprint terminato}
	\label{sec:considerazioni}
	Il team in questa riunione ha discusso sulle correzioni da apportare sui vari documenti, nello specifico quei aggiustamenti inerenti ai grafici dei casi d'uso, delle classi e dei package. Dal feedback ricevuto dopo la Revisione di Qualifica è stato deciso di inviare una mail per chiedere chiarimenti al \textit{Prof. Cardin} riguardo l'individuazione dei sotto-casi d'uso nel diagramma in fig. 2, ovvero l'autenticazione presente nel documento Analisi dei Requisiti v3.0.0. Di seguito viene riportata la mail inviata al Professore per ricevere chiarimenti a riguardo:
	\begin{quote}
		\emph{Buongiorno,\\[0.25cm]il team desidererebbe avere dei chiarimenti riguardo ai casi d’uso di autenticazione presenti nel documento Analisi dei Requisiti v3. Ci è stato fatto notare che non vengono individuati sotto-casi. La nostra fase di autenticazione, registrazione e recupero dati è gestita da Cognito, un servizio che è esterno. Dobbiamo quindi rappresentare dei sotto-casi per Cognito anche se il servizio è esterno?\\[0.25cm]Attendiamo risposta e la ringraziamo anticipatamente.\\Cordiali saluti.
		}
	\end{quote}
	\subsection{Miglioramento qualità esperienza d'uso del prodotto}
	\label{sec:migliorie}
	Durante l'incontro è stato fatto notare la necessità di migliorare l'esperienza d'uso all'interno del applicazione per smartphone Android. Si decide quindi rendere esteticamente più accattivante l'interfaccia grafica e la sua usabilità per migliorare l'esperienza di utilizzo. 
