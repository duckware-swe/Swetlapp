\clearpage
\section{Resoconto}
	\subsection{Scelte e proposte implementative}
	\label{sec:implementazione}
	Durante questo incontro di \textit{sprint planning} viene proposto di implementare l’autenticazione appoggiandosi a \textit{Cognito} e migliorare l’interazione con \textit{DynamoDB} tramite \textit{Amplify}: andranno quindi rimosse le Lambda che servivano da supporto per effettuare le \textit{Query} dall’applicazione Android. In seguito a tale proposta andranno rivisti i Design Pattern dell'applicazione e verrà fatto un refactoring del codice. Viene inoltre proposto di aggiungere le seguenti funzionalità sulla Skill: previsioni meteorologiche e il palinsesto televisivo.
	\subsection{Scelte e proposte migliorative}
	\label{sec:migliorie}
	Sono state fatte alcune proposte migliorative per l'interfaccia grafica in relazione di un analisi dei Design Pattern.
	\subsection{Richiesta ricevimento Prof. Tullio Vardanega}
	\label{sec:ricevimento}
	Dopo un confronto collettivo il team ha deciso di richiedere un colloquio al Prof. Tullio Vardanega.\\
	Lo scopo dell'incontro è di chiarire alcuni dubbi riguardante le segnalazioni avute durante la Revisione di Progettazione.\\
	Di seguito viene lasciato il contenuto della mail:
	\begin{quote}
	\emph{Gentile Professore\\[0.25cm]Il Gruppo Duckware desidererebbe chiedere disponibilità per un ricevimento in data 27-03-2019. Se non dovesse avere disponibilità ci indichi Lei il giorno e l'orario.\\[0.25cm]Distinti saluti.
	}
	\end{quote}
	Risposta Prof. Tullio Vardanega:
	\begin{quote}
	\emph{Volentieri: vi presenterete mercoledì 27, alle ore 12:30, in studio da me.}
	\end{quote}
	\subsection{E-Mail inviata al CEO Stefano Dindo di Zero12}
	\label{sec:mail_stefano}
	Il team \textit{duckware} ha sollevato un dubbio per quanto riguarda la stesura di due documenti destinati alla proponente \textit{Zero12}.\\
	Di seguito viene lasciato il contenuto della mail con la domanda posta:
	\begin{quote}
		\emph{Buongiorno Stefano.\\[0.25cm]Il gruppo per la prossima valutazione dovrà presentare, oltre alla documentazione già realizzata fin ora, due manuali: Manuale Utente e Manuale Programmatore. Vorremmo sapere se preferite questi due documenti in lingua inglese o in italiano come il resto della documentazione.\\[0.25cm]Inoltre vorremmo avere conferma del fatto che l'iterazione vocale con la skill debba essere realizzata solo in lingua italiana e non anche in lingua inglese.
		}
	\end{quote}
\pagebreak
	\subsection{Rotazione dei ruoli sui documenti per lo sprint corrente}
	\label{sec:rotazione}
	Viene fatta una rotazione dei ruoli per lo sprint corrente (durante la Revisione di Qualifica) in base alle ore pianificate per ogni singolo membro del gruppo.\\	
	\begin{center}
		\textbf{Verbali}
		\renewcommand{\arraystretch}{1.5}
		\rowcolors{3}{tableLightYellow}{}
		\begin{longtable}{  p{2.5cm} p{3.5cm} }
			\rowcolor{tableHeadYellow}
			\textbf{Ruolo}&\textbf{Membro}\\
			Responsabile & \alberto \\
			Verificatore & \luca \\
			Redattore & \matteo \\
			\rowcolor{white}
			\caption{Ruoli verbali}
			\label{sec:tabella2}
		\end{longtable}	
		\textbf{Norme di Progetto}
		\renewcommand{\arraystretch}{1.5}
		\rowcolors{3}{tableLightYellow}{}
		\begin{longtable}{  p{2.5cm} p{4cm} }
			\rowcolor{tableHeadYellow}
			\textbf{Ruolo}&\textbf{Membro}\\
			Responsabile & \alberto \\
			Verificatore & \mbox{\sonia}, \mbox{\alessandro} \\
			Redattore & \mbox{\matteo}, \mbox{\luca}, \mbox{\pardeep}, \mbox{\andrea} \\
			\rowcolor{white}
			\caption{Ruoli documento Norme di Progetto}
			\label{sec:tabella3}
		\end{longtable}	
		\textbf{Piano di Progetto}
		\renewcommand{\arraystretch}{1.5}
		\rowcolors{3}{tableLightYellow}{}
		\begin{longtable}{  p{2.5cm} p{4cm} }
			\rowcolor{tableHeadYellow}
			\textbf{Ruolo}&\textbf{Membro}\\
			Responsabile & \pardeep \\
			Verificatore & \mbox{\luca}, \mbox{\sonia} \\
			Redattore & \mbox{\matteo}, \mbox{\andrea}, \mbox{\alessandro}, \mbox{\alberto} \\
			\rowcolor{white}
			\caption{Ruoli documento Piano di Progetto}
			\label{sec:tabella4}
		\end{longtable}	
		\textbf{Piano di Qualifica}
		\renewcommand{\arraystretch}{1.5}
		\rowcolors{3}{tableLightYellow}{}
		\begin{longtable}{  p{2.5cm} p{4cm} }
			\rowcolor{tableHeadYellow}
			\textbf{Ruolo}&\textbf{Membro}\\
			Responsabile & \alessandro \\
			Verificatore & \mbox{\matteo}, \mbox{\alberto} \\
			Redattore & \mbox{\sonia}, \mbox{\andrea}, \mbox{\luca}, \mbox{\pardeep} \\
			\rowcolor{white}
			\caption{Ruoli documento Piano di Qualifica}
			\label{sec:tabella5}
		\end{longtable}	
		\textbf{Analisi dei Requisiti}
		\renewcommand{\arraystretch}{1.5}
		\rowcolors{3}{tableLightYellow}{}
		\begin{longtable}{  p{2.5cm} p{4cm} }
			\rowcolor{tableHeadYellow}
			\textbf{Ruolo}&\textbf{Membro}\\
			Responsabile & \alessandro \\
			Verificatore & \mbox{\andrea}, \mbox{\pardeep} \\
			Redattore & \mbox{\matteo}, \mbox{\luca}, \mbox{\alessandro}, \mbox{\sonia} \\
			\rowcolor{white}
			\caption{Ruoli documento Analisi dei Requisiti}
			\label{sec:tabella6}
		\end{longtable}	
	\end{center}
\subsection{Preparazione Product Baseline}
\label{sec:presentazione_pb}
A fine incontro sono stati organizzati i processi per la realizzazione dei diagrammi delle classi, di sequenza, per la presentazione con il \textit{Prof. Riccardo Cardin} e la stesura dei manuali Utente e Programmatore.