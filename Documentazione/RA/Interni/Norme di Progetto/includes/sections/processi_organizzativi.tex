\clearpage
\section{Processi organizzativi}
\label{sec:proc_org}
\subsection{Processi di coordinamento}
\label{sec:processi_coordinamento}
\subsubsection{Comunicazione}
Verranno ora illustrate le norme che regolano la comunicazione sia tra i membri di \emph{duckware} che con entità esterne quali i proponenti ed i committenti.

\paragraph{Comunicazioni interne}\mbox{}\\[0.4cm]
\addcontentsline{toc}{paragraph}{Comunicazioni interne}
Le comunicazioni interne di \emph{duckware} avverranno attraverso Slack, il quale consente la creazione di appositi canali tematici in ciascuno dei quali verrà trattato il relativo argomento. In particolare, i canali saranno così suddivisi:
\begin{itemize}
    \item \textbf{Generale: }Verrà discusso di argomenti generali riguardanti l’organizzazione del progetto, la scelta di strumenti di lavoro o per una rapida comunicazione. Verranno anche stabiliti gli argomenti di discussione durante gli incontri;
    \item \textbf{Incontri: }In questo canale verranno scritte dal Responsabile tutti gli argomenti da toccare durante la successiva sessione d’incontro nel programma. Inoltre, verrà stilato un breve riassunto di quanto deciso in merito ai punti da discutere;
    \item \textbf{GitLab monitor: }Grazie all’utilizzo di un bot, ogni modifica fatta da un utente all’interno della repository verrà segnalato all’interno di questo canale;
    \item \textbf{Link utili: }In questo canale non sarà possibile scrivere messaggi ma si potranno solamente visualizzare dei collegamenti a risorse interne che potranno essere utili durante lo sviluppo del progetto;
    \item \textbf{Glossario: }Ogni volta vi sia bisogno di un aggiornamento del glossario verranno qui notificati i termini con le relative definizioni da inserire. Una volta aggiornato il glossario, grazie a GlossaryHelper, i messaggi verranno rimossi per evitare confusione.
    \item \textbf{Sviluppo: } canale dedicato a Java ed allo sviluppo di applicazioni Android nel quale i membri con esperienze passate nella creazione di app possono aiutare i colleghi.
Vi sono dei canali, ciascuno avente il nome del documento al quale si riferiscono, che contengono un unico messaggio contente il link al file *.tex di documentazione. È stato creato inoltre un sotto-dominio all’interno del server di uno dei membri di \emph{duckware} per fare il backup di documentazione e file.
\end{itemize}
\paragraph{Comunicazioni esterne}\mbox{}\\[0.4cm]
In questa sottosezione vi sono le norme che regolano le comunicazioni con soggetti esterni a \emph{duckware}, come:
\begin{itemize}    
    \item La proponente zero12 rappresentata da Dindo Stefano, con i quali si intende stabilire un rapporto di collaborazione per la definizione dei requisiti necessari al fine di realizzare il prodotto;
    \item Prof Tullio Vardanega e Prof Riccardo Cardin ai quali verrà fornita la documentazione richiesta in ciascuna revisione di progetto. Con essi si intenderà dialogare per portare un continuo miglioramento al progetto.
\end{itemize}
Tutte le comunicazioni esterne vengono effettuate attraverso
\begin{center}
\emph{duckware.swe@gmail.com}
\end{center}
e ogni membro del gruppo ha accesso a tale indirizzo email. I proponenti verranno raggiunti attraverso gli indirizzi da loro forniti:
\begin{center}
\emph{s.dindo@zero12.it}
\end{center}

\subsubsection{Riunioni}
In questa sotto-sezione definiremo le norme relative alle riunioni interne ed esterne. Durante lo svolgimento di ogni riunione ci sarà un responsabile dell’incontro che si prenderà cura di far rispettare tutti i punti di discussione su cui riflettere. Il rapporto finale di quanto discusso verrà poi inserito all’interno del Verbale di Riunione. 

\paragraph{Verbale di riunione}\mbox{}\\[0.4cm]
Il responsabile di riunione redigerà il Verbale di Riunione secondo il seguente schema:
\begin{enumerate}
    \item \textbf{Frontespizio: }Il frontespizio sarà formato dall’indicazione della tipologia del verbale, ovvero Interno o Esterno, e dalla data nella quale tale incontro è avvenuto;
    \item \textbf{Registro modifiche: }Nell'ultima pagina sarà presente un registro delle modifiche relativo a tutti i cambiamenti apportati al verbale prima di giungere alla sua versione finale;
    \item \textbf{Informazione sulla riunione: }In questa sezione ci saranno:
    \begin{itemize}
        \item Lo scopo della riunione: il motivo per il quale si è deciso di fare una riunione;
        \item Data e luogo: quando e dove la riunione si è tenuta;
        \item Durata: include l’orario di inizio e l’orario di fine della riunione, in formato ventiquattro ore, ad esempio 17.40;
        \item Partecipanti alla riunione: nel caso di riunione esterna, verranno prima elencati i committenti e proponenti, altrimenti verrà stilata una lista di partecipanti.
    \end{itemize}
    \item \textbf{Ordine del giorno: }Elenco puntato degli argomenti da discutere;
    \item \textbf{Resoconto: }Riassunto redatto dal responsabile dell’incontro secondo quanto stabilito dai punti dell'ordine del giorno. Dovrà essere inserito in grassetto il titolo di ogni punto trattato e poi su una nuova riga la descrizione di quanto è stato deciso.
\end{enumerate}
Ogni file di un verbale verrà chiamato nel seguente modo: verbale\_{}DATA dove DATA è la data nel quale è stato svolto l'incontro. Ad esempio un titolo valido potrebbe essere verbale\_{}12-12-2018.

\paragraph{Riunioni interne}\mbox{}\\[0.4cm]
La partecipazione alle riunioni interne è rivolta solo ed esclusivamente ai membri di \emph{duckware} ed il Responsabile di progetto ha il compito di stilare l’ordine del giorno. Gli incontri avverranno in data e ora non predefinita, stabilita in maniera collettiva in base alla disponibilità della maggioranza del gruppo. I partecipanti dovranno essere puntuali alle riunioni e comunicare con quanto più anticipo, se possibile, eventuali ritardi o problemi di presenza. Una riunione può avvenire solo se ci sono almeno 5 partecipanti su 7.

\paragraph{Riunioni esterne}\mbox{}\\[0.4cm]
Le riunioni esterne coinvolgono sia la Proponente che i membri di \emph{duckware}; all’interno del verbale esterno relativo all’incontro verranno inseriti tutti gli argomenti trattati. Le riunioni verranno effettuate sede nella della proponente oppure tramite chiamata di gruppo su programmi di video conferenza come Skype. Quando possibile, gli incontri avverranno presso Torre Tullio Levi-Civita.

\subsection{Processi di pianificazione}
\label{sec:processi_pianificazione}
\subsubsection{Ruoli di progetto}
La realizzazione del progetto è il prodotto finale di una collaborazione coesa ed organizzata da parte dei membri del gruppo. Ogni membro avrà il suo ruolo all’interno di \emph{duckware} e ciò rifletterà le rispettive figure che ci sono all’interno di una azienda. A rotazione ogni membro del gruppo ricoprirà questi ruoli
\begin{itemize}
    \item Responsabile di progetto;
    \item Amministratore;
    \item Analista;
    \item Progettista;
    \item Programmatore;
    \item Verificatore.
\end{itemize}
Il cambio del ruolo avverrà in modo tale da poter garantire un costante livello di qualità. Il responsabile avrà l’onere di assegnare i ruoli in modo ragionato al fine di non creare situazioni ambigue. Ad esempio, non sarà possibile che un verificatore debba verificare dei documenti che sono stati creati dallo stesso in quanto non ci sarebbe una sufficiente analisi critica.
\paragraph{Responsabile di progetto}\mbox{}\\[0.4cm]
Il Responsabile di progetto, o Project manager, partecipa al progetto per tutta la sua durata e su di esso ricadono tutte le responsabilità di scelta ed approvazione. Inoltre è colui che rappresenta \emph{duckware} nei confronti della Proponente e dei committente. Egli dovrà:
\begin{itemize}
    \item Approvare i documenti;
    \item Elaborare piani, scadenze e coordinare le attività del gruppo;
    \item Relazionarsi con il controllo di qualità del progetto;
    \item Approvare l’offerta.
\end{itemize}

\paragraph{Amministratore}\mbox{}\\[0.4cm]
L’amministratore la figura è una figura fondamentale del gruppo poiché deve garantire l’efficienza e l’attività del gruppo stesso. Esso dovrà occuparsi dell’organizzazione dell’ambiente di lavoro redigendo i documenti che regolano le attività di verifica e di lavoro. Infatti le sue mansioni sono:
\begin{itemize}
    \item Redigere le Norme di Progetto e aiutare nella scrittura del Piano di Progetto;
    \item Assicurarsi che la documentazione sia corretta ed approvata;
    \item È responsabile della redazione di piani e procedure relativi alla Gestione per la Qualità.
\end{itemize}

\paragraph{Analista}\mbox{}\\[0.4cm]
L’attività degli analisti è necessaria affinché il progetto possa essere realizzato; essi dovranno analizzare il problema e comprenderlo a pieno, al fine di ridurre la probabilità di realizzare gravi errori di progettazione. Dovranno occuparsi nello specifico di:
\begin{itemize}
    \item Realizzare  lo studio di fattibilità, l’analisi dei requisiti e verificare le implicazioni di costi e qualità;
    \item Studiare e definire al meglio, senza ambiguità, il problema da risolvere per capire cosa deve essere realizzato in base ai requisiti richiesti;
    \item Modellare il problema ed effettuare la ripartizione dei requisiti.
\end{itemize}

\paragraph{Progettista}\mbox{}\\[0.4cm]
Il progettista è il responsabile delle attività di progettazione, le quali consistono nella definizione di una soluzione accettabile per ogni stakeholder coinvolto. I suoi obiettivi sono:
\begin{itemize}

    \item Definizione della struttura logica del prodotto in modo che sia ottimizzata e quindi facile da mantenere;
    \item Suddivisione del sistema in problemi di complessità ridotta e trattabili al fine di rendere il lavoro di codifica più facile da realizzare per i programmatori e più facile da verificare per i verificatori.
\end{itemize}

\paragraph{Programmatore}\mbox{}\\[0.4cm]
Il programmatore è responsabile del processo di codifica che porta alla realizzazione del prodotto. Il suo compito è quello di implementare attraverso specifici linguaggi quanto è stato definito dal Progettista nell’architettura del progetto. Esso dovrà:
\begin{itemize}
    \item Scrivere codice propriamente documentato, indentato e versionato;
    \item Scrivere il manuale utente in modo che contenga il minor numero possibile di ambiguità;
    \item Creare le componenti per la verifica e la validazione del codice.
\end{itemize}

\paragraph{Verificatore}\mbox{}\\[0.4cm]
Il verificatore deve essere presente durante tutta la durata del progetto in quanto si occuperà delle attività di verifica. Esse sono:
\begin{itemize}
    \item Redazione del piano di qualifica. Al suo interno saranno presenti tutte le prove effettuate ed i risultati ottenuti;
    \item Accertamento  che le attività di processo non abbiano generato degli errori.
\end{itemize}

\subsubsection{Cambio di ruoli}
Ciascun membro di \emph{duckware} ricoprirà ognuno dei ruoli sopra riportati e la rotazione avverrà secondo le seguenti regole:
\begin{itemize}
    \item Sarà obbligatorio tenere in considerazione gli interessi e gli impegni del singolo al fine di poter garantire un’esecuzione ottima;
    \item Ogni membro non potrà mai effettuare la verifica di un’opera svolta dallo stesso poiché potrebbero sorgere dei conflitti di interesse;
    \item Il trasferimento di ruolo dovrà avvenire in modo ottimale e per tali motivi sarà necessario redigere un breve documento informale all’interno di \emph{duckware} nel quale ci saranno commenti da parte del precedente assegnatario di quel ruolo.
\end{itemize}

\subsubsection{Stesura del consuntivo}
La stesura del consuntivo può essere effettuata solamente dal Responsabile di progetto della fase corrente. Le operazioni che questo deve eseguire sono riportate di seguito:

\begin{enumerate}
    \item Utilizzare \markg{Instagantt} per esportare da Asana un documento di estensione \textit{*.xlsx} (compatibile con Microsoft Office Excel) contenente le ore rendicontate nella fase corrente;
    \item Scaricare da GitLab il template del foglio di calcolo predisposto;
    \item Inserire le ore rendicontate nelle celle ottenendo la differenza di ora tra il preventivo e le ore rendicontate;
    \item Aggiungere al \textit{Piano di progetto} una sezione che mostra i risultati ottenuti al punto precedente;
    \item Creare dunque una tabella nella sezione precedentemente aggiunta che mostra la differenza tra le ore preventivate e quelle rendicontate;
\end{enumerate}

Per la revisione RR sono stati effettuati solamente i primi tre passaggi in quanto non era possibile effettuare differenze poichè mancavano le ore rendicontate con cui confrontarsi. Gli strumenti utilizzati sono dunque \textbf{Instagantt}, tool gratuito per Asana per i \markg{diagrammi di Gantt}, ed \markg{Excel}, prodotto della suite Office di Microsoft per i fogli di calcolo.

\subsection{Formazione}
\label{sec:formazione}
\subsubsection{Formazione dei membri del gruppo}
Tutti i membri di \emph{duckware} devono studiare autonomamente le tecnologie che verranno utilizzate durante il processo di creazione del progetto. Verranno realizzate delle guide informali per uso interno, preferibilmente dai membri con maggiore esperienza nel relativo campo, per facilitare la formazione dei membri di \emph{duckware} che presentano lacune.

\subsubsection{Guide e materiale usato}
La documentazione di riferimento comprende quanto indicato nella seguente lista più tutto ciò che è già stato menzionato all’interno della sottosezione Riferimenti Informativi:
\begin{itemize}
    \item \textbf{GitLab: }\href{https://gitlab.com/help}{https://gitlab.com/help}
    \item \textbf{LaTeX: }\href{https://www.latex-project.org}{https://www.latex-project.org}
    \item \textbf{Java: }\href{https://docs.oracle.com/javase/10/}{https://docs.oracle.com/javase/10/}
    \item \textbf{Android: }\href{https://developer.android.com/guide}{https://developer.android.com/guide}
    \item \textbf{Node.js: }\href{https://nodejs.org/it/docs/}{https://nodejs.org/it/docs/}
\end{itemize}
